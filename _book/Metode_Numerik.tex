\documentclass[]{book}
\usepackage{lmodern}
\usepackage{amssymb,amsmath}
\usepackage{ifxetex,ifluatex}
\usepackage{fixltx2e} % provides \textsubscript
\ifnum 0\ifxetex 1\fi\ifluatex 1\fi=0 % if pdftex
  \usepackage[T1]{fontenc}
  \usepackage[utf8]{inputenc}
\else % if luatex or xelatex
  \ifxetex
    \usepackage{mathspec}
  \else
    \usepackage{fontspec}
  \fi
  \defaultfontfeatures{Ligatures=TeX,Scale=MatchLowercase}
\fi
% use upquote if available, for straight quotes in verbatim environments
\IfFileExists{upquote.sty}{\usepackage{upquote}}{}
% use microtype if available
\IfFileExists{microtype.sty}{%
\usepackage{microtype}
\UseMicrotypeSet[protrusion]{basicmath} % disable protrusion for tt fonts
}{}
\usepackage{hyperref}
\hypersetup{unicode=true,
            pdftitle={Metode Numerik Menggunakan R Untuk Teknik Lingkungan},
            pdfauthor={Mohammad Rosidi},
            pdfborder={0 0 0},
            breaklinks=true}
\urlstyle{same}  % don't use monospace font for urls
\usepackage{natbib}
\bibliographystyle{apalike}
\usepackage{color}
\usepackage{fancyvrb}
\newcommand{\VerbBar}{|}
\newcommand{\VERB}{\Verb[commandchars=\\\{\}]}
\DefineVerbatimEnvironment{Highlighting}{Verbatim}{commandchars=\\\{\}}
% Add ',fontsize=\small' for more characters per line
\usepackage{framed}
\definecolor{shadecolor}{RGB}{248,248,248}
\newenvironment{Shaded}{\begin{snugshade}}{\end{snugshade}}
\newcommand{\AlertTok}[1]{\textcolor[rgb]{0.94,0.16,0.16}{#1}}
\newcommand{\AnnotationTok}[1]{\textcolor[rgb]{0.56,0.35,0.01}{\textbf{\textit{#1}}}}
\newcommand{\AttributeTok}[1]{\textcolor[rgb]{0.77,0.63,0.00}{#1}}
\newcommand{\BaseNTok}[1]{\textcolor[rgb]{0.00,0.00,0.81}{#1}}
\newcommand{\BuiltInTok}[1]{#1}
\newcommand{\CharTok}[1]{\textcolor[rgb]{0.31,0.60,0.02}{#1}}
\newcommand{\CommentTok}[1]{\textcolor[rgb]{0.56,0.35,0.01}{\textit{#1}}}
\newcommand{\CommentVarTok}[1]{\textcolor[rgb]{0.56,0.35,0.01}{\textbf{\textit{#1}}}}
\newcommand{\ConstantTok}[1]{\textcolor[rgb]{0.00,0.00,0.00}{#1}}
\newcommand{\ControlFlowTok}[1]{\textcolor[rgb]{0.13,0.29,0.53}{\textbf{#1}}}
\newcommand{\DataTypeTok}[1]{\textcolor[rgb]{0.13,0.29,0.53}{#1}}
\newcommand{\DecValTok}[1]{\textcolor[rgb]{0.00,0.00,0.81}{#1}}
\newcommand{\DocumentationTok}[1]{\textcolor[rgb]{0.56,0.35,0.01}{\textbf{\textit{#1}}}}
\newcommand{\ErrorTok}[1]{\textcolor[rgb]{0.64,0.00,0.00}{\textbf{#1}}}
\newcommand{\ExtensionTok}[1]{#1}
\newcommand{\FloatTok}[1]{\textcolor[rgb]{0.00,0.00,0.81}{#1}}
\newcommand{\FunctionTok}[1]{\textcolor[rgb]{0.00,0.00,0.00}{#1}}
\newcommand{\ImportTok}[1]{#1}
\newcommand{\InformationTok}[1]{\textcolor[rgb]{0.56,0.35,0.01}{\textbf{\textit{#1}}}}
\newcommand{\KeywordTok}[1]{\textcolor[rgb]{0.13,0.29,0.53}{\textbf{#1}}}
\newcommand{\NormalTok}[1]{#1}
\newcommand{\OperatorTok}[1]{\textcolor[rgb]{0.81,0.36,0.00}{\textbf{#1}}}
\newcommand{\OtherTok}[1]{\textcolor[rgb]{0.56,0.35,0.01}{#1}}
\newcommand{\PreprocessorTok}[1]{\textcolor[rgb]{0.56,0.35,0.01}{\textit{#1}}}
\newcommand{\RegionMarkerTok}[1]{#1}
\newcommand{\SpecialCharTok}[1]{\textcolor[rgb]{0.00,0.00,0.00}{#1}}
\newcommand{\SpecialStringTok}[1]{\textcolor[rgb]{0.31,0.60,0.02}{#1}}
\newcommand{\StringTok}[1]{\textcolor[rgb]{0.31,0.60,0.02}{#1}}
\newcommand{\VariableTok}[1]{\textcolor[rgb]{0.00,0.00,0.00}{#1}}
\newcommand{\VerbatimStringTok}[1]{\textcolor[rgb]{0.31,0.60,0.02}{#1}}
\newcommand{\WarningTok}[1]{\textcolor[rgb]{0.56,0.35,0.01}{\textbf{\textit{#1}}}}
\usepackage{longtable,booktabs}
\usepackage{graphicx,grffile}
\makeatletter
\def\maxwidth{\ifdim\Gin@nat@width>\linewidth\linewidth\else\Gin@nat@width\fi}
\def\maxheight{\ifdim\Gin@nat@height>\textheight\textheight\else\Gin@nat@height\fi}
\makeatother
% Scale images if necessary, so that they will not overflow the page
% margins by default, and it is still possible to overwrite the defaults
% using explicit options in \includegraphics[width, height, ...]{}
\setkeys{Gin}{width=\maxwidth,height=\maxheight,keepaspectratio}
\IfFileExists{parskip.sty}{%
\usepackage{parskip}
}{% else
\setlength{\parindent}{0pt}
\setlength{\parskip}{6pt plus 2pt minus 1pt}
}
\setlength{\emergencystretch}{3em}  % prevent overfull lines
\providecommand{\tightlist}{%
  \setlength{\itemsep}{0pt}\setlength{\parskip}{0pt}}
\setcounter{secnumdepth}{5}
% Redefines (sub)paragraphs to behave more like sections
\ifx\paragraph\undefined\else
\let\oldparagraph\paragraph
\renewcommand{\paragraph}[1]{\oldparagraph{#1}\mbox{}}
\fi
\ifx\subparagraph\undefined\else
\let\oldsubparagraph\subparagraph
\renewcommand{\subparagraph}[1]{\oldsubparagraph{#1}\mbox{}}
\fi

%%% Use protect on footnotes to avoid problems with footnotes in titles
\let\rmarkdownfootnote\footnote%
\def\footnote{\protect\rmarkdownfootnote}

%%% Change title format to be more compact
\usepackage{titling}

% Create subtitle command for use in maketitle
\providecommand{\subtitle}[1]{
  \posttitle{
    \begin{center}\large#1\end{center}
    }
}

\setlength{\droptitle}{-2em}

  \title{Metode Numerik Menggunakan R Untuk Teknik Lingkungan}
    \pretitle{\vspace{\droptitle}\centering\huge}
  \posttitle{\par}
    \author{Mohammad Rosidi}
    \preauthor{\centering\large\emph}
  \postauthor{\par}
      \predate{\centering\large\emph}
  \postdate{\par}
    \date{2019-07-28}

\usepackage{booktabs}
\usepackage{lscape}

\begin{document}
\maketitle

{
\setcounter{tocdepth}{1}
\tableofcontents
}
\hypertarget{kata-pengantar}{%
\chapter*{Kata Pengantar}\label{kata-pengantar}}
\addcontentsline{toc}{chapter}{Kata Pengantar}

\hypertarget{intro}{%
\chapter{Bahasa Pemrograman R}\label{intro}}

Dewasa ini tersedia banyak sekali \emph{software} yang dapat digunakan untuk membantu kita dalam melakukan analisa data. \emph{software} yang digunakan dapat berupa \emph{software} berbayar atau gratis.

\texttt{R} merupakan merupakan salah satu \emph{software} gratis yang sangat populer di Indonesia. Kemudahan penggunaan serta banyaknya besarnya dukungan komunitas membuat \texttt{R} menjadi salah satu bahasa pemrograman paling populer di dunia.

Paket yang disediakan untuk analisis statistika dan analisa numerik juga sangat lengkap dan terus bertambah setiap saat. Hal ini membuat \texttt{R} banyak digunakan oleh para analis data.

Pada \emph{chapter} ini penulis akan memperkenalkan kepada pembaca mengenai bahasa pemrograman \texttt{R}. Mulai dari sejarah, cara instalasi sampai dengan bagaimana kita memanfaatkan fitur dasar bantuan untuk menggali lebih jauh tentang fungsi-fungsi \texttt{R}

\hypertarget{sejarahR}{%
\section{Sejarah R}\label{sejarahR}}

\texttt{R} Merupakan bahasa yang digunakan dalam komputasi \textbf{statistik} yang pertama kali dikembangkan oleh \textbf{Ross Ihaka} dan \textbf{Robert Gentlement} di University of Auckland New Zealand yang merupakan akronim dari nama depan kedua pembuatnya. Sebelum \texttt{R} dikenal ada \texttt{S} yang dikembangkan oleh \textbf{John Chambers} dan rekan-rekan dari \textbf{Bell Laboratories} yang memiliki fungsi yang sama untuk komputasi statistik. Hal yang membedakan antara keduanya adalah \texttt{R} merupakan sistem komputasi yang bersifat gratis.Logo \texttt{R} dapat dilihat pada Gambar \ref{fig:Logo}.

\begin{figure}

{\centering \includegraphics[width=0.4\linewidth]{./images/r-icon} 

}

\caption{Logo R.}\label{fig:Logo}
\end{figure}

\texttt{R} dapat dibilang merupakan aplikasi sistem \textbf{statistik} yang kaya. Hal ini disebabkan banyak sekali paket yang dikembangkan oleh pengembang dan komunitas untuk keperluan analisa statistik seperti \emph{linear regression}, \emph{clustering}, \emph{statistical test}, dll. Selain itu, \texttt{R} juga dapat ditambahkan paket-paket lain yang dapat meningkatkan fiturnya.

Sebagai sebuah bahasa pemrograman yang banyak digunakan untuk keperluan analisa data, \texttt{R} dapat dioperasikan pada berbagai sistem operasi pada komputer. Adapun sistem operasi yang didukung antara lain: \texttt{UNIX}, \texttt{Linux}, \texttt{Windows}, dan \texttt{MacOS}.

\hypertarget{fiturR}{%
\section{Fitur dan Karakteristik R}\label{fiturR}}

\texttt{R} memiliki karakteristik yang berbeda dengan bahasa pemrograman lain seperti \texttt{C++},\texttt{python}, dll. \texttt{R} memiliki aturan/sintaks yang berbeda dengan bahasa pemrograman yang lain yang membuatnya memiliki ciri khas tersendiri dibanding bahasa pemrograman yang lain.

Beberapa ciri dan fitur pada \texttt{R} antara lain:

\begin{enumerate}
\def\labelenumi{\arabic{enumi}.}
\tightlist
\item
  \textbf{Bahasa \texttt{R} bersifat case sensitif}. maksudnya adalah dalam proses input \texttt{R} huruf besar dan kecil sangat diperhatikan. Sebagai contoh kita ingin melihat apakah objek A dan B pada sintaks berikut:
\end{enumerate}

\begin{Shaded}
\begin{Highlighting}[]
\NormalTok{A <-}\StringTok{ "Andi"}
\NormalTok{B <-}\StringTok{ "andi"}

\CommentTok{# cek kedua objek A dan B}
\NormalTok{A }\OperatorTok{==}\StringTok{ }\NormalTok{B}
\end{Highlighting}
\end{Shaded}

\begin{verbatim}
## [1] FALSE
\end{verbatim}

\begin{Shaded}
\begin{Highlighting}[]
\CommentTok{# Kesimpulan : Kedua objek berbeda}
\end{Highlighting}
\end{Shaded}

\begin{enumerate}
\def\labelenumi{\arabic{enumi}.}
\setcounter{enumi}{1}
\tightlist
\item
  \textbf{Segala sesuatu yang ada pada program \texttt{R} akan diangap sebagai objek}. konsep objek ini sama dengan bahasa pemrograma berbasis objek yang lain seperti \texttt{Java}, \texttt{C++}, \texttt{python}, dll.Perbedaannya adalah bahasa \texttt{R} relatif lebih sederhana dibandingkan bahasa pemrograman berbasis obejk yang lain.
\item
  \textbf{interpreted language atau script}. Bahasa \texttt{R} memungkinkan pengguna untuk melakukan kerja pada \texttt{R} tanpa perlu kompilasi kode program menjadi bahasa mesin.
\item
  Mendukung proses \textbf{loop}, \textbf{decision making}, dan menyediakan berbagai jenis \textbf{operstor} (aritmatika, logika, dll).
\item
  \textbf{Mendukung export dan import berbagai format file}, seperti:TXT, CSV, XLS, dll.
\item
  \textbf{Mudah ditingkatkan melalui penambahan fungsi atau paket}. Penambahan paket dapat dilakukan secara online melalui \href{https://cran.r-project.org/}{CRAN} atau melalui sumber seperti \href{https://github.com/}{github}.
\item
  \textbf{Menyedikan berbagai fungsi untuk keperluan visualisasi data}. Visualisasi data pada \texttt{R} dapat menggunakan paket bawaan atau paket lain seperti \texttt{ggplo2},\texttt{ggvis}, dll.
\end{enumerate}

\hypertarget{prosconsR}{%
\section{Kelebihan dan Kekurangan R}\label{prosconsR}}

Selain karena \texttt{R} dapat digunakan secara gratis terdapat \textbf{kelebihan} lain yang ditawarkan, antara lain:

\begin{enumerate}
\def\labelenumi{\arabic{enumi}.}
\tightlist
\item
  \textbf{Protability}. Penggunaan software dapat digunakan kapanpun tanpa terikat oleh masa berakhirnya lisensi.
\item
  \textbf{Multiplatform}. \texttt{R} bersifat \emph{Multiplatform Operating Systems}, dimana \emph{software} \texttt{R} lebih kompatibel dibanding \emph{software} statistika lainnya. Hal in berdampak pada kemudahan dalam penyesuaian jika pengguna harus berpindah sistem operasi karena \texttt{R} baik pada sistem operasi seperti \texttt{windows} akan sama pengoperasiannya dengan yang ada di \texttt{Linux} (paket yang digunakan sama).
\item
  \textbf{General} dan \textbf{Cutting-edge}. Berbagai metode statistik baik metode klasik maupun baru telah diprogram kedalam \texttt{R}. Dengan demikian \emph{software} ini dapat digunakan untuk analisis statistika dengan pendekatan klasik dan pendekatan modern.
\item
  \textbf{Programable}. Pengguna dapat memprogram metode baru atau mengembangakan modifikasi dari analisis statistika yang telah ada pada sistem \texttt{R}.
\item
  \textbf{Berbasis analisis matriks}. Bahasa \texttt{R} sangat baik digunakan untuk \emph{programming} dengan basis matriks.
\item
  Fasiltas grafik yang lengkap.
\end{enumerate}

Adapun kekurangan dari \texttt{R} antara lain:

\begin{enumerate}
\def\labelenumi{\arabic{enumi}.}
\tightlist
\item
  \textbf{Point and Click GUI}. Interaksi utama dengan \texttt{R} bersifat \emph{CLI} (\emph{Command Line Interface}), walaupun saat ini telah dikembangkan paket yang memungkinkan kita berinteraksi dengan \texttt{R} menggunakan \emph{GUI} (\emph{Graphical User Interface}) sederhana menggunakan paket \texttt{R-Commander} yang memiliki fungsi yang terbatas. \texttt{R-\ Commander} sendiri merupakan \emph{GUI} yang diciptakan dengan tujuan untuk keperluan pengajaran sehingga analisis statistik yang disediakan adalah yang klasik. Meskipun terbatas paket ini berguna jika kita membutuhkan analisis statistik sederhana dengan cara yang simpel.
\item
  \textbf{Missing statistical function}. Meskipun analisis statistika dalam \texttt{R} sudah cukup lengkap, namun tidak semua metode statistika telah diimplementasikan ke dalam \texttt{R}. Namun karena \texttt{R} merupakan \emph{lingua franca} untuk keperluan komputasi statistika modern staan ini, dapat dikatakan ketersediaan fungsi tambahan dalam bentuk paket hanya masalah waktu saja.
\end{enumerate}

\hypertarget{rstudio}{%
\section{RStudio}\label{rstudio}}

Aplikasi \texttt{R} pada dasarnya berbasis teks atau \emph{command line} sehingga pengguna harus mengetikkan perintah-perintah tertentu dan harus hapal perintah-perintahnya. Setidaknya jika kita ingin melakukan kegiatan analisa data menggunakan \texttt{R} kita harus selalu siap dengan perintah-perintah yang hendak digunakan sehingga buku manual menjadi sesuatu yang wajib adasaat berkeja dengan \texttt{R}.

Kondisi ini sering kali membingunkan bagi pengguna pemula maupun pengguna mahir yang sudah terbiasa dengan aplikasi statistik lain seperti SAS, SPSS, Minitab, dll. Alasan itulah yang menyebabkan pengembang \texttt{R} membuat berbagai \emph{frontend} untuk \texttt{R} yang berguna untuk memudahkan dalam pengoperasian \texttt{R}.

\texttt{RStudio} merupakan salah satu bentuk \emph{frontend} \texttt{R} yang cukup populer dan nyaman digunakan. Selain nyaman digunakan, \texttt{RStudio} memungkinkan kita melakukan penulisan laporan menggunakan \texttt{Rmarkdown} atau \texttt{RNotebook} serta membuat berbagai bentuk project seperti shyni, dll. Pada \texttt{R} studio juga memungkinkan kita mengatur \emph{working directory} tanpa perlu mengetikkan sintaks pada Commander, yang diperlukan hanya memilihnya di menu \texttt{RStudio}. Selain itu, kita juga dapat meng-import file berisikan data tanpa perlu mengetikkan pada Commander dengan cara memilih pada menu \texttt{Environment}.

\hypertarget{installR}{%
\section{Menginstall R dan RStudio}\label{installR}}

Pada tutorial ini hanya akan dijelaskan bagaimana menginstal \texttt{R} dan \texttt{RStudio} pada sistem operasi \texttt{windows}. Sebelum memulai menginstal sebaiknya pembaca mengunduh terlebih dahulu \emph{installer} \href{https://cran.r-project.org/bin/windows/base/}{R} dan \href{https://www.rstudio.com/products/rstudio/download/}{RStudio}.

\begin{enumerate}
\def\labelenumi{\arabic{enumi}.}
\tightlist
\item
  Jalankan proses pemasangan dengan meng-klik \emph{installer} aplikasi \texttt{R} dan \texttt{RStudio}.
\item
  Ikuti langkah proses pemasangan aplikasi yang ditampilkan dengan klik \texttt{OK} atau \texttt{Next}.
\item
  Apabila pemasangan telah dilakukan, jalankan aplikasi yang telah terpasang untuk menguji jika aplikasi telah berjalan dengan baik.
\end{enumerate}

Jendela aplikasi yang telah terpasang ditampilkan pada Gambar \ref{fig:jendela-R} dan Gambar \ref{fig:jendela-RStudio}.

\begin{figure}

{\centering \includegraphics[width=0.8\linewidth]{./images/jendela_r} 

}

\caption{Jendela R.}\label{fig:jendela-R}
\end{figure}

\begin{figure}

{\centering \includegraphics[width=0.8\linewidth]{./images/jendela_rstudio} 

}

\caption{Jendela RStudio.}\label{fig:jendela-RStudio}
\end{figure}

\begin{quote}
\textbf{Tips:} Sebaiknya install \texttt{R} terlebih dahulu sebelum \texttt{RStudio}
\end{quote}

\hypertarget{wdR}{%
\section{Working Directory}\label{wdR}}

Setiap pengguna akan bekerja pada tempat khusus yang disebut sebagai \emph{working directory}. \emph{working directory} merupakan sebuah folder dimana \texttt{R} akan membaca dan menyimpan file kerja kita. Pada pengguna \texttt{windows}, \emph{working directory} secara default pada saat pertama kali menginstall \texttt{R} terletak pada folder \texttt{c:\textbackslash{}\textbackslash{}Document}.

\hypertarget{changewdR}{%
\subsection{Mengubah Lokasi Working Directory}\label{changewdR}}

Kita dapat mengubah lokasi \emph{working directory} berdasarkan lokasi yang kita inginkan, misalnya letak data yang akan kita olah tidak ada pada folder default atau kita ingin pekerjaan kita terkait \texttt{R} dapat berlangsung pada satu folder khusus.

Berikut adalah cara mengubah \emph{working directory} pada \texttt{R}.

\begin{enumerate}
\def\labelenumi{\arabic{enumi}.}
\tightlist
\item
  Buatlah folder pada drive (kita bisa membuat folder pada selain drive c) dan namai dengan nama yang kalian inginkan. Pada tutorial ini penulis menggunakan nama folder \texttt{R}.
\item
  Jika pengguna menggunakan \texttt{RStudio}, pada menu \texttt{RStudio} pilih \textbf{Session \textgreater{} Set Working Directory \textgreater{} Chooses Directory}. Proses tersebut ditampilkan pada Gambar \ref{fig:working}
\item
  Pilih folder yang telah dibuat pada step 1 sebagai *working directory.
\end{enumerate}

\begin{quote}
\textbf{Penting:} Data atau file yang hendak dibaca selama proses kerja pada \texttt{R} harus selalu diletakkan pada working directory. Jika tidak maka data atau file tidak akan terbaca.
\end{quote}

Untuk mengecek apakah proses perubahan telah terjadi, kita dapat mengeceknya dengan menjalankan perintah berikut untuk melihat lokasi \emph{working directory} kita yang baru.

\begin{Shaded}
\begin{Highlighting}[]
\KeywordTok{getwd}\NormalTok{()}
\end{Highlighting}
\end{Shaded}

\begin{figure}

{\centering \includegraphics[width=0.8\linewidth]{./images/working} 

}

\caption{Mengubah working directory.}\label{fig:working}
\end{figure}

Selain itu kita dapat mengubah \emph{working directory} menggunakan perintah berikut:

\begin{Shaded}
\begin{Highlighting}[]
\CommentTok{# Ubah working directori pada folder R}
\KeywordTok{setwd}\NormalTok{(}\StringTok{"/Documents/R"}\NormalTok{)}
\end{Highlighting}
\end{Shaded}

\begin{quote}
\textbf{Peringatan !!!}

Pada proses pengisian lokasi folder pastikan pemisah pada lokasi folder menggunakan tanda ``/'' bukan ""
\end{quote}

\hypertarget{defaultwdR}{%
\subsection{Mengubah Lokasi Working Directory Default}\label{defaultwdR}}

Pada proses yang telah penulis jelaskan sebelumnya. Proses perubahan \emph{working directory} hanya berlaku pada saat pekerjaan tersebut dilakukan. Setelah pekerjaan selesai dan kita menjalankan kembali \texttt{R} maka \emph{working directory} akan kembali secara default pada working directory lama.

Untuk membuat lokasi default \emph{working directory} pindah, kita dapat melakukannya dengan memilih pada menu: \textbf{Tools \textgreater{} Global options \textgreater{} pada ``General'' klik pada ``Browse'' dan pilih lokasi working directory yang diinginkan}. Proses tersebut ditampilkan pada Gambar \ref{fig:default}

\begin{figure}

{\centering \includegraphics[width=0.8\linewidth]{./images/default} 

}

\caption{Merubah working directory melalui Global options.}\label{fig:default}
\end{figure}

\hypertarget{installlibraryR}{%
\section{Memasang dan Mengaktifkan Paket R}\label{installlibraryR}}

\texttt{R} dapat ditingkatkan fungsionalitasnya melalui paket-paket yang tersedia secara luas. Paket-paket ini dikembangkan secara spesifik oleh para pengembang sesuai dengan tujuan paketnya, seperti: \texttt{tidyverse} untuk \emph{data science}, \texttt{pracma} untuk analisis diferensial, dll.

Untuk menginstall paket yang kita inginkan, kita dapat menggunakan fungsi \texttt{install.packages()}. Berikut adalah contoh bagaimana cara menginstall paket \texttt{tidyverse}:

\begin{Shaded}
\begin{Highlighting}[]
\KeywordTok{install.packages}\NormalTok{(}\StringTok{"tidyverse"}\NormalTok{)}
\end{Highlighting}
\end{Shaded}

Paket yang telah diinstall tidak dapat langsung digunakan. Untuk menggunakan fungsi-fungsi yang tersedia pada paket tersebut kita perlu terlebih dahulu mengaktifkannya menggunakan fungsi \texttt{library()}. Berikut adalah contoh sintaks untuk mengaktifkan paket \texttt{tidyverse}:

\begin{Shaded}
\begin{Highlighting}[]
\KeywordTok{library}\NormalTok{(tidyverse)}
\end{Highlighting}
\end{Shaded}

Bagaimana ingin menggunakan fungsi pada paket namun tidak ingin mengaktifkan paketnya terlebih dahulu menggunakan fungsi \texttt{library()}? Untuk melakukannya kita perlu mengetikkan nama paket dikuti oleh tanda ``::'' diikuti fungsi yang ingin kita gunakan. Berikut adalah contoh penggunaan fungsi \texttt{read\_csv()} dari paket \texttt{readr} (salah satu paket yang terdapat pada kumpulan paket \texttt{tidyverse}) untuk membaca file \texttt{contoh.csv}:

\begin{Shaded}
\begin{Highlighting}[]
\NormalTok{readr}\OperatorTok{::}\KeywordTok{read_csv}\NormalTok{(}\StringTok{"contoh.csv"}\NormalTok{)}
\end{Highlighting}
\end{Shaded}

\hypertarget{helpR}{%
\section{Fasilitas Help}\label{helpR}}

Agar dapat menggunakan \texttt{R} dengan secara lebih baik, pengetahuan untuk mengakses fasilitas \emph{help} in cukup penting untuk disampaikan. Adapun cara yang dapat digunakan adalah sebagai berikut.

\hypertarget{searchhelp}{%
\subsection{Mencari Help dari Suatu Perintah Tertentu}\label{searchhelp}}

Untuk memperoleh bantuan terkait suatu perintah tertentu kita dapat menggunakan fungsi \texttt{help()}. Secara umum format yang digunakan adalah sebagai berikut:

\begin{Shaded}
\begin{Highlighting}[]
\KeywordTok{help}\NormalTok{(nama_perintah)}
\end{Highlighting}
\end{Shaded}

atau dapat juga menggunakan tanda tanya (?) pada awal \texttt{nama\_perintah} seperti berikut:

\begin{Shaded}
\begin{Highlighting}[]
\NormalTok{?nama_perintah}
\end{Highlighting}
\end{Shaded}

Misalkan kita kebingungan terkait bagaimana cara menuliskan perintah untuk menghitung rata-rata suatu vektor. Kita dapat mengetikkan perintah berikut untuk mengakses fasilitas \emph{help}.

\begin{Shaded}
\begin{Highlighting}[]
\KeywordTok{help}\NormalTok{(mean)}

\CommentTok{#atau}
\NormalTok{?mean}
\end{Highlighting}
\end{Shaded}

Perintah tersebut akan memunculkan hasil berupa dokumentasi yang ditampilkan pada Gambar \ref{fig:meandoc}.

\begin{figure}

{\centering \includegraphics[width=0.5\linewidth]{./images/meandoc} 

}

\caption{Jendela help dokumentasi fungsi mean().}\label{fig:meandoc}
\end{figure}

Keterangan pada jendela pada Gambar \ref{fig:meandoc} adalah sebagia berikut:

\begin{enumerate}
\def\labelenumi{\arabic{enumi}.}
\tightlist
\item
  Pada bagian jendela kiri atas jendela \emph{help}, diberikan keterangan nama dari perintah yang sedang ditampilkan.
\item
  Selanjutnya, pada bagian atas dokumen, ditampilkan infomasi terkait nama perintah, dan nama \emph{library} yang memuat perintah tersebut. Pada gambar diatas informasi terkait perintah dan nama \emph{library} ditunjukkan pada teks \texttt{mean\ \{base\}} yang menunjukkan perintah \texttt{mean()} pada paket (\emph{library}) \emph{base} (paket bawaan \texttt{R}).
\item
  Setiap jendela \emph{help} dari suatu perintah tertentu selanjutnya akan memuat bagian-bagian berikut:
\end{enumerate}

\begin{itemize}
\tightlist
\item
  \emph{Title}
\item
  \emph{Description} : deskripsi singkat tentang perintah.
\item
  \emph{Usage} : menampilkan sintaks perintah untuk penggunaan perintah tersebut.
\item
  \emph{Arguments} : keterangan mengenai \emph{argument/input}yang diperlukan pada perintah tersebut.
\item
  \emph{Details} : keterangan lebih lengkap lengkap tentang perintah tersebut.
\item
  \emph{Value} : keterangan tentang \emph{output} suatu perintah dapat diperoleh pada bagian ini.
\item
  \emph{Author(s)} : memberikan keterangan tentang \emph{Author} dari perintah tersebut.
\item
  \emph{References} : seringkali referensi yang dapat digunakan untuk memperoleh keterangan lebih lanjut terhadap suatu perintah ditampilkan pada bagian ini.
\item
  \emph{See also}: bagian ini berisikan daftar perintah/fungsi yang berhubungan erat dengan perintah tersebut.
\item
  \emph{Example} : berisikan contoh-contoh penggunaan perintah tersebut.
\end{itemize}

Kita juga dapat melihat contoh penggunaan dari perintah tersebut. Untuk melakukannya kita dapat menggunakan fungsi \texttt{example()}. Fungsi tersebut akan menampilkan contoh kode penerapan dari fungsi yang kita inginkan. Secara sederhana fungsi tersebut dapat dituliskan sebagai berikut:

\begin{Shaded}
\begin{Highlighting}[]
\KeywordTok{example}\NormalTok{(nama_perintah)}
\end{Highlighting}
\end{Shaded}

Untuk mengetahui contoh kode fungsi \texttt{mean()}, ketikkan sintaks berikut:

\begin{Shaded}
\begin{Highlighting}[]
\KeywordTok{example}\NormalTok{(mean)}
\end{Highlighting}
\end{Shaded}

\begin{verbatim}
## 
## mean> x <- c(0:10, 50)
## 
## mean> xm <- mean(x)
## 
## mean> c(xm, mean(x, trim = 0.10))
## [1] 8.75 5.50
\end{verbatim}

kita juga dapat mencoba kode yang dihasilkan pada console \texttt{R}. Berikut adalah contoh penerapannya:

\begin{Shaded}
\begin{Highlighting}[]
\CommentTok{# Menghitung rata-rata bilangan 1 sampai 10 dan 50}
\CommentTok{# membuat vektor}
\NormalTok{x <-}\StringTok{ }\KeywordTok{c}\NormalTok{(}\DecValTok{0}\OperatorTok{:}\DecValTok{10}\NormalTok{, }\DecValTok{50}\NormalTok{)}

\CommentTok{# Print}
\NormalTok{x}
\end{Highlighting}
\end{Shaded}

\begin{verbatim}
##  [1]  0  1  2  3  4  5  6  7  8  9 10 50
\end{verbatim}

\begin{Shaded}
\begin{Highlighting}[]
\CommentTok{# mean}
\KeywordTok{mean}\NormalTok{(x)}
\end{Highlighting}
\end{Shaded}

\begin{verbatim}
## [1] 8.75
\end{verbatim}

Pembaca dapat mencoba melakukanya sendiri dengan mengganti nilai yang telah ada serta mencoba contoh kode yang lain.

\hypertarget{generalhelp}{%
\subsection{General Help}\label{generalhelp}}

Kita juga dapat membaca beberapa dokumen manual yang ada pada \texttt{R}. Untuk melakukannya jalankan perintah berikut:

\begin{Shaded}
\begin{Highlighting}[]
\KeywordTok{help.start}\NormalTok{()}
\end{Highlighting}
\end{Shaded}

Output yang dihasilkan berupa link pada sejumlah dokumen yang dapat kita klik. Tampilan halaman yang dihasilkan disajikan pada Gambar \ref{fig:generalhelp}.

\begin{figure}

{\centering \includegraphics[width=0.5\linewidth]{./images/generalhelp} 

}

\caption{Jendela general help dokumentasi fungsi mean().}\label{fig:generalhelp}
\end{figure}

\hypertarget{othershelp}{%
\subsection{Fasilitas Help Lainnya}\label{othershelp}}

Selain yang telah penulis sebutkan sebelumnya. Kita juga dapat memanfaatkan fasilitas \emph{help} lainnya melalui fungsi \texttt{apropos()} dan \texttt{help.search()}.

\texttt{apropos\ ()}: mengembalikan daftar objek, berisi pola yang pembaca cari, dengan pencocokan sebagian. Ini berguna ketika pembaca tidak ingat persis nama fungsi yang akan digunakan. Berikut adalah contoh ketika penulis ingin mengetahui fungsi yang digunakan untuk menghitung median.

\begin{Shaded}
\begin{Highlighting}[]
\KeywordTok{apropos}\NormalTok{(}\StringTok{"med"}\NormalTok{)}
\end{Highlighting}
\end{Shaded}

\begin{verbatim}
## [1] "elNamed"        "elNamed<-"      "median"         "median.default"
## [5] "medpolish"      "runmed"
\end{verbatim}

\emph{List} yang dihasilkan berupa fungsi-fungsi yang memiliki elemen kata ``med''. Berdasarkan pencaria tersebut penulis dapat mencoba menggunakan fungsi ``median'' untuk menghitung median.

\texttt{help.search\ ()} (sebagai alternatif ??): mencari dokumentasi yang cocok dengan karakter yang diberikan dengan cara yang berbeda. Ini mengembalikan daftar fungsi yang mengandung istilah yang pembaca cari dengan deskripsi singkat dari fungsi.

Berikut adalah contoh penerapan dari fungsi tersebut:

\begin{Shaded}
\begin{Highlighting}[]
\KeywordTok{help.search}\NormalTok{(}\StringTok{"mean"}\NormalTok{)}

\CommentTok{# atau}
\NormalTok{??mean}
\end{Highlighting}
\end{Shaded}

\emph{Output} yang dihasilkan akan tampak seperti pada Gambar \ref{fig:helpsearch}.

\begin{figure}

{\centering \includegraphics[width=0.5\linewidth]{./images/helpsearch} 

}

\caption{Jendela help search dokumentasi fungsi mean().}\label{fig:helpsearch}
\end{figure}

\hypertarget{referensi}{%
\section{Referensi}\label{referensi}}

\begin{enumerate}
\def\labelenumi{\arabic{enumi}.}
\tightlist
\item
  Primartha, R. 2018. \textbf{Belajar Machine Learning Teori dan Praktik}. Penerbit Informatika : Bandung
\item
  Rosadi,D. 2016. \textbf{Analisis Statistika dengan R}. Gadjah Mada University Press: Yogyakarta
\item
  STHDA. Running RStudio and Setting Up Your Working Directory - Easy R Programming .\url{http://www.sthda.com/english/wiki/running-rstudio-and-setting-up-your-working-directory-easy-r-programming\#set-your-working-directory}
\item
  STDHA. \textbf{Getting Help With Functions In R Programming}. \url{http://www.sthda.com/english/wiki/getting-help-with-functions-in-r-programming} .
\item
  Venables, W.N. Smith D.M. and R Core Team. 2018. \textbf{An Introduction to R}. R Manuals.
\end{enumerate}

\hypertarget{calculation}{%
\chapter{Kalkulasi Menggunakan R}\label{calculation}}

Pada \emph{Chapter} ini penulis akan menjelaskan bagaimana melakukan perhitungan menggunakan \texttt{R}. Hal-hal yang akan dibahas pada \emph{chapter} ini antara lain:

\begin{itemize}
\tightlist
\item
  Operator dan fungsi dasar pada \texttt{R}
\item
  Jenis dan struktur data
\item
  Vektor (cara membuat dan melakukan operasi matematika pada vektor)
\item
  Matriks (cara membuat dan melakukan operasi matematika pada matriks)
\end{itemize}

\hypertarget{aritmetikoperator}{%
\section{Operator Aritmatik}\label{aritmetikoperator}}

Proses perhitungan akan ditangani oleh fungsi khusus. \texttt{R} akan memahami urutannya secara benar. Kecuali kita secara eksplisit menetapkan yang lain. Sebagai contoh jalankan sintaks berikut:

\begin{Shaded}
\begin{Highlighting}[]
\DecValTok{2}\OperatorTok{+}\DecValTok{4}\OperatorTok{*}\DecValTok{2}
\end{Highlighting}
\end{Shaded}

\begin{verbatim}
## [1] 10
\end{verbatim}

Bandingkan dengan sintaks berikut:

\begin{Shaded}
\begin{Highlighting}[]
\NormalTok{(}\DecValTok{2}\OperatorTok{+}\DecValTok{4}\NormalTok{)}\OperatorTok{*}\DecValTok{2}
\end{Highlighting}
\end{Shaded}

\begin{verbatim}
## [1] 12
\end{verbatim}

\begin{quote}
\textbf{Tips:} \texttt{R} dapat digunakan sebagai kalkulator
\end{quote}

Berdasarkan kedua hasil tersebut dapat disimpulkan bahwa ketika kita tidak menetapkan urutan perhitungan menggunakan tanda kurung, \texttt{R} akan secara otomatis akan menghitung terlebih dahulu perkalian atau pembangian.

Operator aritmatika yang disediakan \texttt{R} disajikan pada Tabel \ref{tab:oparitmatika}:

\begin{longtable}[]{@{}ll@{}}
\caption{\label{tab:oparitmatika} Operator Aritmatika \texttt{R}.}\tabularnewline
\toprule
\begin{minipage}[b]{0.14\columnwidth}\raggedright
\textbf{Simbol}\strut
\end{minipage} & \begin{minipage}[b]{0.80\columnwidth}\raggedright
\textbf{Keterangan}\strut
\end{minipage}\tabularnewline
\midrule
\endfirsthead
\toprule
\begin{minipage}[b]{0.14\columnwidth}\raggedright
\textbf{Simbol}\strut
\end{minipage} & \begin{minipage}[b]{0.80\columnwidth}\raggedright
\textbf{Keterangan}\strut
\end{minipage}\tabularnewline
\midrule
\endhead
\begin{minipage}[t]{0.14\columnwidth}\raggedright
+\strut
\end{minipage} & \begin{minipage}[t]{0.80\columnwidth}\raggedright
\emph{Addition}, untuk operasi penjumlahan\strut
\end{minipage}\tabularnewline
\begin{minipage}[t]{0.14\columnwidth}\raggedright
-\strut
\end{minipage} & \begin{minipage}[t]{0.80\columnwidth}\raggedright
\emph{Substraction}, untuk operasi pengurangan\strut
\end{minipage}\tabularnewline
\begin{minipage}[t]{0.14\columnwidth}\raggedright
*\strut
\end{minipage} & \begin{minipage}[t]{0.80\columnwidth}\raggedright
\emph{Multiplication}, untuk operasi pembagian\strut
\end{minipage}\tabularnewline
\begin{minipage}[t]{0.14\columnwidth}\raggedright
/\strut
\end{minipage} & \begin{minipage}[t]{0.80\columnwidth}\raggedright
\emph{Division}, untuk operasi pembagian\strut
\end{minipage}\tabularnewline
\begin{minipage}[t]{0.14\columnwidth}\raggedright
\^{}\strut
\end{minipage} & \begin{minipage}[t]{0.80\columnwidth}\raggedright
\emph{Eksponentiation}, untuk operasi pemangkatan\strut
\end{minipage}\tabularnewline
\begin{minipage}[t]{0.14\columnwidth}\raggedright
\%\%\strut
\end{minipage} & \begin{minipage}[t]{0.80\columnwidth}\raggedright
\emph{Modulus}, Untuk mencari sisa pembagian\strut
\end{minipage}\tabularnewline
\begin{minipage}[t]{0.14\columnwidth}\raggedright
\%/\%\strut
\end{minipage} & \begin{minipage}[t]{0.80\columnwidth}\raggedright
\emph{Integer}, Untuk mencari bilangan bulat hasil pembagian saja dan tanpa sisa pembagian\strut
\end{minipage}\tabularnewline
\bottomrule
\end{longtable}

Untuk lebih memahaminya berikut contoh sintaks penerapan operator tersebut.

\begin{Shaded}
\begin{Highlighting}[]
\CommentTok{# Addition}
\DecValTok{5}\OperatorTok{+}\DecValTok{3}
\end{Highlighting}
\end{Shaded}

\begin{verbatim}
## [1] 8
\end{verbatim}

\begin{Shaded}
\begin{Highlighting}[]
\CommentTok{# Substraction}
\DecValTok{5-3}
\end{Highlighting}
\end{Shaded}

\begin{verbatim}
## [1] 2
\end{verbatim}

\begin{Shaded}
\begin{Highlighting}[]
\CommentTok{# Multiplication}
\DecValTok{5}\OperatorTok{*}\DecValTok{3}
\end{Highlighting}
\end{Shaded}

\begin{verbatim}
## [1] 15
\end{verbatim}

\begin{Shaded}
\begin{Highlighting}[]
\CommentTok{# Division}
\DecValTok{5}\OperatorTok{/}\DecValTok{3}
\end{Highlighting}
\end{Shaded}

\begin{verbatim}
## [1] 1.666667
\end{verbatim}

\begin{Shaded}
\begin{Highlighting}[]
\CommentTok{# Eksponetiation}
\DecValTok{5}\OperatorTok{^}\DecValTok{3}
\end{Highlighting}
\end{Shaded}

\begin{verbatim}
## [1] 125
\end{verbatim}

\begin{Shaded}
\begin{Highlighting}[]
\CommentTok{# Modulus}
\DecValTok{5}\OperatorTok\DecValTok{3}
\end{Highlighting}
\end{Shaded}

\begin{verbatim}
## [1] 2
\end{verbatim}

\begin{Shaded}
\begin{Highlighting}[]
\CommentTok{# Integer}
\DecValTok{5}\OperatorTok\DecValTok{3}
\end{Highlighting}
\end{Shaded}

\begin{verbatim}
## [1] 1
\end{verbatim}

\begin{quote}
\textbf{Tips:} Pada \texttt{R} tanda \texttt{\#} berfungsi menambahkan keterangan untuk menjelaskan sebuah sintaks pada \texttt{R}
\end{quote}

\hypertarget{aritmaticfunction}{%
\section{Fungsi Aritmetik}\label{aritmaticfunction}}

Selain fungsi operator aritmetik, pada \texttt{R} juga telah tersedia fungsi aritmetik yang lain seperti logaritmik, ekponensial, trigonometri, dll.

\begin{enumerate}
\def\labelenumi{\arabic{enumi}.}
\tightlist
\item
  Logaritma dan eksponensial
\end{enumerate}

Untuk contoh fungsi logaritmik dan eksponensial jalankan sintaks berikut:

\begin{Shaded}
\begin{Highlighting}[]
\KeywordTok{log2}\NormalTok{(}\DecValTok{8}\NormalTok{) }\CommentTok{# logaritma basis 2 untuk 8}
\end{Highlighting}
\end{Shaded}

\begin{verbatim}
## [1] 3
\end{verbatim}

\begin{Shaded}
\begin{Highlighting}[]
\KeywordTok{log10}\NormalTok{(}\DecValTok{8}\NormalTok{) }\CommentTok{# logaritma basis 10 untuk 8}
\end{Highlighting}
\end{Shaded}

\begin{verbatim}
## [1] 0.90309
\end{verbatim}

\begin{Shaded}
\begin{Highlighting}[]
\KeywordTok{exp}\NormalTok{(}\DecValTok{8}\NormalTok{) }\CommentTok{# eksponensial 8}
\end{Highlighting}
\end{Shaded}

\begin{verbatim}
## [1] 2980.958
\end{verbatim}

\begin{enumerate}
\def\labelenumi{\arabic{enumi}.}
\setcounter{enumi}{1}
\tightlist
\item
  Fungsi trigonometri
\end{enumerate}

fungsi trigonometri yang ditampilkan seperti sin,cos, tan, dll.

\begin{Shaded}
\begin{Highlighting}[]
\KeywordTok{cos}\NormalTok{(x) }\CommentTok{# cos x}
\KeywordTok{sin}\NormalTok{(x) }\CommentTok{# Sin x}
\KeywordTok{tan}\NormalTok{(x) }\CommentTok{# Tan x}
\KeywordTok{acos}\NormalTok{(x) }\CommentTok{# arc-cos x}
\KeywordTok{asin}\NormalTok{(x) }\CommentTok{# arc-sin x}
\KeywordTok{atan}\NormalTok{(x) }\CommentTok{#arc-tan x}
\end{Highlighting}
\end{Shaded}

\begin{quote}
\textbf{Penting!!!}

x dalam fungsi trigonometri memiliki satuan radian
\end{quote}

Berikut adalah salah satu contoh penggunaannya:

\begin{Shaded}
\begin{Highlighting}[]
\KeywordTok{cos}\NormalTok{(pi)}
\end{Highlighting}
\end{Shaded}

\begin{verbatim}
## [1] -1
\end{verbatim}

Pada paket \texttt{pracma} fungsi-fungsi trigonometri dapat ditambah lagi. Fungsi-fungsi tersebut antara lain:

\begin{Shaded}
\begin{Highlighting}[]
\KeywordTok{cot}\NormalTok{(x) }\CommentTok{# cotan x}
\KeywordTok{csc}\NormalTok{(x) }\CommentTok{# cosecan x}
\KeywordTok{sec}\NormalTok{(x) }\CommentTok{# secan x}
\KeywordTok{acot}\NormalTok{(x) }\CommentTok{# arc-cotan x}
\KeywordTok{acsc}\NormalTok{(x) }\CommentTok{# arc-cosecan x}
\KeywordTok{asec}\NormalTok{(x) }\CommentTok{# arc-secan x}
\end{Highlighting}
\end{Shaded}

\begin{enumerate}
\def\labelenumi{\arabic{enumi}.}
\setcounter{enumi}{2}
\tightlist
\item
  Fungsi Hiperbolik
\end{enumerate}

fungsi hiperbolik yang tersedia antara lain:

\begin{Shaded}
\begin{Highlighting}[]
\KeywordTok{cosh}\NormalTok{(x) }
\KeywordTok{sinh}\NormalTok{(x)}
\KeywordTok{tanh}\NormalTok{(x)}
\KeywordTok{acosh}\NormalTok{(x)}
\KeywordTok{asinh}\NormalTok{(x)}
\KeywordTok{atanh}\NormalTok{(x)}
\end{Highlighting}
\end{Shaded}

Fungsi tersebut dapat ditambah lagi dari paket \texttt{pracma}. Fungsi-fungsi yang tersedia antara lain:

\begin{Shaded}
\begin{Highlighting}[]
\KeywordTok{coth}\NormalTok{(x)}
\KeywordTok{csch}\NormalTok{(x)}
\KeywordTok{sech}\NormalTok{(x)}
\KeywordTok{acoth}\NormalTok{(x)}
\KeywordTok{acsch}\NormalTok{(x)}
\KeywordTok{asech}\NormalTok{(x)}
\end{Highlighting}
\end{Shaded}

\begin{enumerate}
\def\labelenumi{\arabic{enumi}.}
\setcounter{enumi}{3}
\tightlist
\item
  Fungsi matematik lainnya
\end{enumerate}

Fungsi lainnya yang dapat digunakan adalah fungsi absolut, akar kuadrat, dll. Berikut adalah contoh sintaks penggunaan fungsi absolut dan akar kuadrat.

\begin{Shaded}
\begin{Highlighting}[]
\KeywordTok{abs}\NormalTok{(}\OperatorTok{-}\DecValTok{2}\NormalTok{) }\CommentTok{# nilai absolut -2}
\end{Highlighting}
\end{Shaded}

\begin{verbatim}
## [1] 2
\end{verbatim}

\begin{Shaded}
\begin{Highlighting}[]
\KeywordTok{sqrt}\NormalTok{(}\DecValTok{4}\NormalTok{) }\CommentTok{# akar kuadrat 4}
\end{Highlighting}
\end{Shaded}

\begin{verbatim}
## [1] 2
\end{verbatim}

\hypertarget{relationoperators}{%
\section{Operator Relasi}\label{relationoperators}}

Operator relasi digunakan untuk membandingkan satu objek dengan objek lainnya. Operator yang disediakan \texttt{R} disajikan pada Tabel \ref{tab:oprelasi}.

\begin{longtable}[]{@{}ll@{}}
\caption{\label{tab:oprelasi} Operator Relasi \texttt{R}.}\tabularnewline
\toprule
\textbf{Simbol} & \textbf{Keterangan}\tabularnewline
\midrule
\endfirsthead
\toprule
\textbf{Simbol} & \textbf{Keterangan}\tabularnewline
\midrule
\endhead
``\textgreater{}'' & Lebih besar dari\tabularnewline
``\textless{}'' & Lebih Kecil dari\tabularnewline
``=='' & Sama dengan\tabularnewline
``\textgreater{}='' & Lebih besar sama dengan\tabularnewline
``\textless{}='' & Lebih kecil sama dengan\tabularnewline
``!='' & Tidak sama dengan\tabularnewline
\bottomrule
\end{longtable}

Berikut adalah penerapan operator pada tabel tersebut:

\begin{Shaded}
\begin{Highlighting}[]
\NormalTok{x <-}\StringTok{ }\DecValTok{34}
\NormalTok{y <-}\StringTok{ }\DecValTok{35}

\CommentTok{# Operator >}
\NormalTok{x }\OperatorTok{>}\StringTok{ }\NormalTok{y}
\end{Highlighting}
\end{Shaded}

\begin{verbatim}
## [1] FALSE
\end{verbatim}

\begin{Shaded}
\begin{Highlighting}[]
\CommentTok{# Operator <}
\NormalTok{x }\OperatorTok{<}\StringTok{ }\NormalTok{y}
\end{Highlighting}
\end{Shaded}

\begin{verbatim}
## [1] TRUE
\end{verbatim}

\begin{Shaded}
\begin{Highlighting}[]
\CommentTok{# operator ==}
\NormalTok{x }\OperatorTok{==}\StringTok{ }\NormalTok{y}
\end{Highlighting}
\end{Shaded}

\begin{verbatim}
## [1] FALSE
\end{verbatim}

\begin{Shaded}
\begin{Highlighting}[]
\CommentTok{# Operator >=}
\NormalTok{x }\OperatorTok{>=}\StringTok{ }\NormalTok{y}
\end{Highlighting}
\end{Shaded}

\begin{verbatim}
## [1] FALSE
\end{verbatim}

\begin{Shaded}
\begin{Highlighting}[]
\CommentTok{# Operator <=}
\NormalTok{x }\OperatorTok{<=}\StringTok{ }\NormalTok{y}
\end{Highlighting}
\end{Shaded}

\begin{verbatim}
## [1] TRUE
\end{verbatim}

\begin{Shaded}
\begin{Highlighting}[]
\CommentTok{# Operator !=}
\NormalTok{x }\OperatorTok{!=}\StringTok{ }\NormalTok{y}
\end{Highlighting}
\end{Shaded}

\begin{verbatim}
## [1] TRUE
\end{verbatim}

\hypertarget{logicoperators}{%
\section{Operator Logika}\label{logicoperators}}

Operator logika hanya berlaku pada vektor dengan tipe logical, numeric, atau complex. Semua angka bernilai 1 akan dianggap bernilai logika \texttt{TRUE}. Operator logika yang disediakan \texttt{R} dapat dilihat pada Tabel \ref{tab:oplogika}.

\begin{longtable}[]{@{}ll@{}}
\caption{\label{tab:oplogika} Operator logika \texttt{R}.}\tabularnewline
\toprule
\textbf{Simbol} & \textbf{Keterangan}\tabularnewline
\midrule
\endfirsthead
\toprule
\textbf{Simbol} & \textbf{Keterangan}\tabularnewline
\midrule
\endhead
``\&\&'' & Operator logika AND\tabularnewline
" &\tabularnewline
``!'' & Opeartor logika NOT\tabularnewline
``\&'' & Operator logika AND element wise\tabularnewline
" & "\tabularnewline
\bottomrule
\end{longtable}

Penerapannya terdapat pada sintaks berikut:

\begin{Shaded}
\begin{Highlighting}[]
\NormalTok{v <-}\StringTok{ }\KeywordTok{c}\NormalTok{(}\OtherTok{TRUE}\NormalTok{,}\OtherTok{TRUE}\NormalTok{, }\OtherTok{FALSE}\NormalTok{)}
\NormalTok{t <-}\StringTok{ }\KeywordTok{c}\NormalTok{(}\OtherTok{FALSE}\NormalTok{,}\OtherTok{FALSE}\NormalTok{,}\OtherTok{FALSE}\NormalTok{)}

\CommentTok{# Operator &&}
\KeywordTok{print}\NormalTok{(v}\OperatorTok{&&}\NormalTok{t)}
\end{Highlighting}
\end{Shaded}

\begin{verbatim}
## [1] FALSE
\end{verbatim}

\begin{Shaded}
\begin{Highlighting}[]
\CommentTok{# Operator ||}
\KeywordTok{print}\NormalTok{(v}\OperatorTok{||}\NormalTok{t)}
\end{Highlighting}
\end{Shaded}

\begin{verbatim}
## [1] TRUE
\end{verbatim}

\begin{Shaded}
\begin{Highlighting}[]
\CommentTok{# Operator !}
\KeywordTok{print}\NormalTok{(}\OperatorTok{!}\NormalTok{v)}
\end{Highlighting}
\end{Shaded}

\begin{verbatim}
## [1] FALSE FALSE  TRUE
\end{verbatim}

\begin{Shaded}
\begin{Highlighting}[]
\CommentTok{# operator &}
\KeywordTok{print}\NormalTok{(v}\OperatorTok{&}\NormalTok{t)}
\end{Highlighting}
\end{Shaded}

\begin{verbatim}
## [1] FALSE FALSE FALSE
\end{verbatim}

\begin{Shaded}
\begin{Highlighting}[]
\CommentTok{# Operator |}
\KeywordTok{print}\NormalTok{(v}\OperatorTok{|}\NormalTok{t)}
\end{Highlighting}
\end{Shaded}

\begin{verbatim}
## [1]  TRUE  TRUE FALSE
\end{verbatim}

\begin{quote}
\textbf{Penting!!!}

\begin{itemize}
\tightlist
\item
  operator \texttt{\&} dan \texttt{\textbar{}} akan mengecek logika tiap elemen pada vektor secara berpesangan (sesuai urutan dari kiri ke kanan).
  Operator \texttt{\%\%} dan \texttt{\textbar{}\textbar{}} hanya mengecek dari kiri ke kanan pada
\item
  observasi pertama. Misal saat menggunakan \&\& jika observasi pertama \texttt{TRUE} maka observasi pertama pada vektor lainnya akan dicek, namun jika observasi pertama \texttt{FALSE} maka proses akan segera dihentikan dan menghasilkan FALSE.
\end{itemize}
\end{quote}

\hypertarget{assigningvar}{%
\section{Memasukkan Nilai Kedalam Variabel}\label{assigningvar}}

Variabel pada \texttt{R} dapat digunakan untuk menyimpan nilai. Sebagai contoh jalankan sintaks berikut:

\begin{Shaded}
\begin{Highlighting}[]
\CommentTok{# Harga sebuah lemon adalah 500 rupiah}
\NormalTok{lemon <-}\StringTok{ }\DecValTok{500}

\CommentTok{# Atau}
\DecValTok{500}\NormalTok{ ->}\StringTok{ }\NormalTok{lemon}

\CommentTok{# dapat juga menggunakan tanda "="}
\NormalTok{lemon =}\StringTok{ }\DecValTok{500}
\end{Highlighting}
\end{Shaded}

\begin{quote}
\textbf{Penting!!!}

\begin{enumerate}
\def\labelenumi{\arabic{enumi}.}
\tightlist
\item
  \texttt{R} memungkinkan penggunaan \texttt{\textless{}-},\texttt{-\textgreater{}}, atau \texttt{=} sebagai perintah pengisi nilai variabel
\item
  \texttt{R} bersifat \emph{case-sensitive}. Maksudnya adalah variabel Lemon tidak sama dengan lemon (Besar kecil huruf berpengaruh)
\end{enumerate}
\end{quote}

Untuk mengetahui nilai dari objek \texttt{lemon} kita dapat menggunakan fungsi \texttt{print()} atau mengetikkan nama objeknya secara langsung.

\begin{Shaded}
\begin{Highlighting}[]
\CommentTok{# Menggunakan fungsi print()}
\KeywordTok{print}\NormalTok{(lemon)}
\end{Highlighting}
\end{Shaded}

\begin{verbatim}
## [1] 500
\end{verbatim}

\begin{Shaded}
\begin{Highlighting}[]
\CommentTok{# Atau}
\NormalTok{lemon}
\end{Highlighting}
\end{Shaded}

\begin{verbatim}
## [1] 500
\end{verbatim}

\texttt{R} akan menyimpan variabel \texttt{lemon} sebagai objek pada memori. Sehingga kita dapat melakukan operasi terhadap objek tersebut seperti mengalikannya atau menjumlahkannya dengan bilangan lain. Sebagai contoh jalankan sintaks berikut:

\begin{Shaded}
\begin{Highlighting}[]
\CommentTok{# Operasi perkalian terhadap objek lemon}
\DecValTok{5}\OperatorTok{*}\NormalTok{lemon}
\end{Highlighting}
\end{Shaded}

\begin{verbatim}
## [1] 2500
\end{verbatim}

Kita dapat juga mengubah nilai dari objek \texttt{lemon} dengan cara menginput nilai baru terhadap objek yang sama. \texttt{R} secara otomatis akan menggatikan nilai sebelumnya. Untuk lebih memahaminya jalankan sintaks berikut:

\begin{Shaded}
\begin{Highlighting}[]
\NormalTok{lemon <-}\StringTok{ }\DecValTok{1000}

\CommentTok{# Print lemon}
\KeywordTok{print}\NormalTok{(lemon)}
\end{Highlighting}
\end{Shaded}

\begin{verbatim}
## [1] 1000
\end{verbatim}

Untuk lebih memahaminya berikut adalah sintaks untuk menghitung volume suatu objek.

\begin{Shaded}
\begin{Highlighting}[]
\CommentTok{# Dimensi objek}
\NormalTok{panjang <-}\StringTok{ }\DecValTok{10}
\NormalTok{lebar <-}\StringTok{ }\DecValTok{5}
\NormalTok{tinggi <-}\StringTok{ }\DecValTok{5}

\CommentTok{# Menghitung volume}
\NormalTok{volume <-}\StringTok{ }\NormalTok{panjang}\OperatorTok{*}\NormalTok{lebar}\OperatorTok{*}\NormalTok{tinggi}

\CommentTok{# Print objek volume}
\KeywordTok{print}\NormalTok{(volume)}
\end{Highlighting}
\end{Shaded}

\begin{verbatim}
## [1] 250
\end{verbatim}

Untuk mengetahui objek apa saja yang telah kita buat sepanjang artikel ini kita dapang menggunakan fungsi \texttt{ls()}.

\begin{Shaded}
\begin{Highlighting}[]
\KeywordTok{ls}\NormalTok{()}
\end{Highlighting}
\end{Shaded}

\begin{verbatim}
##  [1] "A"         "B"         "img1_path" "lebar"     "lemon"    
##  [6] "panjang"   "t"         "tinggi"    "v"         "volume"   
## [11] "x"         "xm"        "y"
\end{verbatim}

\begin{quote}
\textbf{Catatan:} Kumpulan objek yang telah tersimpan dalam memori disebut sebagai \textbf{workspace}
\end{quote}

Untuk menghapus objek pada memori kita dapat menggunakan fungsi \texttt{rm()}. Pada sintaks berikut penulis hendak menghapus objek \texttt{lemon} dan \texttt{volume}.

\begin{Shaded}
\begin{Highlighting}[]
\CommentTok{# Menghapus objek lemon dan volume}
\KeywordTok{rm}\NormalTok{(lemon, volume)}

\CommentTok{# Tampilkan kembali objek yang tersisa}
\KeywordTok{ls}\NormalTok{()}
\end{Highlighting}
\end{Shaded}

\begin{verbatim}
##  [1] "A"         "B"         "img1_path" "lebar"     "panjang"  
##  [6] "t"         "tinggi"    "v"         "x"         "xm"       
## [11] "y"
\end{verbatim}

\begin{quote}
\textbf{Tips:} Setiap variabel atau objek yang dibuat akan menempati sejumlah memori pada komputer sehingga jika kita bekerja dengan jumlah data yang banyak pastikan kita menghapus seluruh objek pada memori sebelum memulai kerja.
\end{quote}

\hypertarget{typedata}{%
\section{Tipe dan Struktur Data}\label{typedata}}

Data pada \texttt{R} dapat dikelompokan berdasarkan beberapa tipe. Tipe data pada \texttt{R} disajikan pada Tabel \ref{tab:tipedata}.

\begin{longtable}[]{@{}lll@{}}
\caption{\label{tab:tipedata} Tipe data \texttt{R}.}\tabularnewline
\toprule
\begin{minipage}[b]{0.11\columnwidth}\raggedright
\textbf{Tipe Data}\strut
\end{minipage} & \begin{minipage}[b]{0.19\columnwidth}\raggedright
\textbf{Contoh}\strut
\end{minipage} & \begin{minipage}[b]{0.61\columnwidth}\raggedright
\textbf{Keterangan}\strut
\end{minipage}\tabularnewline
\midrule
\endfirsthead
\toprule
\begin{minipage}[b]{0.11\columnwidth}\raggedright
\textbf{Tipe Data}\strut
\end{minipage} & \begin{minipage}[b]{0.19\columnwidth}\raggedright
\textbf{Contoh}\strut
\end{minipage} & \begin{minipage}[b]{0.61\columnwidth}\raggedright
\textbf{Keterangan}\strut
\end{minipage}\tabularnewline
\midrule
\endhead
\begin{minipage}[t]{0.11\columnwidth}\raggedright
Logical\strut
\end{minipage} & \begin{minipage}[t]{0.19\columnwidth}\raggedright
TRUE, FALSE\strut
\end{minipage} & \begin{minipage}[t]{0.61\columnwidth}\raggedright
Nilai Boolean\strut
\end{minipage}\tabularnewline
\begin{minipage}[t]{0.11\columnwidth}\raggedright
Numeric\strut
\end{minipage} & \begin{minipage}[t]{0.19\columnwidth}\raggedright
12.3, 5, 999\strut
\end{minipage} & \begin{minipage}[t]{0.61\columnwidth}\raggedright
Segala jenis angka\strut
\end{minipage}\tabularnewline
\begin{minipage}[t]{0.11\columnwidth}\raggedright
Integer\strut
\end{minipage} & \begin{minipage}[t]{0.19\columnwidth}\raggedright
23L, 97L, 3L\strut
\end{minipage} & \begin{minipage}[t]{0.61\columnwidth}\raggedright
Bilangan integer (bilangan bulat)\strut
\end{minipage}\tabularnewline
\begin{minipage}[t]{0.11\columnwidth}\raggedright
Complex\strut
\end{minipage} & \begin{minipage}[t]{0.19\columnwidth}\raggedright
2i, 3i, 9i\strut
\end{minipage} & \begin{minipage}[t]{0.61\columnwidth}\raggedright
Bilangan kompleks\strut
\end{minipage}\tabularnewline
\begin{minipage}[t]{0.11\columnwidth}\raggedright
Character\strut
\end{minipage} & \begin{minipage}[t]{0.19\columnwidth}\raggedright
`a', ``b'', ``123''\strut
\end{minipage} & \begin{minipage}[t]{0.61\columnwidth}\raggedright
Karakter dan string\strut
\end{minipage}\tabularnewline
\begin{minipage}[t]{0.11\columnwidth}\raggedright
Factor\strut
\end{minipage} & \begin{minipage}[t]{0.19\columnwidth}\raggedright
1, 0, ``Merah''\strut
\end{minipage} & \begin{minipage}[t]{0.61\columnwidth}\raggedright
Dapat berupa numerik atau string (namun pada proses akan terbaca sebagai angka)\strut
\end{minipage}\tabularnewline
\begin{minipage}[t]{0.11\columnwidth}\raggedright
Raw\strut
\end{minipage} & \begin{minipage}[t]{0.19\columnwidth}\raggedright
Identik dengan ``hello''\strut
\end{minipage} & \begin{minipage}[t]{0.61\columnwidth}\raggedright
Segala jenis data yang disimpan sebagai raw bytes\strut
\end{minipage}\tabularnewline
\bottomrule
\end{longtable}

Sintaks berikut adalah contoh dari tipe data pada \texttt{R}. Untuk mengetahui tipa data suatu objek kita dapat menggunakan perintah \texttt{class()}

\begin{Shaded}
\begin{Highlighting}[]
\CommentTok{# Logical}
\NormalTok{apel <-}\StringTok{ }\OtherTok{TRUE}
\KeywordTok{class}\NormalTok{(apel)}
\end{Highlighting}
\end{Shaded}

\begin{verbatim}
## [1] "logical"
\end{verbatim}

\begin{Shaded}
\begin{Highlighting}[]
\CommentTok{# Numeric}
\NormalTok{x <-}\StringTok{ }\FloatTok{2.3}
\KeywordTok{class}\NormalTok{(x)}
\end{Highlighting}
\end{Shaded}

\begin{verbatim}
## [1] "numeric"
\end{verbatim}

\begin{Shaded}
\begin{Highlighting}[]
\CommentTok{# Integer}
\NormalTok{y <-}\StringTok{ }\NormalTok{2L}
\KeywordTok{class}\NormalTok{(y)}
\end{Highlighting}
\end{Shaded}

\begin{verbatim}
## [1] "integer"
\end{verbatim}

\begin{Shaded}
\begin{Highlighting}[]
\CommentTok{# Compleks}
\NormalTok{z <-}\StringTok{ }\DecValTok{5}\OperatorTok{+}\NormalTok{2i}
\KeywordTok{class}\NormalTok{(z)}
\end{Highlighting}
\end{Shaded}

\begin{verbatim}
## [1] "complex"
\end{verbatim}

\begin{Shaded}
\begin{Highlighting}[]
\CommentTok{# string}
\NormalTok{w <-}\StringTok{ "saya"}
\KeywordTok{class}\NormalTok{(w)}
\end{Highlighting}
\end{Shaded}

\begin{verbatim}
## [1] "character"
\end{verbatim}

\begin{Shaded}
\begin{Highlighting}[]
\CommentTok{# Raw}
\NormalTok{xy <-}\StringTok{ }\KeywordTok{charToRaw}\NormalTok{(}\StringTok{"hello world"}\NormalTok{)}
\KeywordTok{class}\NormalTok{(xy)}
\end{Highlighting}
\end{Shaded}

\begin{verbatim}
## [1] "raw"
\end{verbatim}

Keenam jenis data tersebut disebut sebagai tipe data atomik. Hal ini disebabkan karena hanya dapat menangani satu tipe data saja. Misalnya hanya numeric atau hanya integer.

Selain menggunakan fungsi \texttt{class()}, kita dapat pula menggunakan fungsi \texttt{is\_numeric()}, \texttt{is.character()}, \texttt{is.logical()}, dan sebagainya berdasarkan jenis data apa yang ingin kita cek. Berbeda dengan fungsi \texttt{class()}, ouput yang dihasilkan pada fungsi seperti \texttt{is\_numeric()} adalah nilai Boolean sehingga fungsi ini hanya digunakan untuk mengecek apakah jenis data pada objek sama seperti yang kita pikirkan. Sebagai contoh disajikan pada sintaks berikut:

\begin{Shaded}
\begin{Highlighting}[]
\NormalTok{data <-}\StringTok{ }\DecValTok{25}

\CommentTok{# Cek apakah objek berisi data numerik}
\KeywordTok{is.numeric}\NormalTok{(data)}
\end{Highlighting}
\end{Shaded}

\begin{verbatim}
## [1] TRUE
\end{verbatim}

\begin{Shaded}
\begin{Highlighting}[]
\CommentTok{# Cek apakah objek adalah karakter}
\KeywordTok{is.character}\NormalTok{(data)}
\end{Highlighting}
\end{Shaded}

\begin{verbatim}
## [1] FALSE
\end{verbatim}

Kita juga dapat mengubah jenis data menjadi jenis lainnya seperti integer menjadi numerik atau sebaliknya. Fungsi yang digunakan adalah \texttt{as.numeric()} jika ingin mengubah suatu jenis data menjadi numerik. Fungsi lainnya juga dapat digunakan sesuai dengan kita ingin mengubah jenis data objek menjadi jenis data lainnya.

\begin{Shaded}
\begin{Highlighting}[]
\CommentTok{# Integer}
\NormalTok{apel <-}\StringTok{ }\NormalTok{2L}

\CommentTok{# Ubah menjadi numerik}
\KeywordTok{as.numeric}\NormalTok{(apel)}
\end{Highlighting}
\end{Shaded}

\begin{verbatim}
## [1] 2
\end{verbatim}

\begin{Shaded}
\begin{Highlighting}[]
\CommentTok{# Cek}
\KeywordTok{is.numeric}\NormalTok{(apel)}
\end{Highlighting}
\end{Shaded}

\begin{verbatim}
## [1] TRUE
\end{verbatim}

\begin{Shaded}
\begin{Highlighting}[]
\CommentTok{# Logical}
\NormalTok{nangka <-}\StringTok{ }\OtherTok{TRUE}

\CommentTok{# Ubah logical menjadi numeric}
\KeywordTok{as.numeric}\NormalTok{(nangka)}
\end{Highlighting}
\end{Shaded}

\begin{verbatim}
## [1] 1
\end{verbatim}

\begin{Shaded}
\begin{Highlighting}[]
\CommentTok{# Karakter}
\NormalTok{minum <-}\StringTok{ "minum"}

\CommentTok{# ubah karakter menjadi numerik}
\KeywordTok{as.numeric}\NormalTok{(minum)}
\end{Highlighting}
\end{Shaded}

\begin{verbatim}
## Warning: NAs introduced by coercion
\end{verbatim}

\begin{verbatim}
## [1] NA
\end{verbatim}

\begin{quote}
\textbf{Penting!!!}

Konversi karakter menjadi numerik akan menghasilkan output NA (\emph{not available}). \texttt{R} tidak mengetahui bagaimana cara merubah karakter menjadi bentuk numerik.
\end{quote}

Berdasarkan Tabel 2, vektor karakter dapat dibuat menggunakan tanda kurung baik \emph{double quote} ("") maupun \emph{single quote} ('').Jika pada teks yang kita tuliskan mengandung \emph{quote} maka kita harus menghentikannya menggunakan tanda ( ~). Sbegai contoh kita ingin menuliskan `\textbf{My friend's name is ``Adi''}, pada sintaks akan dituliskan:

\begin{Shaded}
\begin{Highlighting}[]
\StringTok{'My friend\textbackslash{}`s name is "Adi"'}
\end{Highlighting}
\end{Shaded}

\begin{verbatim}
## [1] "My friend`s name is \"Adi\""
\end{verbatim}

\begin{Shaded}
\begin{Highlighting}[]
\CommentTok{# Atau}

\StringTok{"My friend's name }\CharTok{\textbackslash{}"}\StringTok{Adi}\CharTok{\textbackslash{}"}\StringTok{"}
\end{Highlighting}
\end{Shaded}

\begin{verbatim}
## [1] "My friend's name \"Adi\""
\end{verbatim}

Struktur data diklasifikasikan berdasarkan dimensi data dan tie data di dalamnya (homogen atau heterogen). Klasifikasi jenis data disajikan pada Tabel \ref{tab:strukturdata}.

\begin{longtable}[]{@{}lll@{}}
\caption{\label{tab:strukturdata} Struktur data \texttt{R}.}\tabularnewline
\toprule
\textbf{Dimensi} & \textbf{Homogen} & \textbf{Heterogen}\tabularnewline
\midrule
\endfirsthead
\toprule
\textbf{Dimensi} & \textbf{Homogen} & \textbf{Heterogen}\tabularnewline
\midrule
\endhead
1d & Atomik vektor & List\tabularnewline
2d & Matriks & Dataframe\tabularnewline
nd & Array &\tabularnewline
\bottomrule
\end{longtable}

Berdasarkan Tabel tersebut dapat kita lihat bahwa objek terbagi atas dua buah struktur data yaitu homogen dan heterogen. Objek dengan struktur data homogen hanya dapat menyimpan satu tipe atau jenis data saja (numerik saja atau factor saja), sedangkan objek dengan struktur data heterogen akan dapat menyimpan berbagai jenis data.

\hypertarget{vector}{%
\section{Vektor}\label{vector}}

Vektor merupakan kombinasi berbagai nilai (numerik, karakter, logical, dan sebagainya berdasarkan jenis input data) pada objek yang sma. Pada contoh kasus berikut, pembaca akan memiliki sesuai jenis data input yaitu\textbf{vektor numerik}, \textbf{vector karakter}, \textbf{vektor logical}, dll.

\hypertarget{createvector}{%
\subsection{Membuat vektor}\label{createvector}}

Vektor dibuat dengan menggunakan fungsi \texttt{c()}(concatenate) seperti yang disajikan pada sintaks berikut:

\begin{Shaded}
\begin{Highlighting}[]
\CommentTok{# membuat vektor numerik}
\NormalTok{x <-}\StringTok{ }\KeywordTok{c}\NormalTok{(}\DecValTok{3}\NormalTok{,}\FloatTok{3.5}\NormalTok{,}\DecValTok{4}\NormalTok{,}\DecValTok{7}\NormalTok{)}
\NormalTok{x }\CommentTok{# print vektor}
\end{Highlighting}
\end{Shaded}

\begin{verbatim}
## [1] 3.0 3.5 4.0 7.0
\end{verbatim}

\begin{Shaded}
\begin{Highlighting}[]
\CommentTok{# membuat vektor karakter}
\NormalTok{y <-}\StringTok{ }\KeywordTok{c}\NormalTok{(}\StringTok{"Apel"}\NormalTok{, }\StringTok{"Jeruk"}\NormalTok{, }\StringTok{"Rambutan"}\NormalTok{, }\StringTok{"Salak"}\NormalTok{)}
\NormalTok{y }\CommentTok{# print vektor}
\end{Highlighting}
\end{Shaded}

\begin{verbatim}
## [1] "Apel"     "Jeruk"    "Rambutan" "Salak"
\end{verbatim}

\begin{Shaded}
\begin{Highlighting}[]
\CommentTok{# membuat vektor logical}
\NormalTok{t <-}\StringTok{ }\KeywordTok{c}\NormalTok{(}\StringTok{"TRUE"}\NormalTok{, }\StringTok{"FALSE"}\NormalTok{, }\StringTok{"TRUE"}\NormalTok{)}
\NormalTok{t }\CommentTok{# print vektor}
\end{Highlighting}
\end{Shaded}

\begin{verbatim}
## [1] "TRUE"  "FALSE" "TRUE"
\end{verbatim}

selain menginput nilai pada vektor, kita juga dapat memberi nama nilai setiap vektor menggunakan fungsi \texttt{names()}.

\begin{Shaded}
\begin{Highlighting}[]
\CommentTok{# Membuat vektor jumlah buah yang dibeli}
\NormalTok{Jumlah <-}\StringTok{ }\KeywordTok{c}\NormalTok{(}\DecValTok{5}\NormalTok{,}\DecValTok{5}\NormalTok{,}\DecValTok{6}\NormalTok{,}\DecValTok{7}\NormalTok{)}
\KeywordTok{names}\NormalTok{(Jumlah) <-}\StringTok{ }\KeywordTok{c}\NormalTok{(}\StringTok{"Apel"}\NormalTok{, }\StringTok{"Jeruk"}\NormalTok{, }\StringTok{"Rambutan"}\NormalTok{, }\StringTok{"Salak"}\NormalTok{)}

\CommentTok{# Atau}
\NormalTok{Jumlah <-}\StringTok{ }\KeywordTok{c}\NormalTok{(}\DataTypeTok{Apel=}\DecValTok{5}\NormalTok{, }\DataTypeTok{Jeruk=}\DecValTok{5}\NormalTok{, }\DataTypeTok{Rambutan=}\DecValTok{6}\NormalTok{, }\DataTypeTok{Salak=}\DecValTok{7}\NormalTok{)}

\CommentTok{# Print}
\NormalTok{Jumlah}
\end{Highlighting}
\end{Shaded}

\begin{verbatim}
##     Apel    Jeruk Rambutan    Salak 
##        5        5        6        7
\end{verbatim}

\begin{quote}
\textbf{Penting!!!}

Vektor hanya dapat memuat satu buah jenis data. Vektor hanya dapat mengandung jenis data numerik saja, karakter saja, dll.
\end{quote}

Untuk menentukan panjang sebuah vektor kita dapat menggunakan fungsi \texttt{lenght()}.

\begin{Shaded}
\begin{Highlighting}[]
\KeywordTok{length}\NormalTok{(Jumlah)}
\end{Highlighting}
\end{Shaded}

\begin{verbatim}
## [1] 4
\end{verbatim}

\hypertarget{missingvalue}{%
\subsection{Missing Values}\label{missingvalue}}

Seringkali nilai pada vektor kita tidak lengkap atau terdapat nilai yang hilang (\emph{missing value}) pada vektor. \emph{Missing value} pada \texttt{R} dilambangkan oleh \texttt{NA}(\emph{not available}). Berikut adalah contoh vektor dengan \emph{missing value}.

\begin{Shaded}
\begin{Highlighting}[]
\NormalTok{Jumlah <-}\StringTok{ }\KeywordTok{c}\NormalTok{(}\DataTypeTok{Apel=}\DecValTok{5}\NormalTok{, }\DataTypeTok{Jeruk=}\OtherTok{NA}\NormalTok{, }\DataTypeTok{Rambutan=}\DecValTok{6}\NormalTok{, }\DataTypeTok{Salak=}\DecValTok{7}\NormalTok{)}
\end{Highlighting}
\end{Shaded}

Untuk mengecek apakah dalam objek terdapat \emph{missing value} dapat menggunakan fungsi \texttt{is.na()}. ouput dari fungsi tersebut adalah nilai Boolean. Jika terdapat \emph{Missing value}, maka output yang dihasilkan akan memberikan nilai \texttt{TRUE}.

\begin{Shaded}
\begin{Highlighting}[]
\KeywordTok{is.na}\NormalTok{(Jumlah)}
\end{Highlighting}
\end{Shaded}

\begin{verbatim}
##     Apel    Jeruk Rambutan    Salak 
##    FALSE     TRUE    FALSE    FALSE
\end{verbatim}

\begin{quote}
\textbf{Penting!!!}

\begin{enumerate}
\def\labelenumi{\arabic{enumi}.}
\tightlist
\item
  Selain \texttt{NA} terdapat NaN (\emph{not a number}) sebagai \emph{missing value8}. Nilai tersebut muncul ketika fungsi matematika yang digunakan pada proses perhitungan tidak bekerja sebagaimana mestinya. Contoh: \texttt{0/0\ =\ NaN}
\item
  \texttt{is.na()} juga akan menghasilkan nilai \texttt{TRUE} pada NaN. Untuk membedakannya dengan \texttt{NA} dapat digunakan fungsi \texttt{is.nan()}.
\end{enumerate}
\end{quote}

\hypertarget{subsetvector}{%
\subsection{Subset Pada Vektor}\label{subsetvector}}

\emph{Subseting vector} terdiri atas tiga jenis, yaitu: \emph{positive indexing}, \emph{Negative Indexing}, dan .

\begin{itemize}
\tightlist
\item
  \textbf{Positive indexing}: memilih elemen vektor berdasarkan posisinya (indeks) dalam kurung siku.
\end{itemize}

\begin{Shaded}
\begin{Highlighting}[]
\CommentTok{# Subset vektor pada urutan kedua}
\NormalTok{Jumlah[}\DecValTok{2}\NormalTok{]}
\end{Highlighting}
\end{Shaded}

\begin{verbatim}
## Jeruk 
##    NA
\end{verbatim}

\begin{Shaded}
\begin{Highlighting}[]
\CommentTok{# Subset vektor pada urutan 2 dan 4}
\NormalTok{Jumlah[}\KeywordTok{c}\NormalTok{(}\DecValTok{2}\NormalTok{, }\DecValTok{4}\NormalTok{)]}
\end{Highlighting}
\end{Shaded}

\begin{verbatim}
## Jeruk Salak 
##    NA     7
\end{verbatim}

Selain melalui urutan (indeks), kita juga dapat melakukan subset (membuat himpunan bagian) berdasarkan nama elemen vektornya.

\begin{Shaded}
\begin{Highlighting}[]
\NormalTok{Jumlah[}\StringTok{"Jeruk"}\NormalTok{]}
\end{Highlighting}
\end{Shaded}

\begin{verbatim}
## Jeruk 
##    NA
\end{verbatim}

\begin{quote}
\textbf{Penting!!!}

Indeks pada \texttt{R} dimulai dari 1. Sehingga kolom atau elemen pertama vektor dimulai dari {[}1{]}
\end{quote}

\begin{itemize}
\tightlist
\item
  \textbf{Negative indexing}: mengecualikan (\emph{exclude}) elemen vektor.
\end{itemize}

\begin{Shaded}
\begin{Highlighting}[]
\CommentTok{# mengecualikan elemen vektor 2 dan 4}
\NormalTok{Jumlah[}\OperatorTok{-}\KeywordTok{c}\NormalTok{(}\DecValTok{2}\NormalTok{,}\DecValTok{4}\NormalTok{)]}
\end{Highlighting}
\end{Shaded}

\begin{verbatim}
##     Apel Rambutan 
##        5        6
\end{verbatim}

\begin{Shaded}
\begin{Highlighting}[]
\CommentTok{# mengecualikan elemen vektor 1 sampai 3}
\NormalTok{Jumlah[}\OperatorTok{-}\KeywordTok{c}\NormalTok{(}\DecValTok{1}\OperatorTok{:}\DecValTok{3}\NormalTok{)]}
\end{Highlighting}
\end{Shaded}

\begin{verbatim}
## Salak 
##     7
\end{verbatim}

\begin{itemize}
\tightlist
\item
  \textbf{Subset berdasarkan vektor logical}: Hanya, elemen-elemen yang nilai yang bersesuaian dalam vektor pemilihan bernilai TRUE, akan disimpan dalam subset.
\end{itemize}

\begin{quote}
\textbf{Penting!!!}

panjang vektor yang digunakan untuk subset harus sama.
\end{quote}

\begin{Shaded}
\begin{Highlighting}[]
\NormalTok{Jumlah <-}\StringTok{ }\KeywordTok{c}\NormalTok{(}\DataTypeTok{Apel=}\DecValTok{5}\NormalTok{, }\DataTypeTok{Jeruk=}\OtherTok{NA}\NormalTok{, }\DataTypeTok{Rambutan=}\DecValTok{6}\NormalTok{, }\DataTypeTok{Salak=}\DecValTok{7}\NormalTok{)}

\CommentTok{# selecting vector}
\NormalTok{merah <-}\StringTok{ }\KeywordTok{c}\NormalTok{(}\OtherTok{TRUE}\NormalTok{, }\OtherTok{FALSE}\NormalTok{, }\OtherTok{TRUE}\NormalTok{, }\OtherTok{FALSE}\NormalTok{)}

\CommentTok{# Subset}
\NormalTok{Jumlah[merah}\OperatorTok{==}\OtherTok{TRUE}\NormalTok{]}
\end{Highlighting}
\end{Shaded}

\begin{verbatim}
##     Apel Rambutan 
##        5        6
\end{verbatim}

\begin{Shaded}
\begin{Highlighting}[]
\CommentTok{# Subset untuk elemen vektor bukan missing value}
\NormalTok{Jumlah[}\OperatorTok{!}\KeywordTok{is.na}\NormalTok{(Jumlah)]}
\end{Highlighting}
\end{Shaded}

\begin{verbatim}
##     Apel Rambutan    Salak 
##        5        6        7
\end{verbatim}

\hypertarget{vectorops}{%
\subsection{Operasi Matematis Menggunakan Vektor}\label{vectorops}}

Jika pembaca melakukan operasi dengan vektor, operasi akan diterapkan ke setiap elemen vektor. Contoh disediakan pada sintaks di bawah ini:

\begin{Shaded}
\begin{Highlighting}[]
\NormalTok{pendapatan <-}\StringTok{ }\KeywordTok{c}\NormalTok{(}\DecValTok{2000}\NormalTok{, }\DecValTok{1800}\NormalTok{, }\DecValTok{2500}\NormalTok{, }\DecValTok{3000}\NormalTok{)}
\KeywordTok{names}\NormalTok{(pendapatan) <-}\StringTok{ }\KeywordTok{c}\NormalTok{(}\StringTok{"Andi"}\NormalTok{, }\StringTok{"Joni"}\NormalTok{, }\StringTok{"Lina"}\NormalTok{, }\StringTok{"Rani"}\NormalTok{)}
\NormalTok{pendapatan}
\end{Highlighting}
\end{Shaded}

\begin{verbatim}
## Andi Joni Lina Rani 
## 2000 1800 2500 3000
\end{verbatim}

\begin{Shaded}
\begin{Highlighting}[]
\CommentTok{# Kalikan pendapatan dengan 3}
\NormalTok{pendapatan}\OperatorTok{*}\DecValTok{3}
\end{Highlighting}
\end{Shaded}

\begin{verbatim}
## Andi Joni Lina Rani 
## 6000 5400 7500 9000
\end{verbatim}

Seperti yang dapat dilihat, \texttt{R} mengalikan setiap elemen dengan bilangan pengali.

Kita juga dapat mengalikan vektor dengan vektor lainnya.Contohnya disajikan pada sintaks berikut:

\begin{Shaded}
\begin{Highlighting}[]
\CommentTok{# membuat vektor dengan panjang }
\CommentTok{# sama dengan dengan vektor pendapatan}
\NormalTok{coefs <-}\StringTok{ }\KeywordTok{c}\NormalTok{(}\DecValTok{2}\NormalTok{, }\FloatTok{1.5}\NormalTok{, }\DecValTok{1}\NormalTok{, }\DecValTok{3}\NormalTok{)}

\CommentTok{# Mengalikan pendapatan dengan vektor coefs}
\NormalTok{pendapatan}\OperatorTok{*}\NormalTok{coefs}
\end{Highlighting}
\end{Shaded}

\begin{verbatim}
## Andi Joni Lina Rani 
## 4000 2700 2500 9000
\end{verbatim}

Berdasarkan sintaks tersebut dapat terlihat bahwa operasi matematik terhadap masing-masing vektor dapat berlangsung jika panjang vektornya sama.

Berikut adalah fungsi lain yang dapat digunakan pada operasi matematika vektor.

\begin{Shaded}
\begin{Highlighting}[]
\KeywordTok{max}\NormalTok{(x) }\CommentTok{# memperoleh nilai maksimum x}
\KeywordTok{min}\NormalTok{(x) }\CommentTok{# memperoleh nilai minimum x}
\KeywordTok{range}\NormalTok{(x) }\CommentTok{# memperoleh range vektor x}
\KeywordTok{length}\NormalTok{(x) }\CommentTok{# memperoleh jumlah vektor x}
\KeywordTok{sum}\NormalTok{(x) }\CommentTok{# memperoleh total penjumlahan vektor x}
\KeywordTok{prod}\NormalTok{(x) }\CommentTok{# memeperoleh produk elemen vektor x}
\KeywordTok{mean}\NormalTok{(x) }\CommentTok{# memperoleh nilai mean vektor x}
\KeywordTok{sd}\NormalTok{(x) }\CommentTok{# standar deviasi vektor x}
\KeywordTok{var}\NormalTok{(x) }\CommentTok{# varian vektor x}
\KeywordTok{sort}\NormalTok{(x) }\CommentTok{# mengurutkan elemen vektor x dari yang terbesar}
\end{Highlighting}
\end{Shaded}

Contoh penggunaan fungsi tersebut disajikan beberapa pada sintaks berikut:

\begin{Shaded}
\begin{Highlighting}[]
\CommentTok{# Menghitung range pendapatan}
\KeywordTok{range}\NormalTok{(pendapatan)}
\end{Highlighting}
\end{Shaded}

\begin{verbatim}
## [1] 1800 3000
\end{verbatim}

\begin{Shaded}
\begin{Highlighting}[]
\CommentTok{# menghitung rata-rata dan standar deviasi pendapatan}
\KeywordTok{mean}\NormalTok{(pendapatan)}
\end{Highlighting}
\end{Shaded}

\begin{verbatim}
## [1] 2325
\end{verbatim}

\begin{Shaded}
\begin{Highlighting}[]
\KeywordTok{sd}\NormalTok{(pendapatan)}
\end{Highlighting}
\end{Shaded}

\begin{verbatim}
## [1] 537.7422
\end{verbatim}

\hypertarget{seq}{%
\subsection{Membuat Deret Angka}\label{seq}}

Secara sederhana vektor merupakan deret angka. Vektor bisa jadi berupa data yang kita miliki atau sengaja kita buat untuk tujuan simulasi matematika. Urutan angka-angka ini bisa memiliki interval konstan, contoh: titik waktu pada analisis reaksi kimia, atau dapat pula intervalnya bersifat acak seperti pada simulasi Monte Carlo.

\hypertarget{regseq}{%
\subsubsection{\texorpdfstring{\emph{Regular Sequences}}{Regular Sequences}}\label{regseq}}

Operator \emph{colon} (``:'') dapat digunakan untuk membuat \emph{sequence vector}. Operator tersebut berfungsi sebagai pemisah antara nilai awal dan akhir deret bilangan. Interval nilai \emph{sequence} yang terbentuk adalah `. Berikut adalah contoh bagaimana cara membuat \emph{sequence vector} menggunakan operator \emph{colon}:

\begin{Shaded}
\begin{Highlighting}[]
\CommentTok{# vektor benilai 1 s/d 10}
\DecValTok{1}\OperatorTok{:}\DecValTok{10}
\end{Highlighting}
\end{Shaded}

\begin{verbatim}
##  [1]  1  2  3  4  5  6  7  8  9 10
\end{verbatim}

\begin{Shaded}
\begin{Highlighting}[]
\CommentTok{# vektor bernilai 10 s/d -1}
\DecValTok{10}\OperatorTok{:-}\DecValTok{1}
\end{Highlighting}
\end{Shaded}

\begin{verbatim}
##  [1] 10  9  8  7  6  5  4  3  2  1  0 -1
\end{verbatim}

Perlu diperhatikan bahwa dalam aplikasinya operator \emph{colon} memiliki prioritas tinggi untuk dilakukan komputasi terlebih dahulu dibandingkan operator matematika. Perhatikan sintaks berikut:

\begin{Shaded}
\begin{Highlighting}[]
\NormalTok{n =}\StringTok{ }\DecValTok{10}

\CommentTok{# membuat vektor bernilai 0 s/d 9}
\DecValTok{1}\OperatorTok{:}\NormalTok{n}\DecValTok{-1}
\end{Highlighting}
\end{Shaded}

\begin{verbatim}
##  [1] 0 1 2 3 4 5 6 7 8 9
\end{verbatim}

\begin{Shaded}
\begin{Highlighting}[]
\CommentTok{# membuat vektor bernilai 1 s/d 9}
\DecValTok{1}\OperatorTok{:}\NormalTok{(n}\DecValTok{-1}\NormalTok{)}
\end{Highlighting}
\end{Shaded}

\begin{verbatim}
## [1] 1 2 3 4 5 6 7 8 9
\end{verbatim}

Jika kita menginginkan interval antar angka selain 1, kita dapat menggunakan fungsi \texttt{seq()}. Format sintaks tersebut adalah sebagai berikut:

\begin{Shaded}
\begin{Highlighting}[]
\KeywordTok{seq}\NormalTok{(from, to, by)}
\end{Highlighting}
\end{Shaded}

\begin{quote}
\textbf{Catatan:}

\begin{itemize}
\tightlist
\item
  \textbf{from, to}: angka awal dan akhir atau nilai maksimum dan minimum deret bilangan yang diinginkan.
\item
  \textbf{by}: interval antar nilai
\end{itemize}
\end{quote}

Misalkan kita akan membuat deret bilangan dari 3 sampai 8 dengan interval antar deret sebesar 0,5. Berikut adalah sintaks yang digunakan:

\begin{Shaded}
\begin{Highlighting}[]
\KeywordTok{seq}\NormalTok{(}\DataTypeTok{from=}\DecValTok{3}\NormalTok{,}\DataTypeTok{to=}\DecValTok{8}\NormalTok{,}\DataTypeTok{by=}\FloatTok{0.5}\NormalTok{)}
\end{Highlighting}
\end{Shaded}

\begin{verbatim}
##  [1] 3.0 3.5 4.0 4.5 5.0 5.5 6.0 6.5 7.0 7.5 8.0
\end{verbatim}

\hypertarget{repseq}{%
\subsection{Nilai Berulang}\label{repseq}}

Fungsi \texttt{rep()} dapat digunakan untuk membuat deret dengan nilai berulang. Format fungsi tersebut adalah sebagai berikut:

\begin{Shaded}
\begin{Highlighting}[]
\KeywordTok{rep}\NormalTok{(x, times, each)}
\end{Highlighting}
\end{Shaded}

\begin{quote}
\textbf{Catatan:}

\begin{itemize}
\tightlist
\item
  \textbf{x}: nilai yang hendak dibuat berulang.
\item
  \textbf{times}: jumlah pengulangan.
\item
  \textbf{each}: argumen tambahan yang menentukan jumlah masing-masing elemen vektor akan dicetak.
\end{itemize}
\end{quote}

\begin{Shaded}
\begin{Highlighting}[]
\CommentTok{# cetak angka 5 sebanyak 5 kali}
\KeywordTok{rep}\NormalTok{(}\DataTypeTok{x=}\DecValTok{5}\NormalTok{, }\DataTypeTok{times=}\DecValTok{5}\NormalTok{)}
\end{Highlighting}
\end{Shaded}

\begin{verbatim}
## [1] 5 5 5 5 5
\end{verbatim}

\begin{Shaded}
\begin{Highlighting}[]
\CommentTok{# cetak angka 5 dan 6 sebanyak 3 kali}
\KeywordTok{rep}\NormalTok{(}\KeywordTok{c}\NormalTok{(}\DecValTok{5}\NormalTok{,}\DecValTok{6}\NormalTok{), }\DataTypeTok{times=}\DecValTok{3}\NormalTok{)}
\end{Highlighting}
\end{Shaded}

\begin{verbatim}
## [1] 5 6 5 6 5 6
\end{verbatim}

\begin{Shaded}
\begin{Highlighting}[]
\CommentTok{# cetak angka 5 dan 6 masing-masing 3 kali}
\KeywordTok{rep}\NormalTok{(}\KeywordTok{c}\NormalTok{(}\DecValTok{5}\NormalTok{,}\DecValTok{6}\NormalTok{), }\DataTypeTok{each=}\DecValTok{3}\NormalTok{)}
\end{Highlighting}
\end{Shaded}

\begin{verbatim}
## [1] 5 5 5 6 6 6
\end{verbatim}

\hypertarget{randnumb}{%
\subsection{Deret Bilangan Acak}\label{randnumb}}

Deret bilangan acak biasanya banyak digunakan dalam sebuah simulasi. \texttt{R} menyediakan fungsi untuk memproduksi bilangan-bilangan acak tersebut berdasarkan distribusi tertentu. Berikut adalah tabel rangkuman nama distribusi, fungsi, dan argumen yang digunakan:

\begin{longtable}[]{@{}lll@{}}
\caption{\label{tab:randomsequence} Ringkasan Fungsi dan Argumen Distribusi Probabilitas.}\tabularnewline
\toprule
\begin{minipage}[b]{0.07\columnwidth}\raggedright
\textbf{Distribusi}\strut
\end{minipage} & \begin{minipage}[b]{0.19\columnwidth}\raggedright
\textbf{Fungsi}\strut
\end{minipage} & \begin{minipage}[b]{0.65\columnwidth}\raggedright
\textbf{Argumen}\strut
\end{minipage}\tabularnewline
\midrule
\endfirsthead
\toprule
\begin{minipage}[b]{0.07\columnwidth}\raggedright
\textbf{Distribusi}\strut
\end{minipage} & \begin{minipage}[b]{0.19\columnwidth}\raggedright
\textbf{Fungsi}\strut
\end{minipage} & \begin{minipage}[b]{0.65\columnwidth}\raggedright
\textbf{Argumen}\strut
\end{minipage}\tabularnewline
\midrule
\endhead
\begin{minipage}[t]{0.07\columnwidth}\raggedright
Beta\strut
\end{minipage} & \begin{minipage}[t]{0.19\columnwidth}\raggedright
\texttt{rbeta(n,\ shape1,\ shape2,\ ncp\ =\ 0)}\strut
\end{minipage} & \begin{minipage}[t]{0.65\columnwidth}\raggedright
\texttt{n} = jumlah observasi; \texttt{shape1,shape2} = parameter non-negatif distribusi beta; \texttt{ncp} = \emph{non-centrality parameter}\strut
\end{minipage}\tabularnewline
\begin{minipage}[t]{0.07\columnwidth}\raggedright
Binomial\strut
\end{minipage} & \begin{minipage}[t]{0.19\columnwidth}\raggedright
\texttt{rbinom(n,\ size,\ prob)}\strut
\end{minipage} & \begin{minipage}[t]{0.65\columnwidth}\raggedright
\texttt{n}= jumlah observasi; \texttt{prob} = probabilitas sukses; \texttt{size} = jumlah percobaan\strut
\end{minipage}\tabularnewline
\begin{minipage}[t]{0.07\columnwidth}\raggedright
Cauchy\strut
\end{minipage} & \begin{minipage}[t]{0.19\columnwidth}\raggedright
\texttt{rcauchy(n,\ location\ =\ 0,\ scale\ =\ 1)}\strut
\end{minipage} & \begin{minipage}[t]{0.65\columnwidth}\raggedright
\texttt{n} = jumlah observasi; \texttt{location,\ scale} = parameter lokasi dan skala distribusi Cauchy\strut
\end{minipage}\tabularnewline
\begin{minipage}[t]{0.07\columnwidth}\raggedright
Chi-Square\strut
\end{minipage} & \begin{minipage}[t]{0.19\columnwidth}\raggedright
\texttt{rchisq(n,\ df,\ ncp\ =\ 0)}\strut
\end{minipage} & \begin{minipage}[t]{0.65\columnwidth}\raggedright
\texttt{n} = jumlah observasi; \texttt{df} = derajat kebebasan; \texttt{ncp} = \emph{non-centrality parameter}\strut
\end{minipage}\tabularnewline
\begin{minipage}[t]{0.07\columnwidth}\raggedright
Exponensial\strut
\end{minipage} & \begin{minipage}[t]{0.19\columnwidth}\raggedright
\texttt{rexp(n,\ rate\ =\ 1)}\strut
\end{minipage} & \begin{minipage}[t]{0.65\columnwidth}\raggedright
\texttt{n} = jumlah observasi; \texttt{rate} = vektor parameter \emph{rate}\strut
\end{minipage}\tabularnewline
\begin{minipage}[t]{0.07\columnwidth}\raggedright
F\strut
\end{minipage} & \begin{minipage}[t]{0.19\columnwidth}\raggedright
\texttt{rf(n,\ df1,\ df2,\ ncp)}\strut
\end{minipage} & \begin{minipage}[t]{0.65\columnwidth}\raggedright
\texttt{n} = jumlah observasi; \texttt{df1,\ df2} = derajat kebebasan; \texttt{ncp} = \emph{non-centrality parameter}\strut
\end{minipage}\tabularnewline
\begin{minipage}[t]{0.07\columnwidth}\raggedright
Gamma\strut
\end{minipage} & \begin{minipage}[t]{0.19\columnwidth}\raggedright
\texttt{rgamma(n,\ shape,\ rate\ =\ 1,\ scale\ =\ 1/rate)}\strut
\end{minipage} & \begin{minipage}[t]{0.65\columnwidth}\raggedright
\texttt{n} = jumlah observasi; \texttt{shape,\ scale} = parameter \emph{shape} dan \emph{scale}; \texttt{rate} = alternatif lain argumen \texttt{rate}\strut
\end{minipage}\tabularnewline
\begin{minipage}[t]{0.07\columnwidth}\raggedright
Geometri\strut
\end{minipage} & \begin{minipage}[t]{0.19\columnwidth}\raggedright
\texttt{rgeom(n,\ prob)}\strut
\end{minipage} & \begin{minipage}[t]{0.65\columnwidth}\raggedright
\texttt{n} = jumlah observasi; \texttt{prob} = probabilitas sukses\strut
\end{minipage}\tabularnewline
\begin{minipage}[t]{0.07\columnwidth}\raggedright
Hipergeometri\strut
\end{minipage} & \begin{minipage}[t]{0.19\columnwidth}\raggedright
\texttt{rhyper(nn,\ m,\ n,\ k)}\strut
\end{minipage} & \begin{minipage}[t]{0.65\columnwidth}\raggedright
\texttt{nn} = jumlah observasi; \texttt{m} = jumlah bola putih dalam wadah; \texttt{n} = jumlah bola hitam dalam wadah; \texttt{k} = jumlah pengambilan\strut
\end{minipage}\tabularnewline
\begin{minipage}[t]{0.07\columnwidth}\raggedright
Log-normal\strut
\end{minipage} & \begin{minipage}[t]{0.19\columnwidth}\raggedright
\texttt{rlnorm(n,\ meanlog\ =\ 0,\ sdlog\ =\ 1)}\strut
\end{minipage} & \begin{minipage}[t]{0.65\columnwidth}\raggedright
\texttt{n} = jumlah observasi; \texttt{meanlog,\ sdlog} = nilai mean dan simpangan baku dalam skala logaritmik\strut
\end{minipage}\tabularnewline
\begin{minipage}[t]{0.07\columnwidth}\raggedright
Negatif Binomial\strut
\end{minipage} & \begin{minipage}[t]{0.19\columnwidth}\raggedright
\texttt{rnbinom(n,\ size,\ prob,\ mu)}\strut
\end{minipage} & \begin{minipage}[t]{0.65\columnwidth}\raggedright
\texttt{n} = jumlah observasi; \texttt{size} = target jumlah percobaan sukses pertama kali; \texttt{prob} = probabilitas sukses; \texttt{mu} = parameterisasi alternatif melalui mean\strut
\end{minipage}\tabularnewline
\begin{minipage}[t]{0.07\columnwidth}\raggedright
Normal\strut
\end{minipage} & \begin{minipage}[t]{0.19\columnwidth}\raggedright
\texttt{rnorm(n,\ mean\ =\ 0,\ sd\ =\ 1)}\strut
\end{minipage} & \begin{minipage}[t]{0.65\columnwidth}\raggedright
\texttt{n} = jumlah observasi; \texttt{mead,\ sd} = nilai mean dan simpangan baku\strut
\end{minipage}\tabularnewline
\begin{minipage}[t]{0.07\columnwidth}\raggedright
Poisson\strut
\end{minipage} & \begin{minipage}[t]{0.19\columnwidth}\raggedright
\texttt{rpois(n,\ lambda)}\strut
\end{minipage} & \begin{minipage}[t]{0.65\columnwidth}\raggedright
\texttt{n} = jumlah observasi; \texttt{lambda} = vektor nilai mean\strut
\end{minipage}\tabularnewline
\begin{minipage}[t]{0.07\columnwidth}\raggedright
Student t\strut
\end{minipage} & \begin{minipage}[t]{0.19\columnwidth}\raggedright
\texttt{rt(n,\ df,\ ncp)}\strut
\end{minipage} & \begin{minipage}[t]{0.65\columnwidth}\raggedright
\texttt{n} = jumlah observasi; \texttt{df} = derajat kebebasan; \texttt{ncp} = \emph{non-centrality parameter}\strut
\end{minipage}\tabularnewline
\begin{minipage}[t]{0.07\columnwidth}\raggedright
Uniform\strut
\end{minipage} & \begin{minipage}[t]{0.19\columnwidth}\raggedright
\texttt{runif(n,\ min\ =\ 0,\ max\ =\ 1)}\strut
\end{minipage} & \begin{minipage}[t]{0.65\columnwidth}\raggedright
\texttt{n} = jumlah observasi; \texttt{min,\ max} = nilai maksimum dan minimum distribusi\strut
\end{minipage}\tabularnewline
\begin{minipage}[t]{0.07\columnwidth}\raggedright
Weibull\strut
\end{minipage} & \begin{minipage}[t]{0.19\columnwidth}\raggedright
\texttt{rweibull(n,\ shape,\ scale\ =\ 1)}\strut
\end{minipage} & \begin{minipage}[t]{0.65\columnwidth}\raggedright
\texttt{n} = jumlah observasi; \texttt{shape,\ scale} = parameter \emph{shape} dan \emph{scale}\strut
\end{minipage}\tabularnewline
\bottomrule
\end{longtable}

Berikut adalah contoh pembuatan vektor menggunakan bilangan acak berdistribusi normal:

\begin{Shaded}
\begin{Highlighting}[]
\NormalTok{x <-}\StringTok{ }\DecValTok{1}\OperatorTok{:}\DecValTok{6}
\NormalTok{error <-}\StringTok{ }\KeywordTok{rnorm}\NormalTok{(}\DataTypeTok{n=}\DecValTok{1}\NormalTok{, }\DataTypeTok{mean=}\DecValTok{0}\NormalTok{, }\DataTypeTok{sd=}\DecValTok{1}\NormalTok{)}

\CommentTok{# cetak x + error dengan 3 nilai signifikan}
\KeywordTok{round}\NormalTok{((x}\OperatorTok{+}\NormalTok{error), }\DecValTok{3}\NormalTok{)}
\end{Highlighting}
\end{Shaded}

\begin{verbatim}
## [1] 1.105 2.105 3.105 4.105 5.105 6.105
\end{verbatim}

\hypertarget{matriks}{%
\section{Matriks}\label{matriks}}

Matriks seperti Excel sheet yang berisi banyak baris dan kolom (kumpulan bebrapa vektor). Matriks digunakan untuk menggabungkan vektor dengan tipe yang sama, yang bisa berupa numerik, karakter, atau logis. Matriks digunakan untuk menyimpan tabel data dalam R. Baris-baris matriks pada umumnya adalah individu / pengamatan dan kolom adalah variabel.

\hypertarget{creatematrix}{%
\subsection{Membuat matriks}\label{creatematrix}}

Untuk membuat matriks kita dapat menggunakan fungsi \texttt{cbind()} atau \texttt{rbind()}. Berikut adalah contoh sintaks untuk membuat matriks.

\begin{Shaded}
\begin{Highlighting}[]
\CommentTok{# membuat vektor numerik}
\NormalTok{col1 <-}\StringTok{ }\KeywordTok{c}\NormalTok{(}\DecValTok{5}\NormalTok{, }\DecValTok{6}\NormalTok{, }\DecValTok{7}\NormalTok{, }\DecValTok{8}\NormalTok{, }\DecValTok{9}\NormalTok{)}
\NormalTok{col2 <-}\StringTok{ }\KeywordTok{c}\NormalTok{(}\DecValTok{2}\NormalTok{, }\DecValTok{4}\NormalTok{, }\DecValTok{5}\NormalTok{, }\DecValTok{9}\NormalTok{, }\DecValTok{8}\NormalTok{)}
\NormalTok{col3 <-}\StringTok{ }\KeywordTok{c}\NormalTok{(}\DecValTok{7}\NormalTok{, }\DecValTok{3}\NormalTok{, }\DecValTok{4}\NormalTok{, }\DecValTok{8}\NormalTok{, }\DecValTok{7}\NormalTok{)}

\CommentTok{# menggabungkan vektor berdasarkan kolom}
\NormalTok{my_data <-}\StringTok{ }\KeywordTok{cbind}\NormalTok{(col1, col2, col3)}
\NormalTok{my_data}
\end{Highlighting}
\end{Shaded}

\begin{verbatim}
##      col1 col2 col3
## [1,]    5    2    7
## [2,]    6    4    3
## [3,]    7    5    4
## [4,]    8    9    8
## [5,]    9    8    7
\end{verbatim}

\begin{Shaded}
\begin{Highlighting}[]
\CommentTok{# Mengubah atau menambahkan nama baris}
\KeywordTok{rownames}\NormalTok{(my_data) <-}\StringTok{ }\KeywordTok{c}\NormalTok{(}\StringTok{"row1"}\NormalTok{, }\StringTok{"row2"}\NormalTok{, }
                       \StringTok{"row3"}\NormalTok{, }\StringTok{"row4"}\NormalTok{, }
                       \StringTok{"row5"}\NormalTok{)}
\NormalTok{my_data}
\end{Highlighting}
\end{Shaded}

\begin{verbatim}
##      col1 col2 col3
## row1    5    2    7
## row2    6    4    3
## row3    7    5    4
## row4    8    9    8
## row5    9    8    7
\end{verbatim}

\begin{quote}
\textbf{Catatan:}

\begin{itemize}
\tightlist
\item
  \texttt{cbind()}: menggabungkan objek \texttt{R} berdasarkan kolom
\item
  \texttt{rbind()}: menggabungkan objek \texttt{R} berdasarkan baris
\item
  \texttt{rownames()}: mengambil atau menetapkan nama-nama baris dari objek seperti-matriks
\item
  \texttt{colnames()}: mengambil atau menetapkan nama-nama kolom dari objek seperti-matriks
  ```
\end{itemize}
\end{quote}

Kita dapat melakukan tranpose (merotasi matriks sehingga kolom menjadi baris dan sebaliknya) menggunakan fungsi \texttt{t()}. Berikut adalah contoh penerapannya:

\begin{Shaded}
\begin{Highlighting}[]
\KeywordTok{t}\NormalTok{(my_data)}
\end{Highlighting}
\end{Shaded}

\begin{verbatim}
##      row1 row2 row3 row4 row5
## col1    5    6    7    8    9
## col2    2    4    5    9    8
## col3    7    3    4    8    7
\end{verbatim}

Selain melalui pembentukan sejumlah objek vektor, kita juga dapat membuat matriks menggunakan fungsi \texttt{matrix()}. Secara sederhana fungsi tersebut dapat dituliskan sebagai berikut:

\begin{Shaded}
\begin{Highlighting}[]
\KeywordTok{matrix}\NormalTok{(}\DataTypeTok{data =} \OtherTok{NA}\NormalTok{, }\DataTypeTok{nrow =} \DecValTok{1}\NormalTok{, }\DataTypeTok{ncol =} \DecValTok{1}\NormalTok{, }\DataTypeTok{byrow =} \OtherTok{FALSE}\NormalTok{,}
       \DataTypeTok{dimnames =} \OtherTok{NULL}\NormalTok{)}
\end{Highlighting}
\end{Shaded}

\begin{quote}
\textbf{Catatan:}

\begin{itemize}
\tightlist
\item
  \texttt{data}: vektor data opsional
\item
  \texttt{nrow}, \textbf{ncol}: jumlah baris dan kolom yang diinginkan, masing-masing.
\item
  \texttt{byrow}: nilai logis. Jika FALSE (default) matriks diisi oleh kolom, jika tidak, matriks diisi oleh baris.
\item
  \texttt{dimnames}: Daftar dua vektor yang memberikan nama baris dan kolom masing-masing.
  ```
\end{itemize}
\end{quote}

Dalam kode \texttt{R} di bawah ini, data input memiliki panjang 6. Kita ingin membuat matriks dengan dua kolom. Kita tidak perlu menentukan jumlah baris (di sini \texttt{nrow\ =\ 3}). \texttt{R} akan menyimpulkan ini secara otomatis. Matriks diisi kolom demi kolom saat argumen \texttt{byrow\ =\ FALSE}. Jika kita ingin mengisi matriks dengan baris, gunakan \texttt{byrow\ =\ TRUE}. Berikut adalah contoh pembuatan matriks menggunakan fungsi \texttt{matrix()}.

\begin{Shaded}
\begin{Highlighting}[]
\NormalTok{data <-}\StringTok{ }\KeywordTok{matrix}\NormalTok{(}
           \DataTypeTok{data =} \KeywordTok{c}\NormalTok{(}\DecValTok{1}\NormalTok{,}\DecValTok{2}\NormalTok{,}\DecValTok{3}\NormalTok{, }\DecValTok{11}\NormalTok{,}\DecValTok{12}\NormalTok{,}\DecValTok{13}\NormalTok{), }
           \DataTypeTok{nrow =} \DecValTok{2}\NormalTok{, }\DataTypeTok{byrow =} \OtherTok{TRUE}\NormalTok{,}
           \DataTypeTok{dimnames =} \KeywordTok{list}\NormalTok{(}\KeywordTok{c}\NormalTok{(}\StringTok{"row1"}\NormalTok{, }\StringTok{"row2"}\NormalTok{), }
                           \KeywordTok{c}\NormalTok{(}\StringTok{"C.1"}\NormalTok{, }\StringTok{"C.2"}\NormalTok{, }\StringTok{"C.3"}\NormalTok{))}
\NormalTok{           )}
\NormalTok{data}
\end{Highlighting}
\end{Shaded}

\begin{verbatim}
##      C.1 C.2 C.3
## row1   1   2   3
## row2  11  12  13
\end{verbatim}

Untuk mengetahui dimensi dari suatu matriks, kita dapat menggunakan fungsi \texttt{ncol()} untuk mengetahui jumlah kolom matriks dan \texttt{nrow()} untuk mengetahui jumlah baris pada matriks. Berikut adalah contoh penerapannya:

\begin{Shaded}
\begin{Highlighting}[]
\CommentTok{# mengetahui jumlah kolom}
\KeywordTok{ncol}\NormalTok{(my_data)}
\end{Highlighting}
\end{Shaded}

\begin{verbatim}
## [1] 3
\end{verbatim}

\begin{Shaded}
\begin{Highlighting}[]
\CommentTok{# mengetahui jumlah baris}
\KeywordTok{nrow}\NormalTok{(my_data)}
\end{Highlighting}
\end{Shaded}

\begin{verbatim}
## [1] 5
\end{verbatim}

Jika ingin memperoleh ringkasan terkait dimensi matriks kita juga dapat mengunakan fungsi \texttt{dim()} untuk mengetahui jumlah baris dan kolom matriks. Berikut adalah contoh penerapannya:

\begin{Shaded}
\begin{Highlighting}[]
\KeywordTok{dim}\NormalTok{(my_data) }\CommentTok{# jumlah baris dan kolom}
\end{Highlighting}
\end{Shaded}

\begin{verbatim}
## [1] 5 3
\end{verbatim}

\hypertarget{subsetmatrix}{%
\subsection{Subset Pada Matriks}\label{subsetmatrix}}

Sama dengan vektor, subset juga dapat dilakukan pada matriks. Bedanya subset dilakukan berdasarkan baris dan kolom pada matriks.

\begin{itemize}
\tightlist
\item
  \textbf{Memilih baris/kolom} berdasarkan pengindeksan positif
\end{itemize}

baris atau kolom dapat diseleksi menggunakan format \texttt{data{[}row,\ col{]}}. Cara selesi ini sama dengan vektor, bedanya kita harus menetukan baris dan kolom dari data yang akan kita pilih. Berikut adalah contoh penerapannya:

\begin{Shaded}
\begin{Highlighting}[]
\CommentTok{# Pilih baris ke-2}
\NormalTok{my_data[}\DecValTok{2}\NormalTok{,]}
\end{Highlighting}
\end{Shaded}

\begin{verbatim}
## col1 col2 col3 
##    6    4    3
\end{verbatim}

\begin{Shaded}
\begin{Highlighting}[]
\CommentTok{# Pilih baris 2 sampai 4}
\NormalTok{my_data[}\DecValTok{2}\OperatorTok{:}\DecValTok{4}\NormalTok{,]}
\end{Highlighting}
\end{Shaded}

\begin{verbatim}
##      col1 col2 col3
## row2    6    4    3
## row3    7    5    4
## row4    8    9    8
\end{verbatim}

\begin{Shaded}
\begin{Highlighting}[]
\CommentTok{# Pilih baris 2 dan 4}
\NormalTok{my_data[}\KeywordTok{c}\NormalTok{(}\DecValTok{2}\NormalTok{,}\DecValTok{4}\NormalTok{),]}
\end{Highlighting}
\end{Shaded}

\begin{verbatim}
##      col1 col2 col3
## row2    6    4    3
## row4    8    9    8
\end{verbatim}

\begin{Shaded}
\begin{Highlighting}[]
\CommentTok{# Pilih baris 2 dan kolom 3}
\NormalTok{my_data[}\DecValTok{2}\NormalTok{, }\DecValTok{3}\NormalTok{]}
\end{Highlighting}
\end{Shaded}

\begin{verbatim}
## [1] 3
\end{verbatim}

\begin{itemize}
\tightlist
\item
  \textbf{Pilih berdasarkan nama baris/kolom}
\end{itemize}

Berikut adalah contoh subset berdasarkan nama baris atau kolom.

\begin{Shaded}
\begin{Highlighting}[]
\CommentTok{# Pilih baris 1 dan kolom 3}
\NormalTok{my_data[}\StringTok{"row1"}\NormalTok{,}\StringTok{"col3"}\NormalTok{]}
\end{Highlighting}
\end{Shaded}

\begin{verbatim}
## [1] 7
\end{verbatim}

\begin{Shaded}
\begin{Highlighting}[]
\CommentTok{# Pilih baris 1 sampai 4 dan kolom 3}
\NormalTok{baris <-}\StringTok{ }\KeywordTok{c}\NormalTok{(}\StringTok{"row1"}\NormalTok{,}\StringTok{"row2"}\NormalTok{,}\StringTok{"row3"}\NormalTok{)}
\NormalTok{my_data[baris, }\StringTok{"col3"}\NormalTok{]}
\end{Highlighting}
\end{Shaded}

\begin{verbatim}
## row1 row2 row3 
##    7    3    4
\end{verbatim}

\begin{itemize}
\tightlist
\item
  \textbf{Kecualikan baris/kolom} dengan pengindeksan negatif
\end{itemize}

Sama seperti vektor pengecualian data dapat dilakukan di matriks menggunakan pengindeksan negatif. Berikut cara melakukannya:

\begin{Shaded}
\begin{Highlighting}[]
\CommentTok{# Kecualikan baris 2 dan 3 serta kolom 3}
\NormalTok{my_data[}\OperatorTok{-}\KeywordTok{c}\NormalTok{(}\DecValTok{2}\NormalTok{,}\DecValTok{3}\NormalTok{), }\DecValTok{-3}\NormalTok{]}
\end{Highlighting}
\end{Shaded}

\begin{verbatim}
##      col1 col2
## row1    5    2
## row4    8    9
## row5    9    8
\end{verbatim}

\begin{itemize}
\tightlist
\item
  \textbf{Pilihan dengan logik}
\end{itemize}

Dalam kode \texttt{R} di bawah ini, misalkan kita ingin hanya menyimpan baris di mana col3\textgreater{} = 4:

\begin{Shaded}
\begin{Highlighting}[]
\NormalTok{col3 <-}\StringTok{ }\NormalTok{my_data[, }\StringTok{"col3"}\NormalTok{]}
\NormalTok{my_data[col3 }\OperatorTok{>=}\StringTok{ }\DecValTok{4}\NormalTok{, ]}
\end{Highlighting}
\end{Shaded}

\begin{verbatim}
##      col1 col2 col3
## row1    5    2    7
## row3    7    5    4
## row4    8    9    8
## row5    9    8    7
\end{verbatim}

\hypertarget{matrixcalculation}{%
\subsection{Perhitungan Menggunakan Matriks}\label{matrixcalculation}}

\_
Kita juga dapat melakukan operasi matematika pada matriks. Pada operasi matematika pada matriks proses yang terjadi bisa lebih kompleks dibanding pada vektor, dimana kita dapat melakukan operasi untuk memperoleh gambaran data pada tiap kolom atau baris.

Berikut adalah contoh operasi matematika sederhana pada matriks:

\begin{Shaded}
\begin{Highlighting}[]
\CommentTok{# mengalikan masing-masing elemen matriks dengan 2}
\NormalTok{my_data}\OperatorTok{*}\DecValTok{2}
\end{Highlighting}
\end{Shaded}

\begin{verbatim}
##      col1 col2 col3
## row1   10    4   14
## row2   12    8    6
## row3   14   10    8
## row4   16   18   16
## row5   18   16   14
\end{verbatim}

\begin{Shaded}
\begin{Highlighting}[]
\CommentTok{# memperoleh nilai log basis 2 pada masing-masing elemen matriks}
\KeywordTok{log2}\NormalTok{(my_data)}
\end{Highlighting}
\end{Shaded}

\begin{verbatim}
##          col1     col2     col3
## row1 2.321928 1.000000 2.807355
## row2 2.584963 2.000000 1.584963
## row3 2.807355 2.321928 2.000000
## row4 3.000000 3.169925 3.000000
## row5 3.169925 3.000000 2.807355
\end{verbatim}

Seperti yang telah penulis jelaskan sebelumnya, kita juga dapat melakukan operasi matematika untuk memperoleh hasil penjumlahan elemen pada tiap baris atau kolom dengan menggunakan fungsi \texttt{rowSums()} untuk baris dan \texttt{colSums()} untuk kolom.

\begin{Shaded}
\begin{Highlighting}[]
\CommentTok{# Total pada tiap kolom}
\KeywordTok{colSums}\NormalTok{(my_data)}
\end{Highlighting}
\end{Shaded}

\begin{verbatim}
## col1 col2 col3 
##   35   28   29
\end{verbatim}

\begin{Shaded}
\begin{Highlighting}[]
\CommentTok{# Total pada tiap baris}
\KeywordTok{rowSums}\NormalTok{(my_data)}
\end{Highlighting}
\end{Shaded}

\begin{verbatim}
## row1 row2 row3 row4 row5 
##   14   13   16   25   24
\end{verbatim}

Jika kita tertarik untuk mencari nilai rata-rata tiap baris arau kolom kita juga dapat menggunakan fungsi \texttt{rowMeans()} atau \texttt{colMeans()}. Berikut adalah contoh penerapannya:

\begin{Shaded}
\begin{Highlighting}[]
\CommentTok{# Rata-rata tiap baris}
\KeywordTok{rowMeans}\NormalTok{(my_data)}
\end{Highlighting}
\end{Shaded}

\begin{verbatim}
##     row1     row2     row3     row4     row5 
## 4.666667 4.333333 5.333333 8.333333 8.000000
\end{verbatim}

\begin{Shaded}
\begin{Highlighting}[]
\CommentTok{# Rata-rata tiap kolom}
\KeywordTok{colMeans}\NormalTok{(my_data)}
\end{Highlighting}
\end{Shaded}

\begin{verbatim}
## col1 col2 col3 
##  7.0  5.6  5.8
\end{verbatim}

Kita juga dapat melakukan perhitungan statistika lainnya menggunakan fungsi \texttt{apply()}. Berikut adalah format sederhananya:

\begin{Shaded}
\begin{Highlighting}[]
\KeywordTok{apply}\NormalTok{(x, MARGIN, FUN)}
\end{Highlighting}
\end{Shaded}

\begin{quote}
\textbf{Catatan:}

\begin{itemize}
\tightlist
\item
  \texttt{x} : data matriks
\item
  \texttt{MARGIN} : Nilai yang dapat digunakan adalah \texttt{1} (untuk operasi pada baris) dan \texttt{2} (untuk operasi pada kolom)
\item
  \texttt{FUN} : fungsi yang diterapkan pada baris atau kolom
\end{itemize}
\end{quote}

untuk mengetahui fungsi (\texttt{FUN}) apa saja yang dapat diterapkan pada fungsi \texttt{apply()} jalankan sintaks bantuan berikut:

\begin{Shaded}
\begin{Highlighting}[]
\KeywordTok{help}\NormalTok{(apply)}
\end{Highlighting}
\end{Shaded}

Berikut adalah contoh penerapannya:

\begin{Shaded}
\begin{Highlighting}[]
\CommentTok{# Rata-rata pada tiap baris}
\KeywordTok{apply}\NormalTok{(my_data, }\DecValTok{1}\NormalTok{, mean)}
\end{Highlighting}
\end{Shaded}

\begin{verbatim}
##     row1     row2     row3     row4     row5 
## 4.666667 4.333333 5.333333 8.333333 8.000000
\end{verbatim}

\begin{Shaded}
\begin{Highlighting}[]
\CommentTok{# Median pada tiap kolom}
\KeywordTok{apply}\NormalTok{(my_data, }\DecValTok{2}\NormalTok{, median)}
\end{Highlighting}
\end{Shaded}

\begin{verbatim}
## col1 col2 col3 
##    7    5    7
\end{verbatim}

Perhitungan lainnya tidak akan dibahas pada \emph{chapter} ini. Operasi matriks lebih lengkap selanjutnya akan dibahas pada \emph{chapter} selanjutnya.

\hypertarget{referensi-1}{%
\section{Referensi}\label{referensi-1}}

\begin{enumerate}
\def\labelenumi{\arabic{enumi}.}
\tightlist
\item
  Bloomfield, V.A. 2014. \textbf{Using R for Numerical Analysis in Science and Engineering}. CRC Press
\item
  Primartha, R. 2018. \textbf{Belajar Machine Learning Teori dan Praktik}. Penerbit Informatika : Bandung.
\item
  Rosadi,D. 2016. \textbf{Analisis Statistika dengan R}. Gadjah Mada University Press: Yogyakarta.
\item
  STHDA. \textbf{Easy R Programming Basics}. \url{http://www.sthda.com/english/wiki/easy-r-programming-basics}
\item
  The R Core Team. 2018. \textbf{R: A Language and Environment for Statistical Computing}. R Manuals.
\item
  Venables, W.N. Smith D.M. and R Core Team. 2018. \textbf{An Introduction to R}. R Manuals.
\end{enumerate}

\hypertarget{dataviz}{%
\chapter{Visualisasi Data}\label{dataviz}}

Visualisasi data merupakan bagian yang sangat penting untuk mengkomunikasikan hasil analisa yang telah kita lakukan. Selain itu, komunikasi juga membantu kita untuk memperoleh gambaran terkait data selama proses analisa data sehingga membantu kita dalam memutuskan metode analisa apa yang dapat kita terapkan pada data tersebut.

\texttt{R} memiliki library visualisasi yang sangat beragam, baik yang merupakan fungsi dasar pada \texttt{R} maupun dari sumber lain seperti ggplot dan lattice. Seluruh library visualisasi tersebut memiliki kelebihan dan kekurangannya masing-masing.

Pada \emph{chapter} ini kita tidak akan membahas seluruh library tersebut. Kita akab berfokus pada fungsi visualisasi dasar bawaan dari \texttt{R}. kita akan mempelajari mengenai jenis visualisasi data sampai dengan melakukan kustomisasi pada parameter grafik yang kita buat.

\hypertarget{plotfunc}{%
\section{Visualisasi Data Menggunakan Fungsi plot()}\label{plotfunc}}

Fungsi \texttt{plot()} merupakan fungsi umum yang digunakan untuk membuat plot pada \texttt{R}. Format dasarnya adalah sebagai berikut:

\begin{Shaded}
\begin{Highlighting}[]
\KeywordTok{plot}\NormalTok{(x, y, }\DataTypeTok{type=}\StringTok{"p"}\NormalTok{)}
\end{Highlighting}
\end{Shaded}

\begin{quote}
\textbf{Catatan:}

\begin{itemize}
\tightlist
\item
  \textbf{x dan y}: titik koordinat plot Berupa variabel dengan panjang atau jumlah observasi yang sama.
\item
  \textbf{type}: jenis grafik yang hendak dibuat. Nilai yang dapat dimasukkan antara lain:

  \begin{itemize}
  \tightlist
  \item
    type=``p'' : membuat plot titik atau scatterplot. Nilai ini merupakan default pada fungsi \texttt{plot()}.
  \item
    type=``l'' : membuat plot garis.
  \item
    type=``b'' : membuat plot titik yang terhubung dengan garis.
  \item
    type=``o'' : membuat plot titik yang ditimpa oleh garis.
  \item
    type=``h'' : membuat plot garis vertikal dari titik ke garis y=0.
  \item
    type=``s'' : membuat fungsi tangga.
  \item
    type=``n'' : tidak membuat grafik plot sama sekali, kecuali plot dari axis. Dapat digunakan untuk mengatur tampilan suatu plot utama yang diikuti oleh sekelompok plot tambahan.
  \end{itemize}
\end{itemize}
\end{quote}

Untuk lebih memahaminya berikut penulis akan sajikan contoh untuk masing-masing grafik tersebut. Berikut adalah contoh sintaks dan hasil plot yang disajikan pada Gambar \ref{fig:plot}:

\begin{Shaded}
\begin{Highlighting}[]
\CommentTok{# membuat vektor data }
\NormalTok{x <-}\StringTok{ }\KeywordTok{c}\NormalTok{(}\DecValTok{1}\OperatorTok{:}\DecValTok{10}\NormalTok{); y <-}\StringTok{ }\NormalTok{x}\OperatorTok{^}\DecValTok{2}
\end{Highlighting}
\end{Shaded}

\begin{Shaded}
\begin{Highlighting}[]
\CommentTok{# membagi jendela grafik menajdi 2 baris dan 4 kolom}
\KeywordTok{par}\NormalTok{(}\DataTypeTok{mfrow=}\KeywordTok{c}\NormalTok{(}\DecValTok{2}\NormalTok{,}\DecValTok{4}\NormalTok{))}

\CommentTok{# loop}
\NormalTok{type <-}\StringTok{ }\KeywordTok{c}\NormalTok{(}\StringTok{"p"}\NormalTok{,}\StringTok{"l"}\NormalTok{,}\StringTok{"b"}\NormalTok{,}\StringTok{"o"}\NormalTok{,}\StringTok{"h"}\NormalTok{,}\StringTok{"s"}\NormalTok{,}\StringTok{"n"}\NormalTok{)}
\ControlFlowTok{for}\NormalTok{ (i }\ControlFlowTok{in}\NormalTok{ type)\{}
  \KeywordTok{plot}\NormalTok{(x,y, }\DataTypeTok{type=}\NormalTok{ i,}
       \DataTypeTok{main=} \KeywordTok{paste}\NormalTok{(}\StringTok{"type="}\NormalTok{, i))}
\NormalTok{\}}
\end{Highlighting}
\end{Shaded}

\begin{figure}

{\centering \includegraphics[width=0.8\linewidth]{Metode_Numerik_files/figure-latex/plot-1} 

}

\caption{Plot berbagai jenis setting type}\label{fig:plot}
\end{figure}

Pada contoh selanjutnya kita akan mencoba membuat kembali data yang akan kita plotkan. Data pada contoh kali ini merupakan data suatu fungsi matematika. Berikut adalah sintaks yang digunakan:

\begin{Shaded}
\begin{Highlighting}[]
\KeywordTok{set.seed}\NormalTok{(}\DecValTok{123}\NormalTok{)}
\NormalTok{x <-}\StringTok{ }\KeywordTok{seq}\NormalTok{(}\DataTypeTok{from=}\DecValTok{0}\NormalTok{, }\DataTypeTok{to=}\DecValTok{10}\NormalTok{, }\DataTypeTok{by=}\FloatTok{0.1}\NormalTok{)}
\NormalTok{y <-}\StringTok{ }\NormalTok{x}\OperatorTok{^}\DecValTok{2}\OperatorTok{*}\KeywordTok{exp}\NormalTok{(}\OperatorTok{-}\NormalTok{x}\OperatorTok{/}\DecValTok{2}\NormalTok{)}\OperatorTok{*}\NormalTok{(}\DecValTok{1}\OperatorTok{+}\KeywordTok{rnorm}\NormalTok{(}\DataTypeTok{n=}\KeywordTok{length}\NormalTok{(x), }\DataTypeTok{mean=}\DecValTok{0}\NormalTok{, }\DataTypeTok{sd=}\FloatTok{0.05}\NormalTok{))}
\end{Highlighting}
\end{Shaded}

\begin{Shaded}
\begin{Highlighting}[]
\KeywordTok{par}\NormalTok{(}\DataTypeTok{mfrow=}\KeywordTok{c}\NormalTok{(}\DecValTok{1}\NormalTok{,}\DecValTok{2}\NormalTok{),}
    \CommentTok{# mengatur margin grafik}
    \DataTypeTok{mar=}\KeywordTok{c}\NormalTok{(}\DecValTok{4}\NormalTok{,}\DecValTok{4}\NormalTok{,}\FloatTok{1.5}\NormalTok{,}\FloatTok{1.5}\NormalTok{),}
    \CommentTok{# mengatur margin sumbu}
    \DataTypeTok{mex=}\FloatTok{0.8}\NormalTok{,}
    \CommentTok{# arah tick sumbu koordinat}
    \DataTypeTok{tcl=}\FloatTok{0.3}\NormalTok{)}
\KeywordTok{plot}\NormalTok{(x, y, }\DataTypeTok{type=}\StringTok{"l"}\NormalTok{)}
\KeywordTok{plot}\NormalTok{(x, y, }\DataTypeTok{type=}\StringTok{"o"}\NormalTok{)}
\end{Highlighting}
\end{Shaded}

\begin{figure}

{\centering \includegraphics[width=0.8\linewidth]{Metode_Numerik_files/figure-latex/plot2-1} 

}

\caption{Plot fungsi matematika}\label{fig:plot2}
\end{figure}

Fungsi lain yang dapat digunakan untuk membuat kurva suatu persamaan matematis adalah fungsi \texttt{curve()}. Berbeda dengan fungsi \texttt{plot()} yang perlu menspesifikasi objek pada sumbu x dan y, fungsi \texttt{curve()} hanya perlu menspesifikasi objek sumbu x saja. Format fungsi \texttt{curve()} adalah sebagai berikut:

\begin{Shaded}
\begin{Highlighting}[]
\KeywordTok{curve}\NormalTok{(expr, }\DataTypeTok{from =} \OtherTok{NULL}\NormalTok{, }\DataTypeTok{to =} \OtherTok{NULL}\NormalTok{, }\DataTypeTok{add =} \OtherTok{FALSE}\NormalTok{)}
\end{Highlighting}
\end{Shaded}

\begin{quote}
\textbf{Catatan:}

\begin{itemize}
\tightlist
\item
  \textbf{expr}: persamaan matematika
\item
  \textbf{from dan to}: nilai awal dan akhir (maksimum atau minimum)
\item
  \textbf{add}: nilai logik yang menentukan apakah kurva perlu ditambahkan kedalam kurva sebelumnya.
\end{itemize}
\end{quote}

Berikut adalah contoh visualisasi menggunakan fungsi \texttt{curve()}:

\begin{Shaded}
\begin{Highlighting}[]
\KeywordTok{par}\NormalTok{(}\DataTypeTok{mfrow=}\KeywordTok{c}\NormalTok{(}\DecValTok{1}\NormalTok{,}\DecValTok{2}\NormalTok{),}
    \CommentTok{# mengatur margin grafik}
    \DataTypeTok{mar=}\KeywordTok{c}\NormalTok{(}\DecValTok{4}\NormalTok{,}\DecValTok{4}\NormalTok{,}\FloatTok{1.5}\NormalTok{,}\FloatTok{1.5}\NormalTok{),}
    \CommentTok{# mengatur margin sumbu}
    \DataTypeTok{mex=}\FloatTok{0.8}\NormalTok{,}
    \CommentTok{# arah tick sumbu koordinat}
    \DataTypeTok{tcl=}\FloatTok{0.3}\NormalTok{)}

\CommentTok{# Grafik kiri}
\KeywordTok{curve}\NormalTok{(}\DataTypeTok{expr=}\NormalTok{x}\OperatorTok{^}\DecValTok{2}\OperatorTok{*}\KeywordTok{exp}\NormalTok{(}\OperatorTok{-}\NormalTok{x}\OperatorTok{/}\DecValTok{2}\NormalTok{), }
      \DataTypeTok{from=}\DecValTok{0}\NormalTok{, }\DataTypeTok{to=}\DecValTok{10}\NormalTok{)}

\CommentTok{# Grafik kanan}
\KeywordTok{plot}\NormalTok{(x, y, }\DataTypeTok{pch=}\DecValTok{19}\NormalTok{, }\DataTypeTok{cex=}\FloatTok{0.7}\NormalTok{,}
     \DataTypeTok{xlab=}\StringTok{"Waktu (detik)"}\NormalTok{,}
     \DataTypeTok{ylab=}\StringTok{"Sinyal Intensitas"}\NormalTok{)}
\KeywordTok{curve}\NormalTok{(}\DataTypeTok{expr=}\NormalTok{x}\OperatorTok{^}\DecValTok{2}\OperatorTok{*}\KeywordTok{exp}\NormalTok{(}\OperatorTok{-}\NormalTok{x}\OperatorTok{/}\DecValTok{2}\NormalTok{), }
      \DataTypeTok{from=}\DecValTok{0}\NormalTok{, }\DataTypeTok{to=}\DecValTok{10}\NormalTok{, }\DataTypeTok{add=}\OtherTok{TRUE}\NormalTok{)}
\end{Highlighting}
\end{Shaded}

\begin{figure}

{\centering \includegraphics[width=0.8\linewidth]{Metode_Numerik_files/figure-latex/curve-1} 

}

\caption{Visualisasi menggunakan fungsi curve (sebelah kiri) dan visualisasi menggunakan fungsi plot dan curve (sebelah kanan)}\label{fig:curve}
\end{figure}

\hypertarget{otherviz}{%
\section{Visualisasi Lainnya}\label{otherviz}}

Visualisasi lainnya yang sering digunakan antara lain: histogram, density plot, bar plot, dan box plot.

\hypertarget{barplot}{%
\subsection{Bar Plot}\label{barplot}}

Barplot pada \texttt{R} dapat dibuat menggunakan fungsi \texttt{barplot()}. Untuk lebih memahaminya berikut disajikan contoh barplot menggunakan dataset \texttt{VADeaths}. Untuk memuatnya jalankan sintaks berikut:

\begin{Shaded}
\begin{Highlighting}[]
\NormalTok{VADeaths}
\end{Highlighting}
\end{Shaded}

\begin{verbatim}
##       Rural Male Rural Female Urban Male Urban Female
## 50-54       11.7          8.7       15.4          8.4
## 55-59       18.1         11.7       24.3         13.6
## 60-64       26.9         20.3       37.0         19.3
## 65-69       41.0         30.9       54.6         35.1
## 70-74       66.0         54.3       71.1         50.0
\end{verbatim}

Contoh bar plot untuk variabel \texttt{Rural\ Male} disajikan pada Gambar \ref{fig:barplot}:

\begin{Shaded}
\begin{Highlighting}[]
\KeywordTok{par}\NormalTok{(}\DataTypeTok{mfrow=}\KeywordTok{c}\NormalTok{(}\DecValTok{1}\NormalTok{,}\DecValTok{2}\NormalTok{))}
\KeywordTok{barplot}\NormalTok{(VADeaths[, }\StringTok{"Rural Male"}\NormalTok{], }\DataTypeTok{main=}\StringTok{"a"}\NormalTok{)}
\KeywordTok{barplot}\NormalTok{(VADeaths[, }\StringTok{"Rural Male"}\NormalTok{], }\DataTypeTok{main=}\StringTok{"b"}\NormalTok{, }\DataTypeTok{horiz=}\OtherTok{TRUE}\NormalTok{)}
\end{Highlighting}
\end{Shaded}

\begin{figure}

{\centering \includegraphics[width=0.7\linewidth]{Metode_Numerik_files/figure-latex/barplot-1} 

}

\caption{a. bar plot vertikal; b. bar plot horizontal}\label{fig:barplot}
\end{figure}

\begin{Shaded}
\begin{Highlighting}[]
\KeywordTok{par}\NormalTok{(}\DataTypeTok{mfrow=}\KeywordTok{c}\NormalTok{(}\DecValTok{1}\NormalTok{,}\DecValTok{1}\NormalTok{))}
\end{Highlighting}
\end{Shaded}

Kita dapat mengubah warna pada masing-masing bar, baik outline bar maupun box pada bar. Selain itu kita juga dapat mengubah nama grup yang telah dihasilkan sebelumnya. Berikut sintaks untuk melakukannya dan output yang dihasilkan pada Gambar \ref{fig:barplot2}:

\begin{Shaded}
\begin{Highlighting}[]
\KeywordTok{barplot}\NormalTok{(VADeaths[, }\StringTok{"Rural Male"}\NormalTok{],}
        \CommentTok{# ubah warna ouline menjadi steelblue}
        \DataTypeTok{border=}\StringTok{"steelblue"}\NormalTok{,}
        \CommentTok{# ubah wana box}
        \DataTypeTok{col=} \KeywordTok{c}\NormalTok{(}\StringTok{"grey"}\NormalTok{, }\StringTok{"yellow"}\NormalTok{, }\StringTok{"steelblue"}\NormalTok{, }\StringTok{"green"}\NormalTok{, }\StringTok{"orange"}\NormalTok{),}
        \CommentTok{# ubah nama grup dari A sampai E}
        \DataTypeTok{names.arg =}\NormalTok{ LETTERS[}\DecValTok{1}\OperatorTok{:}\DecValTok{5}\NormalTok{],}
        \CommentTok{# ubah orientasi menajadi horizontal}
        \DataTypeTok{horiz=}\OtherTok{TRUE}\NormalTok{)}
\end{Highlighting}
\end{Shaded}

\begin{figure}

{\centering \includegraphics[width=0.7\linewidth]{Metode_Numerik_files/figure-latex/barplot2-1} 

}

\caption{Kustomisasi bar plot}\label{fig:barplot2}
\end{figure}

Untuk bar plot dengan \emph{multiple group}, tersedia dua pengaturan posisi yaitu \emph{stacked bar plot}(menunjukkan proporsi penyusun pada masing-masing grup) dan \emph{grouped bar plot}(melihat perbedaan individual pada masing-masing grup). Pada Gambar \ref{fig:barplot3} dan Gambar \ref{fig:barplot4} , disajikan kedua jenis bar plot tersebut.

\begin{Shaded}
\begin{Highlighting}[]
\CommentTok{# staked}
\KeywordTok{barplot}\NormalTok{(VADeaths,}
         \DataTypeTok{col =} \KeywordTok{c}\NormalTok{(}\StringTok{"lightblue"}\NormalTok{, }\StringTok{"mistyrose"}\NormalTok{, }\StringTok{"lightcyan"}\NormalTok{, }
                 \StringTok{"lavender"}\NormalTok{, }\StringTok{"cornsilk"}\NormalTok{),}
        \DataTypeTok{legend =} \KeywordTok{rownames}\NormalTok{(VADeaths))}
\end{Highlighting}
\end{Shaded}

\begin{figure}

{\centering \includegraphics[width=0.7\linewidth]{Metode_Numerik_files/figure-latex/barplot3-1} 

}

\caption{Stacked bar plot}\label{fig:barplot3}
\end{figure}

\begin{Shaded}
\begin{Highlighting}[]
\CommentTok{# grouped}
\KeywordTok{barplot}\NormalTok{(VADeaths,}
         \DataTypeTok{col =} \KeywordTok{c}\NormalTok{(}\StringTok{"lightblue"}\NormalTok{, }\StringTok{"mistyrose"}\NormalTok{, }\StringTok{"lightcyan"}\NormalTok{, }
                 \StringTok{"lavender"}\NormalTok{, }\StringTok{"cornsilk"}\NormalTok{),}
        \DataTypeTok{legend =} \KeywordTok{rownames}\NormalTok{(VADeaths), }\DataTypeTok{beside =} \OtherTok{TRUE}\NormalTok{)}
\end{Highlighting}
\end{Shaded}

\begin{figure}

{\centering \includegraphics[width=0.7\linewidth]{Metode_Numerik_files/figure-latex/barplot4-1} 

}

\caption{Grouped bar plot}\label{fig:barplot4}
\end{figure}

\hypertarget{histogram}{%
\subsection{Histogram dan Density Plot}\label{histogram}}

Fungsi \texttt{hist()} dapat digunakan untuk membuat histogram pada \texttt{R}. Secara sederhana fungsi tersebut didefinisikan sebagai berikut:

\begin{Shaded}
\begin{Highlighting}[]
\KeywordTok{hist}\NormalTok{(x, }\DataTypeTok{breaks=}\StringTok{"Sturges"}\NormalTok{)}
\end{Highlighting}
\end{Shaded}

\begin{quote}
\textbf{Catatan: }

\begin{itemize}
\tightlist
\item
  \textbf{x}: vektor numerik
\item
  \textbf{breaks}: \emph{breakpoints} antar sel histogram.
\end{itemize}
\end{quote}

Pada dataset \texttt{trees} akan dibuat histogram variabel \texttt{Height}. Untuk melakukannya jalankan sintaks berikut:

\begin{Shaded}
\begin{Highlighting}[]
\KeywordTok{hist}\NormalTok{(trees}\OperatorTok{$}\NormalTok{Height)}
\end{Highlighting}
\end{Shaded}

Output yang dihasilkan disajikan pada Gambar \ref{fig:hist}:

\begin{figure}

{\centering \includegraphics[width=0.7\linewidth]{Metode_Numerik_files/figure-latex/hist-1} 

}

\caption{Histogram}\label{fig:hist}
\end{figure}

Density plot pada \texttt{R} dapat dibuat menggunakan fungsi \texttt{density()}. Berbeda dengan fungsi \texttt{hist()}, fungsi ini tidak langsung menghasilkan grafik densitas. Fungsi \texttt{density()} hanya menghitung kernel densitas pada data. Densitas yang telah dihitung selanjutnya diplotkan menggunakan fungsi \texttt{plot()}. Berikut adalah sintaks dan output yang dihasilkan pada Gambar \ref{fig:dens}:

\begin{Shaded}
\begin{Highlighting}[]
\CommentTok{# menghitung kernel density}
\NormalTok{dens <-}\StringTok{ }\KeywordTok{density}\NormalTok{(trees}\OperatorTok{$}\NormalTok{Height)}

\CommentTok{# plot densitas dengan outline merah}
\KeywordTok{plot}\NormalTok{(dens,}\DataTypeTok{col=}\StringTok{"red"}\NormalTok{)}
\end{Highlighting}
\end{Shaded}

\begin{figure}

{\centering \includegraphics[width=0.7\linewidth]{Metode_Numerik_files/figure-latex/dens-1} 

}

\caption{Density plot}\label{fig:dens}
\end{figure}

Kita juga dapat menambahkan grafik densitas pada histogram sehingga mempermudah pembacaan pada histogram. Untuk melakukannya kita perlu mengubah kernel histigram dari frekuensi menjadi density dengan menambahkan argumen \texttt{freq=FALSE} pada fungsi \texttt{hist()}. Selanjutnya tambahkan fungsi \texttt{polygon()} untuk memplotkan grafik densitas. Berikut adalah sintak dan output yang dihasilkan pada Gambar \ref{fig:denshist}:

\begin{Shaded}
\begin{Highlighting}[]
\CommentTok{# menghitung kernel density}
\NormalTok{dens <-}\StringTok{ }\KeywordTok{density}\NormalTok{(trees}\OperatorTok{$}\NormalTok{Height)}

\CommentTok{# histogram}
\KeywordTok{hist}\NormalTok{(trees}\OperatorTok{$}\NormalTok{Height, }\DataTypeTok{freq=}\OtherTok{FALSE}\NormalTok{, }\DataTypeTok{col=}\StringTok{"steelblue"}\NormalTok{)}

\CommentTok{# tambahkan density plot}
\KeywordTok{polygon}\NormalTok{(dens, }\DataTypeTok{border=}\StringTok{"red"}\NormalTok{)}
\end{Highlighting}
\end{Shaded}

\begin{figure}

{\centering \includegraphics[width=0.7\linewidth]{Metode_Numerik_files/figure-latex/denshist-1} 

}

\caption{Density plot dan histogram}\label{fig:denshist}
\end{figure}

\hypertarget{boxplot}{%
\subsection{Box plot}\label{boxplot}}

Box plot pada \texttt{R} dapat dibuat menggunakan fungsi \texttt{boxplot()}. Berikut adalah sintaks untuk membuat boxplot variabel \texttt{Sepal.Lenght} pada dataset \texttt{iris} dan output yang dihasilkan pada Gambar \ref{fig:boxplot}:

\begin{Shaded}
\begin{Highlighting}[]
\KeywordTok{boxplot}\NormalTok{(iris}\OperatorTok{$}\NormalTok{Sepal.Length)}
\end{Highlighting}
\end{Shaded}

\begin{figure}

{\centering \includegraphics[width=0.7\linewidth]{Metode_Numerik_files/figure-latex/boxplot-1} 

}

\caption{Boxplot variabel Sepal.Length}\label{fig:boxplot}
\end{figure}

Boxplot juga dapat dibuat berdasarkan variabel factor. Hal ini berguna untuk melihat perbedaan ditribusi data pada masing-masing grup. Pada sintaks berikut dibuat boxplot berdasarkan variabel \texttt{Species}. Output yang dihasilkan disajikan pada Gambar \ref{fig:boxplot2}:

\begin{Shaded}
\begin{Highlighting}[]
\KeywordTok{boxplot}\NormalTok{(iris}\OperatorTok{$}\NormalTok{Sepal.Length}\OperatorTok{~}\NormalTok{iris}\OperatorTok{$}\NormalTok{Species)}
\end{Highlighting}
\end{Shaded}

\begin{figure}

{\centering \includegraphics[width=0.7\linewidth]{Metode_Numerik_files/figure-latex/boxplot2-1} 

}

\caption{Boxplot berdasarkan variabel species}\label{fig:boxplot2}
\end{figure}

Kita juga dapat mengubah warna outline dan box pada boxplot. Berikut adalah contoh sintaks yang digunakan untuk melakukannya dan output yang dihasilkan disajikan pada Gambar \ref{fig:boxplot3}:

\begin{Shaded}
\begin{Highlighting}[]
\KeywordTok{boxplot}\NormalTok{(iris}\OperatorTok{$}\NormalTok{Sepal.Length}\OperatorTok{~}\NormalTok{iris}\OperatorTok{$}\NormalTok{Species,}
        \CommentTok{# ubah warna outline menjadi steelblue}
        \DataTypeTok{border =} \StringTok{"steelblue"}\NormalTok{,}
        \CommentTok{# ubah warna box berdasarkan grup}
        \DataTypeTok{col=} \KeywordTok{c}\NormalTok{(}\StringTok{"#999999"}\NormalTok{, }\StringTok{"#E69F00"}\NormalTok{, }\StringTok{"#56B4E9"}\NormalTok{))}
\end{Highlighting}
\end{Shaded}

\begin{figure}

{\centering \includegraphics[width=0.7\linewidth]{Metode_Numerik_files/figure-latex/boxplot3-1} 

}

\caption{Boxplot dengan warna berdasarkan spesies}\label{fig:boxplot3}
\end{figure}

Kita juga dapat membuat boxplot pada \emph{multiple group}. Data yang digunakan untuk contoh tersebut adalah dataset \texttt{ToothGrowth}. Berikut adalah sintaks untuk memuat dataset tersebut:

\begin{Shaded}
\begin{Highlighting}[]
\CommentTok{# ubah variable dose menjadi factor}
\NormalTok{ToothGrowth}\OperatorTok{$}\NormalTok{dose <-}\StringTok{ }\KeywordTok{as.factor}\NormalTok{(ToothGrowth}\OperatorTok{$}\NormalTok{dose)}

\CommentTok{# print}
\KeywordTok{head}\NormalTok{(ToothGrowth)}
\end{Highlighting}
\end{Shaded}

\begin{verbatim}
##    len supp dose
## 1  4.2   VC  0.5
## 2 11.5   VC  0.5
## 3  7.3   VC  0.5
## 4  5.8   VC  0.5
## 5  6.4   VC  0.5
## 6 10.0   VC  0.5
\end{verbatim}

Contoh sintaks dan output boxplot \emph{multiple group} disajikan pada Gambar \ref{fig:boxplot4}:

\begin{Shaded}
\begin{Highlighting}[]
\KeywordTok{boxplot}\NormalTok{(len }\OperatorTok{~}\StringTok{ }\NormalTok{supp}\OperatorTok{*}\NormalTok{dose, }\DataTypeTok{data =}\NormalTok{ ToothGrowth,}
        \DataTypeTok{col =} \KeywordTok{c}\NormalTok{(}\StringTok{"white"}\NormalTok{, }\StringTok{"steelblue"}\NormalTok{))}
\end{Highlighting}
\end{Shaded}

\begin{figure}

{\centering \includegraphics[width=0.7\linewidth]{Metode_Numerik_files/figure-latex/boxplot4-1} 

}

\caption{Boxplot multiple group}\label{fig:boxplot4}
\end{figure}

\hypertarget{customise}{%
\section{Kustomisasi Parameter Grafik}\label{customise}}

Pada bagian ini penulis akan menjelaskan cara untuk kustomisasi parameter grafik seperti:

\begin{enumerate}
\def\labelenumi{\alph{enumi}.}
\tightlist
\item
  menambahkan judul, legend, teks, axis, dan garis.
\item
  mengubah skala axis, simbol plot, jenis garis, dan warna.
\end{enumerate}

\hypertarget{addtitle}{%
\subsection{Menambahkan Judul}\label{addtitle}}

Pada grafik di \texttt{R}, kita dapat menambahkan judul dengan dua cara, yaitu: pada plot melalui parameter dan melalui fungsi plot(). Kedua cara tersebut tidak berbeda satu sama lain pada parameter input.

Untuk menambahkan judul pada plot secara langsung, kita dapat menggunakan argumen tambahan sebagai berikut:

\begin{enumerate}
\def\labelenumi{\alph{enumi}.}
\tightlist
\item
  \textbf{main}: teks untuk judul.
\item
  \textbf{xlab}: teks untuk keterangan axis X.
\item
  \textbf{ylab}: teks untuk keterangan axis y.
\item
  \textbf{sub}: teks untuk sub-judul.
\end{enumerate}

Berikut contoh sintaks penerapan masing-masing argumen tersebut beserta dengan output yang dihasilkan pada Gambar \ref{fig:title}:

\begin{Shaded}
\begin{Highlighting}[]
\CommentTok{# menambahkan judul}
\KeywordTok{barplot}\NormalTok{(}\KeywordTok{c}\NormalTok{(}\DecValTok{2}\NormalTok{,}\DecValTok{5}\NormalTok{), }\DataTypeTok{main=}\StringTok{"Main title"}\NormalTok{,}
        \DataTypeTok{xlab=}\StringTok{"X axis title"}\NormalTok{,}
        \DataTypeTok{ylab=}\StringTok{"Y axis title"}\NormalTok{,}
        \DataTypeTok{sub=}\StringTok{"Sub-title"}\NormalTok{)}
\end{Highlighting}
\end{Shaded}

\begin{figure}

{\centering \includegraphics[width=0.7\linewidth]{Metode_Numerik_files/figure-latex/title-1} 

}

\caption{Menambahkan Judul}\label{fig:title}
\end{figure}

kita juga dapat melakukan kustomisasi pada warna, \emph{font style}, dan ukuran font judul. Untuk melakukan kustomisasi pada warna pada judul, kita dapat menambahkan argumen sebagai berikut:

\begin{enumerate}
\def\labelenumi{\alph{enumi}.}
\tightlist
\item
  \textbf{col.main}: warna untuk judul.
\item
  \textbf{col.lab}: warna untuk keterangan axis.
\item
  \textbf{col.sub}: warna untuk sub-judul
\end{enumerate}

Untuk kustomisasi font judul, kita dapat menambahkan argumen berikut:

\begin{enumerate}
\def\labelenumi{\alph{enumi}.}
\tightlist
\item
  \textbf{font.main}: \emph{font style} untuk judul.
\item
  \textbf{font.lab}: \emph{font style} untuk keterangan axis.
\item
  \textbf{font.sub}: \emph{font style} untuk sub-judul.
\end{enumerate}

\begin{quote}
\textbf{Penting!!!}

Nilai yang dapat dimasukkan antara lain:

\begin{itemize}
\tightlist
\item
  \textbf{1}: untuk teks normal.
\item
  \textbf{2}: untuk teks cetak tebal.
\item
  \textbf{3}: untuk teks cetak miring.
\item
  \textbf{4}: untuk teks cetak tebal dan miring.
\item
  \textbf{5}: untuk font simbol.
\end{itemize}
\end{quote}

Sedangkan untuk ukuran font, kita dapat menambahkan variabel berikut:

\begin{enumerate}
\def\labelenumi{\alph{enumi}.}
\tightlist
\item
  \textbf{cex.main}: ukuran teks judul.
\item
  \textbf{cex.lab}: ukuran teks keterangan axis.
\item
  \textbf{cex.sub}: ukuran teks sub-judul.
\end{enumerate}

Berikut sintaks penerapan seluruh argumen tersebut beserta output yang dihasilkan pada Gambar \ref{fig:title2}:

\begin{Shaded}
\begin{Highlighting}[]
\CommentTok{# menambahkan judul}
\KeywordTok{barplot}\NormalTok{(}\KeywordTok{c}\NormalTok{(}\DecValTok{2}\NormalTok{,}\DecValTok{5}\NormalTok{), }
        \CommentTok{# menambahkan judul}
        \DataTypeTok{main=}\StringTok{"Main title"}\NormalTok{,}
        \DataTypeTok{xlab=}\StringTok{"X axis title"}\NormalTok{,}
        \DataTypeTok{ylab=}\StringTok{"Y axis title"}\NormalTok{,}
        \DataTypeTok{sub=}\StringTok{"Sub-title"}\NormalTok{,}
        \CommentTok{# kustomisasi warna font}
        \DataTypeTok{col.main=}\StringTok{"red"}\NormalTok{, }
        \DataTypeTok{col.lab=}\StringTok{"blue"}\NormalTok{, }
        \DataTypeTok{col.sub=}\StringTok{"black"}\NormalTok{,}
        \CommentTok{# kustomisasi font style}
        \DataTypeTok{font.main=}\DecValTok{4}\NormalTok{, }
        \DataTypeTok{font.lab=}\DecValTok{4}\NormalTok{, }
        \DataTypeTok{font.sub=}\DecValTok{4}\NormalTok{,}
        \CommentTok{# kustomisasi ukuran font}
        \DataTypeTok{cex.main=}\DecValTok{2}\NormalTok{, }
        \DataTypeTok{cex.lab=}\FloatTok{1.7}\NormalTok{, }
        \DataTypeTok{cex.sub=}\FloatTok{1.2}\NormalTok{)}
\end{Highlighting}
\end{Shaded}

\begin{figure}

{\centering \includegraphics[width=0.7\linewidth]{Metode_Numerik_files/figure-latex/title2-1} 

}

\caption{Menambahkan Judul (2)}\label{fig:title2}
\end{figure}

Kita telah belajar bagaimana menambahkan judul langsung pada fungsi plot. Selain cara tersebut, telah penulis jelaskan bahwa kita dapat menambahkan judul melalui fungsi \texttt{title()}. argumen yang dimasukkan pada dasarnya tidak berbeda dengan ketika kita menambahkan judul secara langsung pada plot. Berikut adalah contoh sintaks dan output yang dihasilkan pada Gambar \ref{fig:title3}:

\begin{Shaded}
\begin{Highlighting}[]
\CommentTok{# menambahkan judul}
\KeywordTok{barplot}\NormalTok{(}\KeywordTok{c}\NormalTok{(}\DecValTok{2}\NormalTok{,}\DecValTok{5}\NormalTok{,}\DecValTok{8}\NormalTok{))}

\CommentTok{# menambahkan judul}
\KeywordTok{title}\NormalTok{(}\DataTypeTok{main=}\StringTok{"Main title"}\NormalTok{,}
      \DataTypeTok{xlab=}\StringTok{"X axis title"}\NormalTok{,}
      \DataTypeTok{ylab=}\StringTok{"Y axis title"}\NormalTok{,}
      \DataTypeTok{sub=}\StringTok{"Sub-title"}\NormalTok{,}
      \CommentTok{# kustomisasi warna font}
      \DataTypeTok{col.main=}\StringTok{"red"}\NormalTok{, }
      \DataTypeTok{col.lab=}\StringTok{"blue"}\NormalTok{, }
      \DataTypeTok{col.sub=}\StringTok{"black"}\NormalTok{,}
      \CommentTok{# kustomisasi font style}
      \DataTypeTok{font.main=}\DecValTok{4}\NormalTok{, }
      \DataTypeTok{font.lab=}\DecValTok{4}\NormalTok{, }
      \DataTypeTok{font.sub=}\DecValTok{4}\NormalTok{,}
      \CommentTok{# kustomisasi ukuran font}
      \DataTypeTok{cex.main=}\DecValTok{2}\NormalTok{, }
      \DataTypeTok{cex.lab=}\FloatTok{1.7}\NormalTok{, }
      \DataTypeTok{cex.sub=}\FloatTok{1.2}\NormalTok{)}
\end{Highlighting}
\end{Shaded}

\begin{figure}

{\centering \includegraphics[width=0.7\linewidth]{Metode_Numerik_files/figure-latex/title3-1} 

}

\caption{Menambahkan Judul (3)}\label{fig:title3}
\end{figure}

\hypertarget{addlegend}{%
\subsection{Menambahkan Legend}\label{addlegend}}

Fungsi \texttt{legend()} pada \texttt{R} dapat digunakan untuk menambahkan legend pada grafik. Format sederhananya adalah sebagai berikut:

\begin{Shaded}
\begin{Highlighting}[]
\KeywordTok{legend}\NormalTok{(x, }\DataTypeTok{y=}\OtherTok{NULL}\NormalTok{, legend, fill, col, bg)}
\end{Highlighting}
\end{Shaded}

\begin{quote}
\textbf{Catatan:}

\begin{itemize}
\tightlist
\item
  \textbf{x} dan \textbf{y}: koordinat yang digunakan untuk posisi legend.
\item
  \textbf{legend}: teks pada legend
\item
  \textbf{fill}: warna yang digunakan untuk mengisi box disamping teks legend.
\item
  \textbf{col}: warna garis dan titik disamping teks legend.
\item
  \textbf{bg}: warna latar belakang legend box.
\end{itemize}
\end{quote}

Berikut adalah contoh sintaks dan ouput penerapan argumen disajikan pada Gambar \ref{fig:legend}:

\begin{Shaded}
\begin{Highlighting}[]
\CommentTok{# membuat vektor numerik}
\NormalTok{x <-}\StringTok{ }\KeywordTok{c}\NormalTok{(}\DecValTok{1}\OperatorTok{:}\DecValTok{10}\NormalTok{)}
\NormalTok{y <-}\StringTok{ }\NormalTok{x}\OperatorTok{^}\DecValTok{2}
\NormalTok{z <-}\StringTok{ }\NormalTok{x}\OperatorTok{*}\DecValTok{2}

\CommentTok{# membuat line plot}
\KeywordTok{plot}\NormalTok{(x,y, }\DataTypeTok{type=}\StringTok{"o"}\NormalTok{, }\DataTypeTok{col=}\StringTok{"red"}\NormalTok{, }\DataTypeTok{lty=}\DecValTok{1}\NormalTok{)}

\CommentTok{# menambahkan line plot}
\KeywordTok{lines}\NormalTok{(x,z, }\DataTypeTok{type=}\StringTok{"o"}\NormalTok{, }\DataTypeTok{col=}\StringTok{"blue"}\NormalTok{, }\DataTypeTok{lty=}\DecValTok{2}\NormalTok{)}

\CommentTok{# menambahkan legend}
\KeywordTok{legend}\NormalTok{(}\DecValTok{1}\NormalTok{, }\DecValTok{95}\NormalTok{, }\DataTypeTok{legend=}\KeywordTok{c}\NormalTok{(}\StringTok{"Line 1"}\NormalTok{, }\StringTok{"Line 2"}\NormalTok{),}
       \DataTypeTok{col=}\KeywordTok{c}\NormalTok{(}\StringTok{"red"}\NormalTok{, }\StringTok{"blue"}\NormalTok{), }\DataTypeTok{lty=}\DecValTok{1}\OperatorTok{:}\DecValTok{2}\NormalTok{, }\DataTypeTok{cex=}\FloatTok{0.8}\NormalTok{)}
\end{Highlighting}
\end{Shaded}

\begin{figure}

{\centering \includegraphics[width=0.7\linewidth]{Metode_Numerik_files/figure-latex/legend-1} 

}

\caption{Menambahkan legend}\label{fig:legend}
\end{figure}

Kita dapat menambahkan judul, merubah font, dan merubah warna backgroud pada legend. Argumen yang ditambahkan pada legend adalah sebagai berikut:

\begin{enumerate}
\def\labelenumi{\alph{enumi}.}
\tightlist
\item
  \textbf{title}: Judul legend
\item
  \textbf{text.font}: integer yang menunjukkan \emph{font style} pada teks legend. Nilai yang dapat dimasukkan adalah sebagai berikut:

  \begin{itemize}
  \tightlist
  \item
    \textbf{1}: normal
  \item
    \textbf{2}: cetak tebal
  \item
    \textbf{3}: cetak miring
  \item
    \textbf{4}: cetak tebal dan miring.
  \end{itemize}
\item
  \textbf{bg}: warna background legend box.
\end{enumerate}

Berikut adalah penerapan sintaks dan output yang dihasilkan pada Gambar \ref{fig:legend2}:

\begin{Shaded}
\begin{Highlighting}[]
\CommentTok{# membuat line plot}
\KeywordTok{plot}\NormalTok{(x,y, }\DataTypeTok{type=}\StringTok{"o"}\NormalTok{, }\DataTypeTok{col=}\StringTok{"red"}\NormalTok{, }\DataTypeTok{lty=}\DecValTok{1}\NormalTok{)}

\CommentTok{# menambahkan line plot}
\KeywordTok{lines}\NormalTok{(x,z, }\DataTypeTok{type=}\StringTok{"o"}\NormalTok{, }\DataTypeTok{col=}\StringTok{"blue"}\NormalTok{, }\DataTypeTok{lty=}\DecValTok{2}\NormalTok{)}

\CommentTok{# menambahkan legend}
\KeywordTok{legend}\NormalTok{(}\DecValTok{1}\NormalTok{, }\DecValTok{95}\NormalTok{, }\DataTypeTok{legend=}\KeywordTok{c}\NormalTok{(}\StringTok{"Line 1"}\NormalTok{, }\StringTok{"Line 2"}\NormalTok{),}
       \DataTypeTok{col=}\KeywordTok{c}\NormalTok{(}\StringTok{"red"}\NormalTok{, }\StringTok{"blue"}\NormalTok{), }\DataTypeTok{lty=}\DecValTok{1}\OperatorTok{:}\DecValTok{2}\NormalTok{, }\DataTypeTok{cex=}\FloatTok{0.8}\NormalTok{,}
       \DataTypeTok{title=}\StringTok{"Line types"}\NormalTok{, }\DataTypeTok{text.font=}\DecValTok{4}\NormalTok{, }\DataTypeTok{bg=}\StringTok{'lightblue'}\NormalTok{)}
\end{Highlighting}
\end{Shaded}

\begin{figure}

{\centering \includegraphics[width=0.7\linewidth]{Metode_Numerik_files/figure-latex/legend2-1} 

}

\caption{Menambahkan legend (2)}\label{fig:legend2}
\end{figure}

Kita dapat melakukan kustomisasi pada border dari legend melalui argumen \texttt{box.lty=}(jenis garis), \texttt{box.lwd=}(ukuran garis), dan \texttt{box.col=}(warna box). Berikut adalah penerapan argumen tersebut beserta output yang dihasilkan pada Gambar \ref{fig:legend3}:

\begin{Shaded}
\begin{Highlighting}[]
\CommentTok{# membuat line plot}
\KeywordTok{plot}\NormalTok{(x,y, }\DataTypeTok{type=}\StringTok{"o"}\NormalTok{, }\DataTypeTok{col=}\StringTok{"red"}\NormalTok{, }\DataTypeTok{lty=}\DecValTok{1}\NormalTok{)}

\CommentTok{# menambahkan line plot}
\KeywordTok{lines}\NormalTok{(x,z, }\DataTypeTok{type=}\StringTok{"o"}\NormalTok{, }\DataTypeTok{col=}\StringTok{"blue"}\NormalTok{, }\DataTypeTok{lty=}\DecValTok{2}\NormalTok{)}

\CommentTok{# menambahkan legend}
\KeywordTok{legend}\NormalTok{(}\DecValTok{1}\NormalTok{, }\DecValTok{95}\NormalTok{, }\DataTypeTok{legend=}\KeywordTok{c}\NormalTok{(}\StringTok{"Line 1"}\NormalTok{, }\StringTok{"Line 2"}\NormalTok{),}
       \DataTypeTok{col=}\KeywordTok{c}\NormalTok{(}\StringTok{"red"}\NormalTok{, }\StringTok{"blue"}\NormalTok{), }\DataTypeTok{lty=}\DecValTok{1}\OperatorTok{:}\DecValTok{2}\NormalTok{, }\DataTypeTok{cex=}\FloatTok{0.8}\NormalTok{,}
       \DataTypeTok{title=}\StringTok{"Line types"}\NormalTok{, }\DataTypeTok{text.font=}\DecValTok{4}\NormalTok{, }\DataTypeTok{bg=}\StringTok{'white'}\NormalTok{,}
       \DataTypeTok{box.lty=}\DecValTok{2}\NormalTok{, }\DataTypeTok{box.lwd=}\DecValTok{2}\NormalTok{, }\DataTypeTok{box.col=}\StringTok{"steelblue"}\NormalTok{)}
\end{Highlighting}
\end{Shaded}

\begin{figure}

{\centering \includegraphics[width=0.7\linewidth]{Metode_Numerik_files/figure-latex/legend3-1} 

}

\caption{Menambahkan legend (3)}\label{fig:legend3}
\end{figure}

Selain menggunakan koordinat, kita juga dapat melakukan kustomisasi posisi legend menggunakan \emph{keyword} seperti: bottomright``,''bottom``,''bottomleft``,''left``,''topleft``,''top``,''topright``,''right" and ``center''. Sejumlah kustomisasi legend berdasarkan \emph{keyword} disajikan pada Gambar \ref{fig:legend4}:

\begin{Shaded}
\begin{Highlighting}[]
\CommentTok{# plot}
\KeywordTok{plot}\NormalTok{(x,y, }\DataTypeTok{type =} \StringTok{"n"}\NormalTok{)}

\CommentTok{# posisi kiri atas, inset =0.05}
\KeywordTok{legend}\NormalTok{(}\StringTok{"topleft"}\NormalTok{,}
  \DataTypeTok{legend =} \StringTok{"(x,y)"}\NormalTok{,}
  \DataTypeTok{title =} \StringTok{"topleft, inset = .05"}\NormalTok{,}
  \DataTypeTok{inset =} \FloatTok{0.05}\NormalTok{)}
\CommentTok{# posisi atas}
\KeywordTok{legend}\NormalTok{(}\StringTok{"top"}\NormalTok{,}
       \DataTypeTok{legend =} \StringTok{"(x,y)"}\NormalTok{,}
       \DataTypeTok{title =} \StringTok{"top"}\NormalTok{)}
\CommentTok{# posisi kanan atas inset = .02}
\KeywordTok{legend}\NormalTok{(}\StringTok{"topright"}\NormalTok{,}
       \DataTypeTok{legend =} \StringTok{"(x,y)"}\NormalTok{,}
       \DataTypeTok{title =} \StringTok{"topright, inset = .02"}\NormalTok{,}
       \DataTypeTok{inset =} \FloatTok{0.02}\NormalTok{)}
\CommentTok{# posisi kiri}
\KeywordTok{legend}\NormalTok{(}\StringTok{"left"}\NormalTok{,}
       \DataTypeTok{legend =} \StringTok{"(x,y)"}\NormalTok{,}
       \DataTypeTok{title =} \StringTok{"left"}\NormalTok{)}
\CommentTok{# posisi tengah}
\KeywordTok{legend}\NormalTok{(}\StringTok{"center"}\NormalTok{,}
       \DataTypeTok{legend =} \StringTok{"(x,y)"}\NormalTok{,}
       \DataTypeTok{title =} \StringTok{"center"}\NormalTok{)}
\CommentTok{# posisi kanan}
\KeywordTok{legend}\NormalTok{(}\StringTok{"right"}\NormalTok{,}
       \DataTypeTok{legend =} \StringTok{"(x,y)"}\NormalTok{,}
       \DataTypeTok{title =} \StringTok{"right"}\NormalTok{)}
\CommentTok{# posisi kiri bawah}
\KeywordTok{legend}\NormalTok{(}\StringTok{"bottomleft"}\NormalTok{,}
       \DataTypeTok{legend =} \StringTok{"(x,y)"}\NormalTok{,}
       \DataTypeTok{title =} \StringTok{"bottomleft"}\NormalTok{)}
\CommentTok{# posisi bawah}
\KeywordTok{legend}\NormalTok{(}\StringTok{"bottom"}\NormalTok{,}
       \DataTypeTok{legend =} \StringTok{"(x,y)"}\NormalTok{,}
       \DataTypeTok{title =} \StringTok{"bottom"}\NormalTok{)}
\CommentTok{# posisi kanan bawah}
\KeywordTok{legend}\NormalTok{(}\StringTok{"bottomright"}\NormalTok{,}
       \DataTypeTok{legend =} \StringTok{"(x,y)"}\NormalTok{,}
       \DataTypeTok{title =} \StringTok{"bottomright"}\NormalTok{)}
\end{Highlighting}
\end{Shaded}

\begin{figure}

{\centering \includegraphics[width=0.7\linewidth]{Metode_Numerik_files/figure-latex/legend4-1} 

}

\caption{Kustomisasi posisi legend}\label{fig:legend4}
\end{figure}

\hypertarget{addtext}{%
\subsection{Menambahkan Teks Pada Grafik}\label{addtext}}

Teks pada grafik dapat kita tambahkan baik sebagai keterangan yang menunjukkan label suatu observasi, keterangan tambahan disekitar bingkai grafik, maupun sebuah persamaan yang ada pada bidang grafik. Untuk menambahkannya kita dapat menggunakan dua buah fungsi yaitu: \texttt{text()} dan \texttt{mtext()}.

FUngsi \texttt{text()} berguna untuk menambahkan teks di dalam bidang grafik seperti label titik observasi dan persamaan di dalam bidang grafik. Format yang digunakan adalah sebagai berikut:

\begin{Shaded}
\begin{Highlighting}[]
\KeywordTok{text}\NormalTok{(x, y, labels)}
\end{Highlighting}
\end{Shaded}

\begin{quote}
\textbf{Catatan:}

\begin{itemize}
\tightlist
\item
  \textbf{x} dan \textbf{y}: vektor numerik yang menunjukkan koordinat posisi teks.
\item
  \textbf{labels}: vektor karakter yang menunjukkan teks yang hendak ditulis.
\end{itemize}
\end{quote}

Berikut adalah contoh sintaks untuk memberi label pada sejumlah data yang memiliki kriteria yang kita inginkan dan output yang dihasilkan pada Gambar \ref{fig:text}:

\begin{Shaded}
\begin{Highlighting}[]
\CommentTok{# tandai observasi yang memiliki nilai}
\CommentTok{# mpg < 15 dan wt > 5}
\NormalTok{d <-}\StringTok{ }\NormalTok{mtcars[mtcars}\OperatorTok{$}\NormalTok{wt }\OperatorTok{>=}\StringTok{ }\DecValTok{5} \OperatorTok{&}\StringTok{ }\NormalTok{mtcars}\OperatorTok{$}\NormalTok{mpg }\OperatorTok{<=}\StringTok{ }\DecValTok{15}\NormalTok{, ]}


\CommentTok{# plot}
\KeywordTok{plot}\NormalTok{(mtcars}\OperatorTok{$}\NormalTok{wt, mtcars}\OperatorTok{$}\NormalTok{mpg, }\DataTypeTok{main=}\StringTok{"Milage vs. Car Weight"}\NormalTok{,}
      \DataTypeTok{xlab=}\StringTok{"Weight"}\NormalTok{, }\DataTypeTok{ylab=}\StringTok{"Miles/(US) gallon"}\NormalTok{)}

\CommentTok{# menambahkan text}
\KeywordTok{text}\NormalTok{(d}\OperatorTok{$}\NormalTok{wt, d}\OperatorTok{$}\NormalTok{mpg,  }\KeywordTok{row.names}\NormalTok{(d),}
     \DataTypeTok{cex=}\FloatTok{0.65}\NormalTok{, }\DataTypeTok{pos=}\DecValTok{3}\NormalTok{,}\DataTypeTok{col=}\StringTok{"red"}\NormalTok{)}
\end{Highlighting}
\end{Shaded}

\begin{figure}

{\centering \includegraphics[width=0.7\linewidth]{Metode_Numerik_files/figure-latex/text-1} 

}

\caption{Menambahkan teks}\label{fig:text}
\end{figure}

Sedangkan sintaks berikut adalah contoh bagaimana menambahkan persamaan kedalam bidang grafik dan output yang dihasilkan pada Gambar \ref{fig:text2}:

\begin{Shaded}
\begin{Highlighting}[]
\KeywordTok{plot}\NormalTok{(}\DecValTok{1}\OperatorTok{:}\DecValTok{10}\NormalTok{, }\DecValTok{1}\OperatorTok{:}\DecValTok{10}\NormalTok{, }
     \DataTypeTok{main=}\StringTok{"text(...) examples}\CharTok{\textbackslash{}n}\StringTok{~~~~~~~~~~~"}\NormalTok{)}
\KeywordTok{text}\NormalTok{(}\DecValTok{4}\NormalTok{, }\DecValTok{9}\NormalTok{, }\KeywordTok{expression}\NormalTok{(}\KeywordTok{hat}\NormalTok{(beta) }\OperatorTok{==}\StringTok{ }\NormalTok{(X}\OperatorTok{^}\NormalTok{t }\OperatorTok{*}\StringTok{ }\NormalTok{X)}\OperatorTok{^}\NormalTok{\{}\OperatorTok{-}\DecValTok{1}\NormalTok{\} }\OperatorTok{*}\StringTok{ }\NormalTok{X}\OperatorTok{^}\NormalTok{t }\OperatorTok{*}\StringTok{ }\NormalTok{y))}
\KeywordTok{text}\NormalTok{(}\DecValTok{7}\NormalTok{, }\DecValTok{4}\NormalTok{, }\KeywordTok{expression}\NormalTok{(}\KeywordTok{bar}\NormalTok{(x) }\OperatorTok{==}\StringTok{ }\KeywordTok{sum}\NormalTok{(}\KeywordTok{frac}\NormalTok{(x[i], n), i}\OperatorTok{==}\DecValTok{1}\NormalTok{, n)))}
\end{Highlighting}
\end{Shaded}

\begin{figure}

{\centering \includegraphics[width=0.7\linewidth]{Metode_Numerik_files/figure-latex/text2-1} 

}

\caption{Menambahkan teks (2)}\label{fig:text2}
\end{figure}

Fungsi \texttt{mtext()} berguna untuk menambahkan teks pada frame sekitar bidang grafik. Format yang digunakan adalah sebagai berikut:

\begin{Shaded}
\begin{Highlighting}[]
\KeywordTok{mtext}\NormalTok{(text, }\DataTypeTok{side=}\DecValTok{3}\NormalTok{)}
\end{Highlighting}
\end{Shaded}

\begin{quote}
\textbf{Catatan:}

\begin{itemize}
\tightlist
\item
  \textbf{text}: teks yang akan ditulis.
\item
  \textbf{side}: integer yang menunjukkan lokasi teks yang akan ditulis. Nilai yang dapat dimasukkan antara lain:
\item
  1: bawah
\item
  2: kiri
\item
  3: atas
\item
  4: kanan.
\end{itemize}
\end{quote}

Berikut adalah contoh penerapan dan output yang dihasilkan pada Gambar \ref{fig:text3}:

\begin{Shaded}
\begin{Highlighting}[]
\KeywordTok{plot}\NormalTok{(}\DecValTok{1}\OperatorTok{:}\DecValTok{10}\NormalTok{, }\DecValTok{1}\OperatorTok{:}\DecValTok{10}\NormalTok{, }
     \DataTypeTok{main=}\StringTok{"mtext(...) examples}\CharTok{\textbackslash{}n}\StringTok{~~~~~~~~~~~"}\NormalTok{)}
\KeywordTok{mtext}\NormalTok{(}\StringTok{"Magic function"}\NormalTok{, }\DataTypeTok{side=}\DecValTok{3}\NormalTok{)}
\end{Highlighting}
\end{Shaded}

\begin{figure}

{\centering \includegraphics[width=0.7\linewidth]{Metode_Numerik_files/figure-latex/text3-1} 

}

\caption{Menambahkan teks (3)}\label{fig:text3}
\end{figure}

\hypertarget{addlines}{%
\subsection{Menambahkan Garis Pada Plot}\label{addlines}}

Fungsi \texttt{abline()} dapat digunakan untuk menamabahkan garis pada plot. Garis yang ditambahkan dapat berupa garis vertikal, horizontal, maupun garis regresi. Format yang digunakan adalah sebagi berikut:

\begin{Shaded}
\begin{Highlighting}[]
\KeywordTok{abline}\NormalTok{(}\DataTypeTok{v=}\NormalTok{y)}
\end{Highlighting}
\end{Shaded}

Berikut adalah contoh sintaks bagaimana menambahkan garis pada sebuah plot dan output yang dihasilkan disajikan pada Gambar \ref{fig:abline}:

\begin{Shaded}
\begin{Highlighting}[]
\CommentTok{# membuat plot}
\KeywordTok{plot}\NormalTok{(mtcars}\OperatorTok{$}\NormalTok{wt, mtcars}\OperatorTok{$}\NormalTok{mpg, }\DataTypeTok{main=}\StringTok{"Milage vs. Car Weight"}\NormalTok{,}
      \DataTypeTok{xlab=}\StringTok{"Weight"}\NormalTok{, }\DataTypeTok{ylab=}\StringTok{"Miles/(US) gallon"}\NormalTok{)}

\CommentTok{# menambahkan garis vertikal di titik rata-rata weight}
\KeywordTok{abline}\NormalTok{(}\DataTypeTok{v=}\KeywordTok{mean}\NormalTok{(mtcars}\OperatorTok{$}\NormalTok{wt), }\DataTypeTok{col=}\StringTok{"red"}\NormalTok{, }\DataTypeTok{lwd=}\DecValTok{3}\NormalTok{, }\DataTypeTok{lty=}\DecValTok{2}\NormalTok{)}

\CommentTok{# menambahkan garis horizontal di titik rata-rata  mpg}
\KeywordTok{abline}\NormalTok{(}\DataTypeTok{h=}\KeywordTok{mean}\NormalTok{(mtcars}\OperatorTok{$}\NormalTok{mpg), }\DataTypeTok{col=}\StringTok{"blue"}\NormalTok{, }\DataTypeTok{lwd=}\DecValTok{3}\NormalTok{, }\DataTypeTok{lty=}\DecValTok{3}\NormalTok{)}

\CommentTok{# menambahkan garis regresi}
\KeywordTok{abline}\NormalTok{(}\KeywordTok{lm}\NormalTok{(mpg}\OperatorTok{~}\NormalTok{wt, }\DataTypeTok{data=}\NormalTok{mtcars), }\DataTypeTok{lwd=}\DecValTok{4}\NormalTok{, }\DataTypeTok{lty=}\DecValTok{4}\NormalTok{)}
\end{Highlighting}
\end{Shaded}

\begin{figure}

{\centering \includegraphics[width=0.7\linewidth]{Metode_Numerik_files/figure-latex/abline-1} 

}

\caption{Menambahkan garis}\label{fig:abline}
\end{figure}

\hypertarget{changepoint}{%
\subsection{Merubah Simbol plot dan Jenis Garis}\label{changepoint}}

Simbol plot (jenis titik) dapat diubah dengan menambahkan argumen \texttt{pch=} pada plot. Nilai yang dimasukkan pada argumen tersebut adalah integer dengan kemungkinan nilai sebagai berikut:

\begin{itemize}
\tightlist
\item
  pch = 0,square
\item
  pch = 1,circle (default)
\item
  pch = 2,triangle point up
\item
  pch = 3,plus
\item
  pch = 4,cross
\item
  pch = 5,diamond
\item
  pch = 6,triangle point down
\item
  pch = 7,square cross
\item
  pch = 8,star
\item
  pch = 9,diamond plus
\item
  pch = 10,circle plus
\item
  pch = 11,triangles up and down
\item
  pch = 12,square plus
\item
  pch = 13,circle cross
\item
  pch = 14,square and triangle down
\item
  pch = 15, filled square
\item
  pch = 16, filled circle
\item
  pch = 17, filled triangle point-up
\item
  pch = 18, filled diamond
\item
  pch = 19, solid circle
\item
  pch = 20,bullet (smaller circle)
\item
  pch = 21, filled circle blue
\item
  pch = 22, filled square blue
\item
  pch = 23, filled diamond blue
\item
  pch = 24, filled triangle point-up blue
\item
  pch = 25, filled triangle point down blue
\end{itemize}

Untuk lebih memahami bentuk simbol tersebut, penulis akan menyajikan sintaks yang menampilkan seluruh simbol tersebut pada satu grafik. Output yang dihasilkan disajikan pada Gambar \ref{fig:symbol}:

\begin{Shaded}
\begin{Highlighting}[]
\NormalTok{generateRPointShapes<-}\ControlFlowTok{function}\NormalTok{()\{}
  \CommentTok{# menentukan parameter plot}
\NormalTok{  oldPar<-}\KeywordTok{par}\NormalTok{()}
  \KeywordTok{par}\NormalTok{(}\DataTypeTok{font=}\DecValTok{2}\NormalTok{, }\DataTypeTok{mar=}\KeywordTok{c}\NormalTok{(}\FloatTok{0.5}\NormalTok{,}\DecValTok{0}\NormalTok{,}\DecValTok{0}\NormalTok{,}\DecValTok{0}\NormalTok{))}
  \CommentTok{# produksi titik axis}
\NormalTok{  y=}\KeywordTok{rev}\NormalTok{(}\KeywordTok{c}\NormalTok{(}\KeywordTok{rep}\NormalTok{(}\DecValTok{1}\NormalTok{,}\DecValTok{6}\NormalTok{),}\KeywordTok{rep}\NormalTok{(}\DecValTok{2}\NormalTok{,}\DecValTok{5}\NormalTok{), }\KeywordTok{rep}\NormalTok{(}\DecValTok{3}\NormalTok{,}\DecValTok{5}\NormalTok{), }\KeywordTok{rep}\NormalTok{(}\DecValTok{4}\NormalTok{,}\DecValTok{5}\NormalTok{), }\KeywordTok{rep}\NormalTok{(}\DecValTok{5}\NormalTok{,}\DecValTok{5}\NormalTok{)))}
\NormalTok{  x=}\KeywordTok{c}\NormalTok{(}\KeywordTok{rep}\NormalTok{(}\DecValTok{1}\OperatorTok{:}\DecValTok{5}\NormalTok{,}\DecValTok{5}\NormalTok{),}\DecValTok{6}\NormalTok{)}
  \CommentTok{# plot seluruh titik dan label}
  \KeywordTok{plot}\NormalTok{(x, y, }\DataTypeTok{pch =} \DecValTok{0}\OperatorTok{:}\DecValTok{25}\NormalTok{, }\DataTypeTok{cex=}\FloatTok{1.5}\NormalTok{, }\DataTypeTok{ylim=}\KeywordTok{c}\NormalTok{(}\DecValTok{1}\NormalTok{,}\FloatTok{5.5}\NormalTok{), }\DataTypeTok{xlim=}\KeywordTok{c}\NormalTok{(}\DecValTok{1}\NormalTok{,}\FloatTok{6.5}\NormalTok{), }
       \DataTypeTok{axes=}\OtherTok{FALSE}\NormalTok{, }\DataTypeTok{xlab=}\StringTok{""}\NormalTok{, }\DataTypeTok{ylab=}\StringTok{""}\NormalTok{, }\DataTypeTok{bg=}\StringTok{"blue"}\NormalTok{)}
  \KeywordTok{text}\NormalTok{(x, y, }\DataTypeTok{labels=}\DecValTok{0}\OperatorTok{:}\DecValTok{25}\NormalTok{, }\DataTypeTok{pos=}\DecValTok{3}\NormalTok{)}
  \KeywordTok{par}\NormalTok{(}\DataTypeTok{mar=}\NormalTok{oldPar}\OperatorTok{$}\NormalTok{mar,}\DataTypeTok{font=}\NormalTok{oldPar}\OperatorTok{$}\NormalTok{font )}
\NormalTok{\}}

\CommentTok{# Print}
\KeywordTok{generateRPointShapes}\NormalTok{()}
\end{Highlighting}
\end{Shaded}

\begin{figure}

{\centering \includegraphics[width=0.7\linewidth]{Metode_Numerik_files/figure-latex/symbol-1} 

}

\caption{Symbol plot}\label{fig:symbol}
\end{figure}

Pada \texttt{R} kita juga dapat mengatur jenis garis yang akan ditampilkan pada plot dengan menambahkan argumen \texttt{lty=} (\emph{line type}) pada fungsi plot. Nilai yang dapat dimasukkan adalah nilai integer. Keterangan masing-masing nilai tersebut adalah sebagai berikut:

\begin{itemize}
\tightlist
\item
  lty = 0, blank
\item
  lty = 1, solid (default)
\item
  lty = 2, dashed
\item
  lty = 3, dotted
\item
  lty = 4, dotdash
\item
  lty = 5, longdash
\item
  lty = 6, twodash
\end{itemize}

Untuk lebih memahaminya, pada sintaks berikut disajikan plot seluruh jenis garis tersebut beserta output yang dihasilkannya pada Gambar \ref{fig:lty}:

\begin{Shaded}
\begin{Highlighting}[]
\NormalTok{generateRLineTypes<-}\ControlFlowTok{function}\NormalTok{()\{}
\NormalTok{  oldPar<-}\KeywordTok{par}\NormalTok{()}
  \KeywordTok{par}\NormalTok{(}\DataTypeTok{font=}\DecValTok{2}\NormalTok{, }\DataTypeTok{mar=}\KeywordTok{c}\NormalTok{(}\DecValTok{0}\NormalTok{,}\DecValTok{0}\NormalTok{,}\DecValTok{0}\NormalTok{,}\DecValTok{0}\NormalTok{))}
  \KeywordTok{plot}\NormalTok{(}\DecValTok{1}\NormalTok{, }\DataTypeTok{pch=}\StringTok{""}\NormalTok{, }\DataTypeTok{ylim=}\KeywordTok{c}\NormalTok{(}\DecValTok{0}\NormalTok{,}\DecValTok{6}\NormalTok{), }\DataTypeTok{xlim=}\KeywordTok{c}\NormalTok{(}\DecValTok{0}\NormalTok{,}\FloatTok{0.7}\NormalTok{), }\DataTypeTok{axes =} \OtherTok{FALSE}\NormalTok{ ,}\DataTypeTok{xlab=}\StringTok{""}\NormalTok{, }\DataTypeTok{ylab=}\StringTok{""}\NormalTok{)}
  \ControlFlowTok{for}\NormalTok{(i }\ControlFlowTok{in} \DecValTok{0}\OperatorTok{:}\DecValTok{6}\NormalTok{) }\KeywordTok{lines}\NormalTok{(}\KeywordTok{c}\NormalTok{(}\FloatTok{0.3}\NormalTok{,}\FloatTok{0.7}\NormalTok{), }\KeywordTok{c}\NormalTok{(i,i), }\DataTypeTok{lty=}\NormalTok{i, }\DataTypeTok{lwd=}\DecValTok{3}\NormalTok{)}
  \KeywordTok{text}\NormalTok{(}\KeywordTok{rep}\NormalTok{(}\FloatTok{0.1}\NormalTok{,}\DecValTok{6}\NormalTok{), }\DecValTok{0}\OperatorTok{:}\DecValTok{6}\NormalTok{, }
       \DataTypeTok{labels=}\KeywordTok{c}\NormalTok{(}\StringTok{"0.'blank'"}\NormalTok{, }\StringTok{"1.'solid'"}\NormalTok{, }\StringTok{"2.'dashed'"}\NormalTok{, }\StringTok{"3.'dotted'"}\NormalTok{, }
                \StringTok{"4.'dotdash'"}\NormalTok{, }\StringTok{"5.'longdash'"}\NormalTok{, }\StringTok{"6.'twodash'"}\NormalTok{))}
  \KeywordTok{par}\NormalTok{(}\DataTypeTok{mar=}\NormalTok{oldPar}\OperatorTok{$}\NormalTok{mar,}\DataTypeTok{font=}\NormalTok{oldPar}\OperatorTok{$}\NormalTok{font )}
\NormalTok{\}}
\KeywordTok{generateRLineTypes}\NormalTok{()}
\end{Highlighting}
\end{Shaded}

\begin{figure}

{\centering \includegraphics[width=0.7\linewidth]{Metode_Numerik_files/figure-latex/lty-1} 

}

\caption{Line type}\label{fig:lty}
\end{figure}

\hypertarget{mengatur-axis-plot}{%
\subsection{Mengatur Axis Plot}\label{mengatur-axis-plot}}

Kita dapat melakukan pengaturan lebih jauh terhadap axis, seperti: menambahkan axis tambahan pada atas dan bawah frame, mengubah rentang nilai axis, serta kustomisasi \emph{tick mark} pada nilai axis. Hal ini diperlukan karena fungsi grafik dasar \texttt{R} tidak dapat mengatur axis secara otomatis saat plot baru ditambahkan pada plot pertama dan rentang nilai plot baru lebih besar dibanding plot pertama, sehingga sebagian nilai plot baru tidak ditampilkan pada hasil akhir.

Untuk menambahkan axis pada \texttt{R} kita dapat menambahkan fungsi \texttt{axis()} setelah plot dilakukan. Format yang digunakan adalah sebagai berikut:

\begin{Shaded}
\begin{Highlighting}[]
\KeywordTok{axis}\NormalTok{(side, }\DataTypeTok{at=}\OtherTok{NULL}\NormalTok{, }\DataTypeTok{labels=}\OtherTok{TRUE}\NormalTok{)}
\end{Highlighting}
\end{Shaded}

\begin{quote}
\textbf{Catatan:}

\begin{itemize}
\tightlist
\item
  \textbf{side}: nilai integer yang mengidikasikan posisi axix yang hendak ditambahkan. Nilai yang dapat dimasukkan adalah sebagai berikut:

  \begin{itemize}
  \tightlist
  \item
    1: bawah
  \item
    2: kiri
  \item
    3: atas
  \item
    4: kanan.
  \end{itemize}
\item
  \textbf{at}: titik dimana \emph{tick-mark} hendak digambarkan. Nilai yang dapat dimasukkan sama dengan \texttt{side}.
\item
  \textbf{labels}: Teks label \emph{tick-mark}. Dapat juga secara logis menentukan apakah anotasi harus dibuat pada \emph{tick mark}.
\end{itemize}
\end{quote}

Berikut contoh sintaks penerapan fungsi tersebut dan output yang dihasilkan pada Gambar \ref{fig:axis}:

\begin{Shaded}
\begin{Highlighting}[]
\CommentTok{# membuat vektor numerik}
\NormalTok{x <-}\StringTok{ }\KeywordTok{c}\NormalTok{(}\DecValTok{1}\OperatorTok{:}\DecValTok{4}\NormalTok{)}
\NormalTok{y <-}\StringTok{ }\NormalTok{x}\OperatorTok{^}\DecValTok{2}

\CommentTok{# plot}
\KeywordTok{plot}\NormalTok{(x, y, }\DataTypeTok{pch=}\DecValTok{18}\NormalTok{, }\DataTypeTok{col=}\StringTok{"red"}\NormalTok{, }\DataTypeTok{type=}\StringTok{"b"}\NormalTok{,}
     \DataTypeTok{frame=}\OtherTok{FALSE}\NormalTok{, }\DataTypeTok{xaxt=}\StringTok{"n"}\NormalTok{) }\CommentTok{# Remove x axis}

\CommentTok{# menambahkan axis}
\CommentTok{# bawah}
\KeywordTok{axis}\NormalTok{(}\DecValTok{1}\NormalTok{, }\DecValTok{1}\OperatorTok{:}\DecValTok{4}\NormalTok{, LETTERS[}\DecValTok{1}\OperatorTok{:}\DecValTok{4}\NormalTok{], }\DataTypeTok{col.axis=}\StringTok{"blue"}\NormalTok{)}
\CommentTok{# atas}
\KeywordTok{axis}\NormalTok{(}\DecValTok{3}\NormalTok{, }\DataTypeTok{col =} \StringTok{"darkgreen"}\NormalTok{, }\DataTypeTok{lty =} \DecValTok{2}\NormalTok{, }\DataTypeTok{lwd =} \FloatTok{0.5}\NormalTok{)}
\CommentTok{# kanan}
\KeywordTok{axis}\NormalTok{(}\DecValTok{4}\NormalTok{, }\DataTypeTok{col =} \StringTok{"violet"}\NormalTok{, }\DataTypeTok{col.axis =} \StringTok{"dark violet"}\NormalTok{, }\DataTypeTok{lwd =} \DecValTok{2}\NormalTok{)}
\end{Highlighting}
\end{Shaded}

\begin{figure}

{\centering \includegraphics[width=0.7\linewidth]{Metode_Numerik_files/figure-latex/axis-1} 

}

\caption{Modifikasi axis}\label{fig:axis}
\end{figure}

Kita dapat mengubah rentang nilai pada axis menggunakan fungsi \texttt{xlim()} dan \texttt{ylim()} yang menyatakan vektor nilai masimum dan minimum rentang. Selain itu kita dapat juga melakukan tranformasi baik pada sumbu x dan sumbu y. Berikut adalah argumen yang dapat ditambahkan pada fungsi grafik:

\begin{itemize}
\tightlist
\item
  \textbf{xlim}: limit nilai sumbu x dengan format: \texttt{xlim(min,\ max)}.
\item
  \textbf{ylim}: limit nilai sumbu x dengan format: \texttt{ylim(min,\ max)}.
\end{itemize}

Untuk transformasi skala log, kita dapat menambahkan argumen berikut:

\begin{itemize}
\tightlist
\item
  \textbf{log=``x''}: transformasi log sumbu x.
\item
  \textbf{log=``y''}: transformasi log sumbu y.
\item
  \textbf{log=``xy''}: transformasi log sumbu x dan y.
\end{itemize}

Berikut adalah contoh sintaks penerapan argumen tersebut beserta output yang dihasilkan pada Gambar \ref{fig:axis2}:

\begin{Shaded}
\begin{Highlighting}[]
\CommentTok{# membagi jendela grafik menjadi 1 baris dan 3 kolom}
\KeywordTok{par}\NormalTok{(}\DataTypeTok{mfrow=}\KeywordTok{c}\NormalTok{(}\DecValTok{1}\NormalTok{,}\DecValTok{3}\NormalTok{))}

\CommentTok{# membuat vektor numerik}
\NormalTok{x<-}\KeywordTok{c}\NormalTok{(}\DecValTok{1}\OperatorTok{:}\DecValTok{10}\NormalTok{); y<-x}\OperatorTok{*}\NormalTok{x}

\CommentTok{# simple plot}
\KeywordTok{plot}\NormalTok{(x, y)}

\CommentTok{# plot dengan pengaturan rentang skala}
\KeywordTok{plot}\NormalTok{(x, y, }\DataTypeTok{xlim=}\KeywordTok{c}\NormalTok{(}\DecValTok{1}\NormalTok{,}\DecValTok{15}\NormalTok{), }\DataTypeTok{ylim=}\KeywordTok{c}\NormalTok{(}\DecValTok{1}\NormalTok{,}\DecValTok{150}\NormalTok{))}

\CommentTok{# plot dengan transformasi skala log}
\KeywordTok{plot}\NormalTok{(x, y, }\DataTypeTok{log=}\StringTok{"y"}\NormalTok{)}
\end{Highlighting}
\end{Shaded}

\begin{figure}

{\centering \includegraphics[width=0.8\linewidth]{Metode_Numerik_files/figure-latex/axis2-1} 

}

\caption{Mengubah rentang dan skala axis}\label{fig:axis2}
\end{figure}

Kita dapat melakukan kustomisasi pada \emph{tick mark}. Kustomisasi yang dapat dilakukan adalah merubah warna, \emph{font style}, ukuran font, orientasi, serta menyembunyikan \emph{tick mark}.

Argumen yang ditambahkan adalah sebagai berikut:

\begin{itemize}
\item
  \textbf{col.axis}: warna \emph{tick mark}.
\item
  \textbf{font.axis}: integer yang menunjukkan \emph{font style}. Sama dengan pengaturan judul.
\item
  \textbf{cex.axis}: pengaturan ukuran \emph{tick mark}.
\item
  \textbf{las}: mengatur orientasi \emph{tick mark}. Nilai yang dapat dimasukkan adalah sebagai berikut:

  \begin{itemize}
  \tightlist
  \item
    \textbf{0}: paralel terhadap posisi axis (default)
  \item
    \textbf{1}: selalu horizontal
  \item
    \textbf{2}: selalu perpendikular dengan posisi axis
  \item
    \textbf{3}: selalu vertikal
  \end{itemize}
\item
  \textbf{xaxt} dan \textbf{yaxt}: karakter untuk menunjukkan apakah axis akan ditampilkan atau tidak. nilai dapat berupa ``n''(sembunyika) dan ``s''(tampilkan).
\end{itemize}

Berikut adalah contoh penerapan argumen tersebut beserta output pada Gambar \ref{fig:axis3}:

\begin{Shaded}
\begin{Highlighting}[]
\CommentTok{# membuat vektor numerik}
\NormalTok{x<-}\KeywordTok{c}\NormalTok{(}\DecValTok{1}\OperatorTok{:}\DecValTok{10}\NormalTok{); y<-x}\OperatorTok{*}\NormalTok{x}

\CommentTok{# plot}
\KeywordTok{plot}\NormalTok{(x,y,}
     \CommentTok{# warna}
     \DataTypeTok{col.axis=}\StringTok{"red"}\NormalTok{,}
     \CommentTok{# font style}
     \DataTypeTok{font.axis=}\DecValTok{2}\NormalTok{,}
     \CommentTok{# ukuran}
     \DataTypeTok{cex=}\FloatTok{1.5}\NormalTok{,}
     \CommentTok{# orientasi}
     \DataTypeTok{las=}\DecValTok{3}\NormalTok{,}
     \CommentTok{# sembunyikan sumbu x}
     \DataTypeTok{xaxt=}\StringTok{"n"}\NormalTok{)}
\end{Highlighting}
\end{Shaded}

\begin{figure}

{\centering \includegraphics[width=0.7\linewidth]{Metode_Numerik_files/figure-latex/axis3-1} 

}

\caption{Kustomisasi tick mark}\label{fig:axis3}
\end{figure}

\hypertarget{changecolor}{%
\subsection{Mengatur Warna}\label{changecolor}}

Pada fungsi dasar \texttt{R}, warna dapat diatur dengan mengetikkan nama warna maupun kode hexadesimal. Selain itu kita juga dapat menamambahkan warna lain melalui library lain yang tidak dijelaskan pada chapter ini.

Untuk penggunaan warna hexadesima kita perlu mengetikkan ``\#'' yang diukuti oleh 6 kode warna. Untuk memperlajari kode-kode dan warna yang dihasilkan, silahkan pembaca mengunjungi situs \url{http://www.visibone.com/}.

Pada sintaks berikut disajikan visualisasi nama-nama warna bawaan yang ada pada \texttt{R}. Output yang dihasilkan disajikan pada Gambar \ref{fig:color}:

\begin{Shaded}
\begin{Highlighting}[]
\NormalTok{showCols <-}\StringTok{ }\ControlFlowTok{function}\NormalTok{(}\DataTypeTok{cl=}\KeywordTok{colors}\NormalTok{(), }\DataTypeTok{bg =} \StringTok{"grey"}\NormalTok{,}
                     \DataTypeTok{cex =} \FloatTok{0.75}\NormalTok{, }\DataTypeTok{rot =} \DecValTok{30}\NormalTok{) \{}
\NormalTok{    m <-}\StringTok{ }\KeywordTok{ceiling}\NormalTok{(}\KeywordTok{sqrt}\NormalTok{(n <-}\KeywordTok{length}\NormalTok{(cl)))}
    \KeywordTok{length}\NormalTok{(cl) <-}\StringTok{ }\NormalTok{m}\OperatorTok{*}\NormalTok{m; cm <-}\StringTok{ }\KeywordTok{matrix}\NormalTok{(cl, m)}
    \KeywordTok{require}\NormalTok{(}\StringTok{"grid"}\NormalTok{)}
    \KeywordTok{grid.newpage}\NormalTok{(); vp <-}\StringTok{ }\KeywordTok{viewport}\NormalTok{(}\DataTypeTok{w =} \FloatTok{.92}\NormalTok{, }\DataTypeTok{h =} \FloatTok{.92}\NormalTok{)}
    \KeywordTok{grid.rect}\NormalTok{(}\DataTypeTok{gp=}\KeywordTok{gpar}\NormalTok{(}\DataTypeTok{fill=}\NormalTok{bg))}
    \KeywordTok{grid.text}\NormalTok{(cm, }\DataTypeTok{x =} \KeywordTok{col}\NormalTok{(cm)}\OperatorTok{/}\NormalTok{m, }\DataTypeTok{y =} \KeywordTok{rev}\NormalTok{(}\KeywordTok{row}\NormalTok{(cm))}\OperatorTok{/}\NormalTok{m, }\DataTypeTok{rot =}\NormalTok{ rot,}
              \DataTypeTok{vp=}\NormalTok{vp, }\DataTypeTok{gp=}\KeywordTok{gpar}\NormalTok{(}\DataTypeTok{cex =}\NormalTok{ cex, }\DataTypeTok{col =}\NormalTok{ cm))}
\NormalTok{\}}

\CommentTok{# print 60 nama warna pertama}
\KeywordTok{showCols}\NormalTok{(}\DataTypeTok{bg=}\StringTok{"gray20"}\NormalTok{, }\DataTypeTok{cl=}\KeywordTok{colors}\NormalTok{()[}\DecValTok{1}\OperatorTok{:}\DecValTok{60}\NormalTok{], }\DataTypeTok{rot=}\DecValTok{30}\NormalTok{, }\DataTypeTok{cex=}\FloatTok{0.9}\NormalTok{)}
\end{Highlighting}
\end{Shaded}

\begin{figure}

{\centering \includegraphics[width=0.7\linewidth]{Metode_Numerik_files/figure-latex/color-1} 

}

\caption{Nama warna}\label{fig:color}
\end{figure}

\hypertarget{plot-dua-dan-tiga-dimensi}{%
\section{Plot Dua dan Tiga Dimensi}\label{plot-dua-dan-tiga-dimensi}}

\texttt{R} dapat digunakan untuk memproduksi visualisasi pada skala 2 dan 3 dimensi. Untuk proyeksi 2 dimensi, fungsi yang digunakan adalah \texttt{image()} atau \texttt{contour()}. Untuk informasi lebih lanjut terkait fungsi tersebut pembaca dapat mengakses menu bantuan. Pada sintak berikut diberikan contoh bagaimana cara memproduksi visualisasi dua dimensi menggunakan kedua fungsi tersebut:

\begin{Shaded}
\begin{Highlighting}[]
\NormalTok{n <-}\StringTok{ }\DecValTok{1}\OperatorTok{:}\DecValTok{20}
\NormalTok{x <-}\StringTok{ }\KeywordTok{sin}\NormalTok{(n)}
\NormalTok{y <-}\StringTok{ }\KeywordTok{cos}\NormalTok{(n)}\OperatorTok{*}\KeywordTok{exp}\NormalTok{(}\OperatorTok{-}\NormalTok{n}\OperatorTok{/}\DecValTok{3}\NormalTok{)}
\NormalTok{z <-}\StringTok{ }\KeywordTok{outer}\NormalTok{(x,y)}
\KeywordTok{par}\NormalTok{(}\DataTypeTok{mar=}\KeywordTok{c}\NormalTok{(}\DecValTok{3}\NormalTok{,}\DecValTok{3}\NormalTok{,}\FloatTok{1.5}\NormalTok{,}\FloatTok{1.5}\NormalTok{), }\DataTypeTok{mex=}\FloatTok{0.8}\NormalTok{, }\DataTypeTok{mgp=}\KeywordTok{c}\NormalTok{(}\DecValTok{2}\NormalTok{,}\FloatTok{0.5}\NormalTok{,}\DecValTok{0}\NormalTok{), }\DataTypeTok{tcl=}\FloatTok{0.3}\NormalTok{)}
\KeywordTok{par}\NormalTok{(}\DataTypeTok{mfrow=}\KeywordTok{c}\NormalTok{(}\DecValTok{1}\NormalTok{,}\DecValTok{2}\NormalTok{))}

\CommentTok{# plot pertama}
\KeywordTok{image}\NormalTok{(z, }\DataTypeTok{col=}\KeywordTok{gray}\NormalTok{(}\DecValTok{1}\OperatorTok{:}\DecValTok{10}\OperatorTok{/}\DecValTok{10}\NormalTok{))}

\CommentTok{# plot kedua}
\KeywordTok{contour}\NormalTok{(z)}
\end{Highlighting}
\end{Shaded}

\begin{figure}

{\centering \includegraphics[width=0.8\linewidth]{Metode_Numerik_files/figure-latex/map-1} 

}

\caption{image plot (kiri) dan contour plot (kanan)}\label{fig:map}
\end{figure}

\begin{Shaded}
\begin{Highlighting}[]
\KeywordTok{par}\NormalTok{(}\DataTypeTok{mfrow=}\KeywordTok{c}\NormalTok{(}\DecValTok{1}\NormalTok{,}\DecValTok{1}\NormalTok{))}
\end{Highlighting}
\end{Shaded}

Proyeksi 3 dimensi dapat dilakukan menggunakan fungsi \texttt{persp()}. Sudut penglihatan dapat diatur melalui argumen\texttt{theta} (sudut) dan \texttt{phi()} (rotasi). Sintaks berikut merupakan contoh bagaimana cara menghasilkan visualisasi 3 dimensi dari data yang telah diproduksi sebelumnya:

\begin{Shaded}
\begin{Highlighting}[]
\KeywordTok{par}\NormalTok{(}\DataTypeTok{mar=}\KeywordTok{c}\NormalTok{(}\DecValTok{3}\NormalTok{,}\DecValTok{3}\NormalTok{,}\FloatTok{1.5}\NormalTok{,}\FloatTok{1.5}\NormalTok{), }\DataTypeTok{mex=}\FloatTok{0.8}\NormalTok{, }\DataTypeTok{mgp=}\KeywordTok{c}\NormalTok{(}\DecValTok{2}\NormalTok{,}\FloatTok{0.5}\NormalTok{,}\DecValTok{0}\NormalTok{), }\DataTypeTok{tcl=}\FloatTok{0.3}\NormalTok{)}
\KeywordTok{par}\NormalTok{(}\DataTypeTok{mfrow=}\KeywordTok{c}\NormalTok{(}\DecValTok{1}\NormalTok{,}\DecValTok{2}\NormalTok{))}

\CommentTok{# plot pertama}
\KeywordTok{persp}\NormalTok{(n,n,z, }\DataTypeTok{theta=}\DecValTok{45}\NormalTok{, }\DataTypeTok{phi=}\DecValTok{20}\NormalTok{)}

\CommentTok{# plot kedua}
\KeywordTok{persp}\NormalTok{(n,n,z, }\DataTypeTok{theta=}\DecValTok{45}\NormalTok{, }\DataTypeTok{phi=}\DecValTok{20}\NormalTok{, }\DataTypeTok{shade=}\FloatTok{0.5}\NormalTok{)}
\end{Highlighting}
\end{Shaded}

\begin{figure}

{\centering \includegraphics[width=0.8\linewidth]{Metode_Numerik_files/figure-latex/persp-1} 

}

\caption{proyeksi 3 dimensi (kanan) dan proyeksi 3 dimensi dengan pewarnaan}\label{fig:persp}
\end{figure}

\begin{Shaded}
\begin{Highlighting}[]
\KeywordTok{par}\NormalTok{(}\DataTypeTok{mfrow=}\KeywordTok{c}\NormalTok{(}\DecValTok{1}\NormalTok{,}\DecValTok{1}\NormalTok{))}
\end{Highlighting}
\end{Shaded}

\hypertarget{referensi-2}{%
\section{Referensi}\label{referensi-2}}

\begin{enumerate}
\def\labelenumi{\arabic{enumi}.}
\tightlist
\item
  Maindonald, J.H. 2008. \textbf{Using R for Data Analysis and Graphics Introduction, Code and Commentary}. Centre for Mathematics and Its Applications Australian National University.
\item
  Scherber, C. 2007. \textbf{An introduction to statistical data analysis using R}. R\_Manual Goettingen.
\item
  STHDA. \textbf{R Base Graphs}. \url{http://www.sthda.com/english/wiki/r-base-graphs}
\item
  Venables, W.N. Smith D.M. and R Core Team. 2018. \textbf{An Introduction to R}. R Manuals.
\end{enumerate}

\hypertarget{programmingandfunction}{%
\chapter{Pemrograman dan Fungsi}\label{programmingandfunction}}

Kita telah membahas dasar-dasar kalkulasi menggunakan \texttt{R} pada Chapter \ref{calculation}. Pada Chapter \ref{programmingandfunction} kita akan membahas dasar pemrograman menggunakan \texttt{R}. Pada chapter ini kita juga akan membahas bagaimana kita dapat membentuk suatu fungsi menggunakan \texttt{R} untuk pekerjaan yang berulang-ulang.

\hypertarget{loop}{%
\section{Loop}\label{loop}}

\emph{Loop} merupakan kode program yang berulang-ulang. \emph{Loop} berguna saat kita ingin melakukan sebuah perintah yang perlu dijalankan berulang-ulang seperti melakukan perhitungan maupaun melakukan visualisasi terhadap banyak variabel secara serentak. Hal ini tentu saja membantu kita karena kita tidak perlu menulis sejumlah sintaks yang berulang-ulang. Kita hanya perlu mengatur \emph{statement} berdasarkan hasil yang kita harapkan.

Pada \texttt{R} bentuk \emph{loop} dapat bermacam-macam (``\emph{for loop}'',``\emph{while loop}'', dll). \texttt{R} menyederhanakan bentuk \emph{loop} ini dengan menyediakan sejumlah fungsi seperti \texttt{apply()},\texttt{tapply()}, dll. Sehingga \texttt{loop} jarang sekali muncul dalam kode \texttt{R}. Sehingga \texttt{R} sering disebut sebagai \emph{loopless loop}.

Meski \emph{loop} jarang muncul bukan berarti kita tidak akan melakukannya. Terkadang saat kita melakukan komputasi statistik atau matematik dan belum terdapat paket yang mendukung proses tersebut, sering kali kita akan membuat sintaks sendiri berdasarkan algoritma metode tersebut. Pada algoritma tersebut sering pula terdapat \emph{loop} yang diperlukan selama proses perhitungan. Secara sederhana diagram umum loop ditampilkan pada Gambar \ref{fig:loop}

\begin{figure}

{\centering \includegraphics[width=0.4\linewidth]{./images/skema_loop} 

}

\caption{Diagram umum loop (sumber: Primartha, 2018).}\label{fig:loop}
\end{figure}

\hypertarget{forloop}{%
\subsection{For Loop}\label{forloop}}

Mengulangi sebuah \emph{statement} atau sekelompok \emph{statement} sebanyak nilai yang ditentukan di awal. Jadi operasi akan terus dilakukan sampai dengan jumlah yang telah ditetapkan di awal atau dengan kata lain tes kondisi (Jika jumlah pengulangan telah cukup) hanya akan dilakukan di akhir. Secara sederhana bentuk dari \emph{for loop} dapat dituliskan sebagai berikut:

\begin{Shaded}
\begin{Highlighting}[]
\ControlFlowTok{for}\NormalTok{ (value }\ControlFlowTok{in}\NormalTok{ vector)\{}
\NormalTok{  statements}
\NormalTok{\}}
\end{Highlighting}
\end{Shaded}

Berikut adalah contoh sintaks penerapan \emph{for loop}:

\begin{Shaded}
\begin{Highlighting}[]
\CommentTok{# Membuat vektor numerik}
\NormalTok{vektor <-}\StringTok{ }\KeywordTok{c}\NormalTok{(}\DecValTok{1}\OperatorTok{:}\DecValTok{5}\NormalTok{)}

\CommentTok{# loop }
\ControlFlowTok{for}\NormalTok{(i }\ControlFlowTok{in}\NormalTok{ vektor)\{}
  \KeywordTok{print}\NormalTok{(i)}
\NormalTok{\}}
\end{Highlighting}
\end{Shaded}

\begin{verbatim}
## [1] 1
## [1] 2
## [1] 3
## [1] 4
## [1] 5
\end{verbatim}

\emph{Loop} akan dimulai dari blok \emph{statement for} sampai dengan \texttt{print(i)}. Berdasarkan \emph{loop} pada contoh tersebut, \emph{loop} hanya dilakukan sebanyak 5 kali sesuai dengan jumlah vektor yang ada.

\hypertarget{whileloop}{%
\subsection{While Loop}\label{whileloop}}

\emph{While loop} merupakan loop yang digunakan ketika kita telah menetapkan \emph{stop condition} sebelumnya. Blok \emph{statement}/kode yang sama akan terus dijalankan sampai \emph{stop condition} ini tercapai. \emph{Stop condition} akan di cek sebelum melakukan proses \emph{loop}. Berikut adalah pola dari \emph{while loop} dapat dituliskan sebagai berikut:

\begin{Shaded}
\begin{Highlighting}[]
\ControlFlowTok{while}\NormalTok{ (test_expression)\{}
\NormalTok{  statement}
\NormalTok{\}}
\end{Highlighting}
\end{Shaded}

Berikut adalah contoh penerapan dari \emph{while loop}:

\begin{Shaded}
\begin{Highlighting}[]
\NormalTok{coba <-}\StringTok{ }\KeywordTok{c}\NormalTok{(}\StringTok{"Contoh"}\NormalTok{)}
\NormalTok{counter <-}\StringTok{ }\DecValTok{1}

\CommentTok{# loop}
\ControlFlowTok{while}\NormalTok{ (counter}\OperatorTok{<}\DecValTok{5}\NormalTok{)\{}
  \CommentTok{# print vektor}
  \KeywordTok{print}\NormalTok{(coba)}
  \CommentTok{# tambahkan nilai counter sehingga proses terus berlangsung sampai counter = 5 }
\NormalTok{  counter <-}\StringTok{ }\NormalTok{counter }\OperatorTok{+}\StringTok{ }\DecValTok{1}
\NormalTok{\}}
\end{Highlighting}
\end{Shaded}

\begin{verbatim}
## [1] "Contoh"
## [1] "Contoh"
## [1] "Contoh"
## [1] "Contoh"
\end{verbatim}

\emph{Loop} akan dimulai dari blok \emph{statement while} sampai dengan \emph{counter} \textless{}- 1. \emph{Loop} hanya akan dilakukan sepanjang nilai \emph{counter} \textless{} 5.

\hypertarget{repeatloop}{%
\subsection{Repeat Loop}\label{repeatloop}}

\emph{Repeat loop} akan menjalankan \emph{statement}/kode yang sama berulang-ulang hingga \emph{stop condition} tercapai. Berikut adalah pola dari \emph{repeat loop}.

\begin{Shaded}
\begin{Highlighting}[]
\ControlFlowTok{repeat}\NormalTok{ \{}
\NormalTok{  commands}
  \ControlFlowTok{if}\NormalTok{(condition)\{}
    \ControlFlowTok{break}
\NormalTok{  \}}
\NormalTok{\}}
\end{Highlighting}
\end{Shaded}

Berikut adalah contoh penerapan dari \emph{repeat loop}:

\begin{Shaded}
\begin{Highlighting}[]
\NormalTok{coba <-}\StringTok{ }\KeywordTok{c}\NormalTok{(}\StringTok{"contoh"}\NormalTok{)}
\NormalTok{counter <-}\StringTok{ }\DecValTok{1}
\ControlFlowTok{repeat}\NormalTok{ \{}
  \KeywordTok{print}\NormalTok{(coba)}
\NormalTok{  counter <-}\StringTok{ }\NormalTok{counter }\OperatorTok{+}\StringTok{ }\DecValTok{1}
  \ControlFlowTok{if}\NormalTok{(counter }\OperatorTok{<}\StringTok{ }\DecValTok{5}\NormalTok{)\{}
\ControlFlowTok{break}
\NormalTok{  \}}
\NormalTok{\}}
\end{Highlighting}
\end{Shaded}

\begin{verbatim}
## [1] "contoh"
\end{verbatim}

\emph{Loop} akan dimulai dari blok \emph{statement while} sampai dengan \emph{break}. \emph{Loop} hanya akan dilakukan sepanjang nilai \emph{counter} \textless{} 5. Hasil yang diperoleh berbeda dengan \emph{while loop}, dimana kita memperoleh 4 buah kata ``contoh''. Hal ini disebabkan karena \emph{repeat loop} melakukan pengecekan \emph{stop condition} tidak di awal loop seperti \emph{while loop} sehingga berapapun nilainya, selama nilainya sesuai dengan \emph{stop condition} maka \emph{loop} akan dihentikan. Hal ini berbeda dengan \emph{while loop} dimana proses dilakukan berulang-ulang sampai jumlahnya mendekati \emph{stop condition}.

\hypertarget{break}{%
\subsection{Break}\label{break}}

\emph{Break} sebenarnya bukan bagian dari \emph{loop}, namun sering digunakan dalam \emph{loop}. \emph{Break} dapat digunakan pada \emph{loop} manakala dirasa perlu, yaitu saat kondisi yang disyaratkan pada \emph{break} tercapai.

Berikut adalah contoh penerapan \emph{break} pada beberapa jenis \emph{loop}.

\begin{Shaded}
\begin{Highlighting}[]
\CommentTok{# for loop}
\NormalTok{a =}\StringTok{ }\KeywordTok{c}\NormalTok{(}\DecValTok{2}\NormalTok{,}\DecValTok{4}\NormalTok{,}\DecValTok{6}\NormalTok{,}\DecValTok{8}\NormalTok{,}\DecValTok{10}\NormalTok{,}\DecValTok{12}\NormalTok{,}\DecValTok{14}\NormalTok{)}
\ControlFlowTok{for}\NormalTok{(i }\ControlFlowTok{in}\NormalTok{ a)\{}
  \ControlFlowTok{if}\NormalTok{(i}\OperatorTok{>}\DecValTok{8}\NormalTok{)\{}
    \ControlFlowTok{break}
\NormalTok{  \}}
  \KeywordTok{print}\NormalTok{(i)}
\NormalTok{\}}
\end{Highlighting}
\end{Shaded}

\begin{verbatim}
## [1] 2
## [1] 4
## [1] 6
## [1] 8
\end{verbatim}

\begin{Shaded}
\begin{Highlighting}[]
\CommentTok{# while loop}
\NormalTok{a =}\StringTok{ }\DecValTok{2}
\NormalTok{b =}\StringTok{ }\DecValTok{4}
\ControlFlowTok{while}\NormalTok{(a}\OperatorTok{<}\DecValTok{7}\NormalTok{)\{}
  \KeywordTok{print}\NormalTok{(a)}
\NormalTok{  a =}\StringTok{ }\NormalTok{a }\OperatorTok{+}\DecValTok{1}
  \ControlFlowTok{if}\NormalTok{(b}\OperatorTok{+}\NormalTok{a}\OperatorTok{>}\DecValTok{10}\NormalTok{)\{}
    \ControlFlowTok{break}
\NormalTok{  \}}
\NormalTok{\}}
\end{Highlighting}
\end{Shaded}

\begin{verbatim}
## [1] 2
## [1] 3
## [1] 4
## [1] 5
## [1] 6
\end{verbatim}

\begin{Shaded}
\begin{Highlighting}[]
\CommentTok{# repeat loop}
\NormalTok{a =}\StringTok{ }\DecValTok{1}
\ControlFlowTok{repeat}\NormalTok{\{}
  \KeywordTok{print}\NormalTok{(a)}
\NormalTok{  a =}\StringTok{ }\NormalTok{a}\OperatorTok{+}\DecValTok{1}
  \ControlFlowTok{if}\NormalTok{(a}\OperatorTok{>}\DecValTok{6}\NormalTok{)\{}
    \ControlFlowTok{break}
\NormalTok{  \}}
\NormalTok{\}}
\end{Highlighting}
\end{Shaded}

\begin{verbatim}
## [1] 1
## [1] 2
## [1] 3
## [1] 4
## [1] 5
## [1] 6
\end{verbatim}

\hypertarget{loopapply}{%
\section{Loop Menggunakan Apply Family Function}\label{loopapply}}

Penggunaan loop sangat membantu kita dalam melakukan proses perhitungan berulang. Namun, metode ini tidak cukup ringkas dalam penerapannya dan perlu penulisan sintaks yang cukup panjang untuk menyelesaikan sebuah kasus yang kita inginkan. Berikut adalah sebuah sintaks yang digunakan untuk menghitung nilai mean pada suatu dataset:

\begin{Shaded}
\begin{Highlighting}[]
\CommentTok{# subset data iris}
\NormalTok{sub_iris <-}\StringTok{ }\NormalTok{iris[,}\OperatorTok{-}\DecValTok{5}\NormalTok{]}
\CommentTok{# membuat vektor untuk menyimpan hasil loop}
\NormalTok{a <-}\StringTok{ }\KeywordTok{rep}\NormalTok{(}\OtherTok{NA}\NormalTok{,}\DecValTok{4}\NormalTok{)}
\CommentTok{# loop}
\ControlFlowTok{for}\NormalTok{(i }\ControlFlowTok{in} \DecValTok{1}\OperatorTok{:}\KeywordTok{length}\NormalTok{(sub_iris))\{}
\NormalTok{  a[i]<-}\KeywordTok{mean}\NormalTok{(sub_iris[,i])}
\NormalTok{\}}
\CommentTok{# print}
\NormalTok{a}
\end{Highlighting}
\end{Shaded}

\begin{verbatim}
## [1] 5.843333 3.057333 3.758000 1.199333
\end{verbatim}

\begin{Shaded}
\begin{Highlighting}[]
\KeywordTok{class}\NormalTok{(a) }\CommentTok{# cek kelas objek}
\end{Highlighting}
\end{Shaded}

\begin{verbatim}
## [1] "numeric"
\end{verbatim}

Metode alternatif lain untuk melakukan loop suatu fungsi adalah dengan menggunakan Apply function family. Metode ini memungkinkan kita untuk melakukan loop suatu fungsi tanpa perlu menuliskan sintaks loop. Berikut adalah beberapa fungsi dari apply family yang nantinya akan sering kita gunakan:

\begin{itemize}
\tightlist
\item
  \texttt{apply()}: fungsi generik yang mengaplikasikan fungsi kepada kolom atau baris pada matriks atau secara lebih general aplikasi dilakukan pada dimensi untuk jenis data array.
\item
  \texttt{lapply()}: fungsi apply yang bekerja pada jenis data list dan memberikan output berupa list juga.
\item
  \texttt{sapply()}: bentuk sederhana dari lapply yang menghasilkan output berupa matriks atau vektor.
\item
  \texttt{vapply()}: disebut juga \emph{verified apply} (memungkinkan untuk menghasilkan output dengan jenis data yang telah ditentukan sebelumnya).
\item
  \texttt{tapply()}: \emph{tagged apply} dimana dimana tag menentukan subset dari data.
\end{itemize}

\hypertarget{apply}{%
\subsection{Apply}\label{apply}}

Fungsi \texttt{apply()} bekerja dengan jenis data matrik atau array (jenis data homogen). Kita dapat melakukan spesifikasi apakah suatu fungsi hanya akan bekerja pada kolom saja, baris saja atau keduanya. Format fungsi ini adalah sebagai berikut:

\begin{Shaded}
\begin{Highlighting}[]
\KeywordTok{apply}\NormalTok{(X, MARGIN, FUN, ...)}
\end{Highlighting}
\end{Shaded}

\begin{quote}
\textbf{Catatan:}

\begin{itemize}
\tightlist
\item
  \textbf{X}: matriks atau array
\item
  \textbf{MARGIN}: menentukan bagaimana fungsi bekerja terhadap matriks atau array. Jika nilai yang diinputkan 1, maka fungsi akan bekerja pada masing-masing baris pada matriks. Jika nilainya 2, maka fungsi akan bekerja pada tiap kolom pada matriks.
\item
  \textbf{FUN}: fungsi yang akan digunakan. Fungsi yang dapat digunakan dapat berupa fungsi dasar matematika atau statistika, serta user define function.
\item
  \textbf{\ldots{}}: opsional argumen pada fungsi yang digunakan.
\end{itemize}
\end{quote}

Berikut adalah contoh bagaimana aplikasi fungsi tersebut pada matriks:

\begin{Shaded}
\begin{Highlighting}[]
\CommentTok{## membuat matriks}
\NormalTok{x <-}\StringTok{ }\KeywordTok{cbind}\NormalTok{(}\DataTypeTok{x1 =} \DecValTok{3}\NormalTok{, }\DataTypeTok{x2 =} \KeywordTok{c}\NormalTok{(}\DecValTok{4}\OperatorTok{:}\DecValTok{1}\NormalTok{, }\DecValTok{2}\OperatorTok{:}\DecValTok{5}\NormalTok{))}
\NormalTok{x }\CommentTok{# print}
\end{Highlighting}
\end{Shaded}

\begin{verbatim}
##      x1 x2
## [1,]  3  4
## [2,]  3  3
## [3,]  3  2
## [4,]  3  1
## [5,]  3  2
## [6,]  3  3
## [7,]  3  4
## [8,]  3  5
\end{verbatim}

\begin{Shaded}
\begin{Highlighting}[]
\KeywordTok{class}\NormalTok{(x) }\CommentTok{# cek kelas objek}
\end{Highlighting}
\end{Shaded}

\begin{verbatim}
## [1] "matrix"
\end{verbatim}

\begin{Shaded}
\begin{Highlighting}[]
\CommentTok{## menghitung mean masing-masing kolom}
\KeywordTok{apply}\NormalTok{(x, }\DataTypeTok{MARGIN=}\DecValTok{2}\NormalTok{ ,}\DataTypeTok{FUN=}\NormalTok{mean, }\DataTypeTok{trim=}\FloatTok{0.2}\NormalTok{, }\DataTypeTok{na.rm=}\OtherTok{TRUE}\NormalTok{)}
\end{Highlighting}
\end{Shaded}

\begin{verbatim}
## x1 x2 
##  3  3
\end{verbatim}

\begin{Shaded}
\begin{Highlighting}[]
\CommentTok{## menghitung range pada masing-masing baris}
\CommentTok{## menggunakan user define function}
\KeywordTok{apply}\NormalTok{(x, }\DataTypeTok{MARGIN=}\DecValTok{1}\NormalTok{,}
      \DataTypeTok{FUN=}\ControlFlowTok{function}\NormalTok{(x)\{}
        \KeywordTok{max}\NormalTok{(x)}\OperatorTok{-}\KeywordTok{min}\NormalTok{(x)}
\NormalTok{      \})}
\end{Highlighting}
\end{Shaded}

\begin{verbatim}
## [1] 1 0 1 2 1 0 1 2
\end{verbatim}

\hypertarget{lapply}{%
\subsection{lapply}\label{lapply}}

Fungsi ini melakukan loop fungsi terhadap input data berupa list. Output yang dihasilkan juga merupakan list dengan panjang list yang sama dengan yang diinputkan. Format yang digunakan adalah sebagai berikut:

\begin{Shaded}
\begin{Highlighting}[]
\KeywordTok{lapply}\NormalTok{(X, FUN, ...)}
\end{Highlighting}
\end{Shaded}

\begin{quote}
\textbf{Catatan:}

\begin{itemize}
\tightlist
\item
  \textbf{X}: vektor, data frame atau list
\item
  \textbf{FUN}: fungsi yang akan digunakan. Fungsi yang dapat digunakan dapat berupa fungsi dasar matematika atau statistika, serta user define function. Subset juga dimungkinkan pada fungsi ini.
\item
  \textbf{\ldots{}}: opsional argumen pada fungsi yang digunakan.
\end{itemize}
\end{quote}

Berikut adalah contoh penerapan fungsi lapply:

\begin{Shaded}
\begin{Highlighting}[]
\CommentTok{## Membuat list}
\NormalTok{x <-}\StringTok{ }\KeywordTok{list}\NormalTok{(}\DataTypeTok{a =} \DecValTok{1}\OperatorTok{:}\DecValTok{10}\NormalTok{, }\DataTypeTok{beta =} \KeywordTok{exp}\NormalTok{(}\OperatorTok{-}\DecValTok{3}\OperatorTok{:}\DecValTok{3}\NormalTok{), }\DataTypeTok{logic =} \KeywordTok{c}\NormalTok{(}\OtherTok{TRUE}\NormalTok{,}\OtherTok{FALSE}\NormalTok{,}\OtherTok{FALSE}\NormalTok{,}\OtherTok{TRUE}\NormalTok{))}
\NormalTok{x }\CommentTok{# print}
\end{Highlighting}
\end{Shaded}

\begin{verbatim}
## $a
##  [1]  1  2  3  4  5  6  7  8  9 10
## 
## $beta
## [1]  0.04978707  0.13533528  0.36787944  1.00000000  2.71828183  7.38905610
## [7] 20.08553692
## 
## $logic
## [1]  TRUE FALSE FALSE  TRUE
\end{verbatim}

\begin{Shaded}
\begin{Highlighting}[]
\KeywordTok{class}\NormalTok{(x) }\CommentTok{# cek kelas objek}
\end{Highlighting}
\end{Shaded}

\begin{verbatim}
## [1] "list"
\end{verbatim}

\begin{Shaded}
\begin{Highlighting}[]
\CommentTok{## Menghitung nilai mean pada masing-masing baris lits}
\KeywordTok{lapply}\NormalTok{(x, }\DataTypeTok{FUN=}\NormalTok{mean)}
\end{Highlighting}
\end{Shaded}

\begin{verbatim}
## $a
## [1] 5.5
## 
## $beta
## [1] 4.535125
## 
## $logic
## [1] 0.5
\end{verbatim}

\begin{Shaded}
\begin{Highlighting}[]
\CommentTok{## Menghitung mean tiap kolom dataset iris}
\KeywordTok{lapply}\NormalTok{(iris, }\DataTypeTok{FUN=}\NormalTok{mean)}
\end{Highlighting}
\end{Shaded}

\begin{verbatim}
## Warning in mean.default(X[[i]], ...): argument is not numeric or logical:
## returning NA
\end{verbatim}

\begin{verbatim}
## $Sepal.Length
## [1] 5.843333
## 
## $Sepal.Width
## [1] 3.057333
## 
## $Petal.Length
## [1] 3.758
## 
## $Petal.Width
## [1] 1.199333
## 
## $Species
## [1] NA
\end{verbatim}

\begin{Shaded}
\begin{Highlighting}[]
\CommentTok{## Mengalikan elemen vektor dengan suatu nilai}
\NormalTok{y <-}\StringTok{ }\KeywordTok{c}\NormalTok{(}\DecValTok{1}\OperatorTok{:}\DecValTok{5}\NormalTok{)}
\KeywordTok{lapply}\NormalTok{(y, }\DataTypeTok{FUN=}\ControlFlowTok{function}\NormalTok{(x)\{x}\OperatorTok{*}\DecValTok{5}\NormalTok{\})}
\end{Highlighting}
\end{Shaded}

\begin{verbatim}
## [[1]]
## [1] 5
## 
## [[2]]
## [1] 10
## 
## [[3]]
## [1] 15
## 
## [[4]]
## [1] 20
## 
## [[5]]
## [1] 25
\end{verbatim}

\begin{Shaded}
\begin{Highlighting}[]
\CommentTok{## Mengubah output menjadi vektor}
\KeywordTok{unlist}\NormalTok{(}\KeywordTok{lapply}\NormalTok{(y, }\DataTypeTok{FUN=}\ControlFlowTok{function}\NormalTok{(x)\{x}\OperatorTok{*}\DecValTok{5}\NormalTok{\}))}
\end{Highlighting}
\end{Shaded}

\begin{verbatim}
## [1]  5 10 15 20 25
\end{verbatim}

\hypertarget{sapply}{%
\subsection{sapply}\label{sapply}}

Fungsi \texttt{sapply()} merupakan bentuk lain dari fungsi \texttt{lapply()}. Perbedaanya terletak pada output default yang dihasilkan. Secara default \texttt{sapply()} menerima input utama berupa list (dapat pula dataframe atau vektor), namun tidak seperti \texttt{lapply()} jenis data output yang dihasilkan adalah vektor. Untuk mengubah output menjadi list perlu argumen tambahan berupa \texttt{simplify=FALSE}. Format fungsi tersebut adalah sebagai berikut:

\begin{Shaded}
\begin{Highlighting}[]
\KeywordTok{sapply}\NormalTok{(X, FUN, ..., }\DataTypeTok{simplify =} \OtherTok{TRUE}\NormalTok{, }\DataTypeTok{USE.NAMES =} \OtherTok{TRUE}\NormalTok{)}
\end{Highlighting}
\end{Shaded}

\begin{quote}
\textbf{Catatan:}

\begin{itemize}
\tightlist
\item
  \textbf{X}: vektor, data frame atau list
\item
  \textbf{FUN}: fungsi yang akan digunakan. Fungsi yang dapat digunakan dapat berupa fungsi dasar matematika atau statistika, serta user define function. Subset juga dimungkinkan pada fungsi ini.
\item
  \textbf{\ldots{}}: opsional argumen pada fungsi yang digunakan.
\item
  \textbf{simplify}: logical. Jika nilainya \texttt{TRUE} maka output yang dihasilkan adalah bentuk sederhana dari vektor, matrix atau array.
\item
  \textbf{USE.NAMES}: jika list memiliki nama pada setiap elemennya, maka nama elemen tersebut akan secara default ditampilkan.
\end{itemize}
\end{quote}

Berikut adalah contoh penerapannya:

\begin{Shaded}
\begin{Highlighting}[]
\CommentTok{## membuat list}
\NormalTok{x <-}\StringTok{ }\KeywordTok{list}\NormalTok{(}\DataTypeTok{a =} \DecValTok{1}\OperatorTok{:}\DecValTok{10}\NormalTok{, }\DataTypeTok{beta =} \KeywordTok{exp}\NormalTok{(}\OperatorTok{-}\DecValTok{3}\OperatorTok{:}\DecValTok{3}\NormalTok{), }\DataTypeTok{logic =} \KeywordTok{c}\NormalTok{(}\OtherTok{TRUE}\NormalTok{,}\OtherTok{FALSE}\NormalTok{,}\OtherTok{FALSE}\NormalTok{,}\OtherTok{TRUE}\NormalTok{))}

\CommentTok{## menghitung nilai mean setiap elemen}
\KeywordTok{sapply}\NormalTok{(x, }\DataTypeTok{FUN=}\NormalTok{mean)}
\end{Highlighting}
\end{Shaded}

\begin{verbatim}
##        a     beta    logic 
## 5.500000 4.535125 0.500000
\end{verbatim}

\begin{Shaded}
\begin{Highlighting}[]
\CommentTok{## menghitung nilai mean dengan output list}
\KeywordTok{sapply}\NormalTok{(x, }\DataTypeTok{FUN=}\NormalTok{mean, }\DataTypeTok{simplify=}\OtherTok{FALSE}\NormalTok{)}
\end{Highlighting}
\end{Shaded}

\begin{verbatim}
## $a
## [1] 5.5
## 
## $beta
## [1] 4.535125
## 
## $logic
## [1] 0.5
\end{verbatim}

\begin{Shaded}
\begin{Highlighting}[]
\CommentTok{## summary objek dataframe}
\KeywordTok{sapply}\NormalTok{(mtcars, }\DataTypeTok{FUN=}\NormalTok{summary)}
\end{Highlighting}
\end{Shaded}

\begin{verbatim}
##              mpg    cyl     disp       hp     drat      wt     qsec     vs
## Min.    10.40000 4.0000  71.1000  52.0000 2.760000 1.51300 14.50000 0.0000
## 1st Qu. 15.42500 4.0000 120.8250  96.5000 3.080000 2.58125 16.89250 0.0000
## Median  19.20000 6.0000 196.3000 123.0000 3.695000 3.32500 17.71000 0.0000
## Mean    20.09062 6.1875 230.7219 146.6875 3.596563 3.21725 17.84875 0.4375
## 3rd Qu. 22.80000 8.0000 326.0000 180.0000 3.920000 3.61000 18.90000 1.0000
## Max.    33.90000 8.0000 472.0000 335.0000 4.930000 5.42400 22.90000 1.0000
##              am   gear   carb
## Min.    0.00000 3.0000 1.0000
## 1st Qu. 0.00000 3.0000 2.0000
## Median  0.00000 4.0000 2.0000
## Mean    0.40625 3.6875 2.8125
## 3rd Qu. 1.00000 4.0000 4.0000
## Max.    1.00000 5.0000 8.0000
\end{verbatim}

\begin{Shaded}
\begin{Highlighting}[]
\CommentTok{## summary objek list}
\NormalTok{a <-}\StringTok{ }\KeywordTok{list}\NormalTok{(}\DataTypeTok{mobil=}\NormalTok{mtcars, }\DataTypeTok{anggrek=}\NormalTok{iris)}
\KeywordTok{sapply}\NormalTok{(a, }\DataTypeTok{FUN=}\NormalTok{summary)}
\end{Highlighting}
\end{Shaded}

\begin{verbatim}
## $mobil
##       mpg             cyl             disp             hp       
##  Min.   :10.40   Min.   :4.000   Min.   : 71.1   Min.   : 52.0  
##  1st Qu.:15.43   1st Qu.:4.000   1st Qu.:120.8   1st Qu.: 96.5  
##  Median :19.20   Median :6.000   Median :196.3   Median :123.0  
##  Mean   :20.09   Mean   :6.188   Mean   :230.7   Mean   :146.7  
##  3rd Qu.:22.80   3rd Qu.:8.000   3rd Qu.:326.0   3rd Qu.:180.0  
##  Max.   :33.90   Max.   :8.000   Max.   :472.0   Max.   :335.0  
##       drat             wt             qsec             vs        
##  Min.   :2.760   Min.   :1.513   Min.   :14.50   Min.   :0.0000  
##  1st Qu.:3.080   1st Qu.:2.581   1st Qu.:16.89   1st Qu.:0.0000  
##  Median :3.695   Median :3.325   Median :17.71   Median :0.0000  
##  Mean   :3.597   Mean   :3.217   Mean   :17.85   Mean   :0.4375  
##  3rd Qu.:3.920   3rd Qu.:3.610   3rd Qu.:18.90   3rd Qu.:1.0000  
##  Max.   :4.930   Max.   :5.424   Max.   :22.90   Max.   :1.0000  
##        am              gear            carb      
##  Min.   :0.0000   Min.   :3.000   Min.   :1.000  
##  1st Qu.:0.0000   1st Qu.:3.000   1st Qu.:2.000  
##  Median :0.0000   Median :4.000   Median :2.000  
##  Mean   :0.4062   Mean   :3.688   Mean   :2.812  
##  3rd Qu.:1.0000   3rd Qu.:4.000   3rd Qu.:4.000  
##  Max.   :1.0000   Max.   :5.000   Max.   :8.000  
## 
## $anggrek
##   Sepal.Length    Sepal.Width     Petal.Length    Petal.Width   
##  Min.   :4.300   Min.   :2.000   Min.   :1.000   Min.   :0.100  
##  1st Qu.:5.100   1st Qu.:2.800   1st Qu.:1.600   1st Qu.:0.300  
##  Median :5.800   Median :3.000   Median :4.350   Median :1.300  
##  Mean   :5.843   Mean   :3.057   Mean   :3.758   Mean   :1.199  
##  3rd Qu.:6.400   3rd Qu.:3.300   3rd Qu.:5.100   3rd Qu.:1.800  
##  Max.   :7.900   Max.   :4.400   Max.   :6.900   Max.   :2.500  
##        Species  
##  setosa    :50  
##  versicolor:50  
##  virginica :50  
##                 
##                 
## 
\end{verbatim}

\hypertarget{vapply}{%
\subsection{vapply}\label{vapply}}

Funsgi ini merupakan bentuk lain dari \texttt{sapply()}. Bedanya secara kecepatan proses fungsi ini lebih cepat dari \texttt{sapply()}. Hal yang menarik dari fungsi ini kita dapat menambahkan argumen \texttt{FUN.VALUE}. pada argumen ini kita memasukkan vektor berupa output fungsi yang diinginkan. Perbedaan lainnya adalah output yang dihasilkan hanya berupa matriks atau array. Format dari fungsi ini adalah sebagai berikut:

\begin{Shaded}
\begin{Highlighting}[]
\KeywordTok{vapply}\NormalTok{(X, FUN, FUN.VALUE, ..., }\DataTypeTok{USE.NAMES =} \OtherTok{TRUE}\NormalTok{)}
\end{Highlighting}
\end{Shaded}

\begin{quote}
\textbf{Catatan:}

\begin{itemize}
\tightlist
\item
  \textbf{X}: vektor, data frame atau list
\item
  \textbf{FUN}: fungsi yang akan digunakan. Fungsi yang dapat digunakan dapat berupa fungsi dasar matematika atau statistika, serta user define function. Subset juga dimungkinkan pada fungsi ini.
\item
  \textbf{FUN.VALUE}: vektor, template dari return value FUN.
\item
  \textbf{\ldots{}}: opsional argumen pada fungsi yang digunakan.
\item
  \textbf{USE.NAMES}: jika list memiliki nama pada setiap elemennya, maka nama elemen tersebut akan secara default ditampilkan.
\end{itemize}
\end{quote}

Berikut adalah contoh penerapannya:

\begin{Shaded}
\begin{Highlighting}[]
\CommentTok{## membuat list}
\NormalTok{x <-}\StringTok{ }\KeywordTok{sapply}\NormalTok{(}\DecValTok{3}\OperatorTok{:}\DecValTok{9}\NormalTok{, seq)}
\NormalTok{x }\CommentTok{# print}
\end{Highlighting}
\end{Shaded}

\begin{verbatim}
## [[1]]
## [1] 1 2 3
## 
## [[2]]
## [1] 1 2 3 4
## 
## [[3]]
## [1] 1 2 3 4 5
## 
## [[4]]
## [1] 1 2 3 4 5 6
## 
## [[5]]
## [1] 1 2 3 4 5 6 7
## 
## [[6]]
## [1] 1 2 3 4 5 6 7 8
## 
## [[7]]
## [1] 1 2 3 4 5 6 7 8 9
\end{verbatim}

\begin{Shaded}
\begin{Highlighting}[]
\CommentTok{## membuat ringkasan data pada tiap elemen list}
\KeywordTok{vapply}\NormalTok{(x, fivenum,}
       \KeywordTok{c}\NormalTok{(}\DataTypeTok{Min. =} \DecValTok{0}\NormalTok{, }\StringTok{"1st Qu."}\NormalTok{ =}\StringTok{ }\DecValTok{0}\NormalTok{, }
         \DataTypeTok{Median =} \DecValTok{0}\NormalTok{, }\StringTok{"3rd Qu."}\NormalTok{ =}\StringTok{ }\DecValTok{0}\NormalTok{, }\DataTypeTok{Max. =} \DecValTok{0}\NormalTok{))}
\end{Highlighting}
\end{Shaded}

\begin{verbatim}
##         [,1] [,2] [,3] [,4] [,5] [,6] [,7]
## Min.     1.0  1.0    1  1.0  1.0  1.0    1
## 1st Qu.  1.5  1.5    2  2.0  2.5  2.5    3
## Median   2.0  2.5    3  3.5  4.0  4.5    5
## 3rd Qu.  2.5  3.5    4  5.0  5.5  6.5    7
## Max.     3.0  4.0    5  6.0  7.0  8.0    9
\end{verbatim}

\begin{Shaded}
\begin{Highlighting}[]
\CommentTok{## membuat ringkasan data pada tiap kolom dataframe}
\KeywordTok{vapply}\NormalTok{(mtcars, summary,}
       \KeywordTok{c}\NormalTok{(}\DataTypeTok{Min. =} \DecValTok{0}\NormalTok{, }\StringTok{"1st Qu."}\NormalTok{ =}\StringTok{ }\DecValTok{0}\NormalTok{, }
         \DataTypeTok{Median =} \DecValTok{0}\NormalTok{, }\StringTok{"3rd Qu."}\NormalTok{ =}\StringTok{ }\DecValTok{0}\NormalTok{, }\DataTypeTok{Max. =} \DecValTok{0}\NormalTok{, }\DataTypeTok{Mean=}\DecValTok{0}\NormalTok{))}
\end{Highlighting}
\end{Shaded}

\begin{verbatim}
##              mpg    cyl     disp       hp     drat      wt     qsec     vs
## Min.    10.40000 4.0000  71.1000  52.0000 2.760000 1.51300 14.50000 0.0000
## 1st Qu. 15.42500 4.0000 120.8250  96.5000 3.080000 2.58125 16.89250 0.0000
## Median  19.20000 6.0000 196.3000 123.0000 3.695000 3.32500 17.71000 0.0000
## 3rd Qu. 20.09062 6.1875 230.7219 146.6875 3.596563 3.21725 17.84875 0.4375
## Max.    22.80000 8.0000 326.0000 180.0000 3.920000 3.61000 18.90000 1.0000
## Mean    33.90000 8.0000 472.0000 335.0000 4.930000 5.42400 22.90000 1.0000
##              am   gear   carb
## Min.    0.00000 3.0000 1.0000
## 1st Qu. 0.00000 3.0000 2.0000
## Median  0.00000 4.0000 2.0000
## 3rd Qu. 0.40625 3.6875 2.8125
## Max.    1.00000 4.0000 4.0000
## Mean    1.00000 5.0000 8.0000
\end{verbatim}

\hypertarget{tapply}{%
\subsection{tapply}\label{tapply}}

Fungsi ini sangat berguna jika pembaca ingin menghitung suatu nilai misalnya mean berdasarkan grup data atau factor. Format fungsi ini adalah sebagi berikut:

\begin{Shaded}
\begin{Highlighting}[]
\KeywordTok{tapply}\NormalTok{(X, INDEX, }\DataTypeTok{FUN =} \OtherTok{NULL}\NormalTok{, ..., }\DataTypeTok{simplify =} \OtherTok{TRUE}\NormalTok{)}
\end{Highlighting}
\end{Shaded}

\begin{quote}
\textbf{Catatan:}

\begin{itemize}
\tightlist
\item
  \textbf{X}: vektor, data frame atau list
\item
  \textbf{INDEX}: list satu atau beberapa factor yang memiliki panjang sama dengan X.
\item
  \textbf{FUN}: fungsi yang akan digunakan. Fungsi yang dapat digunakan dapat berupa fungsi dasar matematika atau statistika, serta user define function. Subset juga dimungkinkan pada fungsi ini.
\item
  \textbf{\ldots{}}: opsional argumen pada fungsi yang digunakan.
\item
  \textbf{simplify}: logical. Jika nilainya \texttt{TRUE} maka output yang dihasilkan adalah bentuk skalar.
\end{itemize}
\end{quote}

Berikut adalah contoh penerapannya:

\begin{Shaded}
\begin{Highlighting}[]
\CommentTok{## membuat tabel frekuensi}
\NormalTok{groups <-}\StringTok{ }\KeywordTok{as.factor}\NormalTok{(}\KeywordTok{rbinom}\NormalTok{(}\DecValTok{32}\NormalTok{, }\DataTypeTok{n =} \DecValTok{5}\NormalTok{, }\DataTypeTok{prob =} \FloatTok{0.4}\NormalTok{))}

\KeywordTok{tapply}\NormalTok{(groups, groups, length)}
\end{Highlighting}
\end{Shaded}

\begin{verbatim}
## 12 13 16 
##  2  2  1
\end{verbatim}

\begin{Shaded}
\begin{Highlighting}[]
\CommentTok{# atau}
\KeywordTok{table}\NormalTok{(groups)}
\end{Highlighting}
\end{Shaded}

\begin{verbatim}
## groups
## 12 13 16 
##  2  2  1
\end{verbatim}

\begin{Shaded}
\begin{Highlighting}[]
\CommentTok{## membuat tabel kontingensi}
\CommentTok{# menghitung jumlah breaks berdasarkan faktor jenis wool}
\CommentTok{# dan tensi level}
\KeywordTok{tapply}\NormalTok{(}\DataTypeTok{X=}\NormalTok{warpbreaks}\OperatorTok{$}\NormalTok{breaks, }\DataTypeTok{INDEX=}\NormalTok{warpbreaks[,}\OperatorTok{-}\DecValTok{1}\NormalTok{], }\DataTypeTok{FUN=}\NormalTok{sum)}
\end{Highlighting}
\end{Shaded}

\begin{verbatim}
##     tension
## wool   L   M   H
##    A 401 216 221
##    B 254 259 169
\end{verbatim}

\begin{Shaded}
\begin{Highlighting}[]
\CommentTok{# menghitung mean panjang gigi babi hutan berdasarkan}
\CommentTok{# jenis suplemen dan dosisnya}
\KeywordTok{tapply}\NormalTok{(ToothGrowth}\OperatorTok{$}\NormalTok{len, ToothGrowth[,}\OperatorTok{-}\DecValTok{1}\NormalTok{], mean)}
\end{Highlighting}
\end{Shaded}

\begin{verbatim}
##     dose
## supp   0.5     1     2
##   OJ 13.23 22.70 26.06
##   VC  7.98 16.77 26.14
\end{verbatim}

\begin{Shaded}
\begin{Highlighting}[]
\CommentTok{# menghitung mpg minimum berdasarkan jumlah silinder pada mobil}
\KeywordTok{tapply}\NormalTok{(mtcars}\OperatorTok{$}\NormalTok{mpg, mtcars}\OperatorTok{$}\NormalTok{cyl, min, }\DataTypeTok{simplify=}\OtherTok{FALSE}\NormalTok{)}
\end{Highlighting}
\end{Shaded}

\begin{verbatim}
## $`4`
## [1] 21.4
## 
## $`6`
## [1] 17.8
## 
## $`8`
## [1] 10.4
\end{verbatim}

\hypertarget{dm}{%
\section{Decision Making}\label{dm}}

\emph{Decicion Making} atau sering disebut sebagai \emph{if then else statement} merupakan bentuk percabagan yang digunakan manakala kita ingin agar program dapat melakukan pengujian terhadap syarat kondisi tertentu. Pada Tabel \ref{tab:percabangan} disajikan daftar percabangan yang digunakan pada \texttt{R}.

\begin{longtable}[]{@{}ll@{}}
\caption{\label{tab:percabangan} Daftar percabangan pada \texttt{R}.}\tabularnewline
\toprule
\begin{minipage}[b]{0.15\columnwidth}\raggedright
\textbf{Statement}\strut
\end{minipage} & \begin{minipage}[b]{0.79\columnwidth}\raggedright
\textbf{Keterangan}\strut
\end{minipage}\tabularnewline
\midrule
\endfirsthead
\toprule
\begin{minipage}[b]{0.15\columnwidth}\raggedright
\textbf{Statement}\strut
\end{minipage} & \begin{minipage}[b]{0.79\columnwidth}\raggedright
\textbf{Keterangan}\strut
\end{minipage}\tabularnewline
\midrule
\endhead
\begin{minipage}[t]{0.15\columnwidth}\raggedright
\emph{if statement}\strut
\end{minipage} & \begin{minipage}[t]{0.79\columnwidth}\raggedright
\emph{if statement} hanya terdiri atas sebuah ekspresi \emph{Boolean}, dan diikuti satu atau lebih \emph{statement}\strut
\end{minipage}\tabularnewline
\begin{minipage}[t]{0.15\columnwidth}\raggedright
\emph{if\ldots{}else statement}\strut
\end{minipage} & \begin{minipage}[t]{0.79\columnwidth}\raggedright
\emph{if else statement} terdiri atas beberapa buah ekspresi \emph{Boolean}. Ekspressi \emph{Boolean} berikutnya akan dijalankan jika ekspresi *Boolan sebelumnya bernilai FALSE\strut
\end{minipage}\tabularnewline
\begin{minipage}[t]{0.15\columnwidth}\raggedright
\emph{switch statement}\strut
\end{minipage} & \begin{minipage}[t]{0.79\columnwidth}\raggedright
\emph{switch statement} digunakan untuk mengevaluasi sebuah variabel beberapa pilihan\strut
\end{minipage}\tabularnewline
\bottomrule
\end{longtable}

\hypertarget{ifstatement}{%
\subsection{if statement}\label{ifstatement}}

Pola \emph{if statement} disajikan pada Gambar \ref{fig:ifstatement}

\begin{figure}

{\centering \includegraphics[width=0.4\linewidth]{./images/ifstatement} 

}

\caption{Diagram if statement (sumber: Primartha, 2018).}\label{fig:ifstatement}
\end{figure}

Berikut adalah contoh penerapan \emph{if statement}:

\begin{Shaded}
\begin{Highlighting}[]
\NormalTok{x <-}\StringTok{ }\KeywordTok{c}\NormalTok{(}\DecValTok{1}\OperatorTok{:}\DecValTok{5}\NormalTok{)}
\ControlFlowTok{if}\NormalTok{(}\KeywordTok{is.vector}\NormalTok{(x))\{}
  \KeywordTok{print}\NormalTok{(}\StringTok{"x adalah sebuah vector"}\NormalTok{)}
\NormalTok{\}}
\end{Highlighting}
\end{Shaded}

\begin{verbatim}
## [1] "x adalah sebuah vector"
\end{verbatim}

\hypertarget{ifelsestatement}{%
\subsection{if else statement}\label{ifelsestatement}}

Pola dari \emph{if else statement} disajikan pada Gambar \ref{fig:ifelse}

\begin{figure}

{\centering \includegraphics[width=0.4\linewidth]{./images/ifelse} 

}

\caption{Diagram if else statement (sumber: Primartha, 2018).}\label{fig:ifelse}
\end{figure}

Berikut adalah contoh penerapan \emph{if else statement}:

\begin{Shaded}
\begin{Highlighting}[]
\NormalTok{x <-}\StringTok{ }\KeywordTok{c}\NormalTok{(}\StringTok{"Andi"}\NormalTok{,}\StringTok{"Iwan"}\NormalTok{, }\StringTok{"Adi"}\NormalTok{)}
\ControlFlowTok{if}\NormalTok{(}\StringTok{"Rina"} \OperatorTok\StringTok{ }\NormalTok{x)\{}
  \KeywordTok{print}\NormalTok{(}\StringTok{"Rina ditemukan"}\NormalTok{)}
\NormalTok{\} }\ControlFlowTok{else} \ControlFlowTok{if}\NormalTok{(}\StringTok{"Adi"} \OperatorTok\StringTok{ }\NormalTok{x)\{}
  \KeywordTok{print}\NormalTok{(}\StringTok{"Adi ditemukan"}\NormalTok{)}
\NormalTok{\} }\ControlFlowTok{else}\NormalTok{\{}
  \KeywordTok{print}\NormalTok{(}\StringTok{"tidak ada yang ditemukan"}\NormalTok{)}
\NormalTok{\}}
\end{Highlighting}
\end{Shaded}

\begin{verbatim}
## [1] "Adi ditemukan"
\end{verbatim}

\hypertarget{switchstatement}{%
\subsection{switch statement}\label{switchstatement}}

Pola dari \emph{switch statement} disajikan pada Gambar \ref{fig:switch}

\begin{figure}

{\centering \includegraphics[width=0.4\linewidth]{./images/switch} 

}

\caption{Diagram switch statement (sumber: Primartha, 2018).}\label{fig:switch}
\end{figure}

Berikut adalah contoh penerapan \emph{switch statement}:

\begin{Shaded}
\begin{Highlighting}[]
\NormalTok{y =}\StringTok{ }\DecValTok{3}

\NormalTok{x =}\StringTok{ }\ControlFlowTok{switch}\NormalTok{(}
\NormalTok{  y,}
  \StringTok{"Selamat Pagi"}\NormalTok{,}
  \StringTok{"Selamat Siang"}\NormalTok{,}
  \StringTok{"Selamat Sore"}\NormalTok{,}
  \StringTok{"Selamat Malam"}
\NormalTok{)}

\KeywordTok{print}\NormalTok{(x)}
\end{Highlighting}
\end{Shaded}

\begin{verbatim}
## [1] "Selamat Sore"
\end{verbatim}

\hypertarget{fungsi}{%
\section{Fungsi}\label{fungsi}}

Fungsi merupakan sekumpulan instruksi atau \emph{statement} yang dapat melakukan tugas khusus. Sebagai contoh fungsi perkalian untuk menyelesaikan operasi perkalian, fungsi pemangkatan hanya untuk operasi pemangkatan, dll.

Pada \texttt{R} terdapat 2 jenis fungsi, yaitu: \emph{build in fuction} dan \emph{user define function}. \emph{build in fnction} merupakan fungsi bawaan \texttt{R} saat pertama kita menginstall \texttt{R}. Contohnya adalah \texttt{mean()}, \texttt{sum()}, \texttt{ls()}, \texttt{rm()}, dll. Sedangkan \emph{user define fuction} merupakan fungsi-fungsi yang dibuat sendiri oleh pengguna.

Fungsi-fungsi buatan pengguna haruslah dideklarasikan (dibuat) terlebih dahulu sebelum dapat dijalankan. Pola pembentukan fungsi adalah sebagai berikut:

\begin{Shaded}
\begin{Highlighting}[]
\NormalTok{function_name <-}\StringTok{ }\ControlFlowTok{function}\NormalTok{(argument_}\DecValTok{1}\NormalTok{, argument_}\DecValTok{2}\NormalTok{, ...)\{}
  \ControlFlowTok{function}\NormalTok{ body}
\NormalTok{\}}
\end{Highlighting}
\end{Shaded}

\begin{quote}
\textbf{Catatan:}

\begin{itemize}
\tightlist
\item
  \textbf{function\_name} : Nama dari fungsi \texttt{R}. \texttt{R} akan menyimpan fungsi tersebut sebagai objek
\item
  \textbf{argument\_1, argument\_2, \ldots{}} : \emph{Argument} bersifat opsional (tidak wajib). \emph{Argument} dapat digunakan untuk memberi inputan kepada fungsi
\item
  \textbf{function body} : Merupakan inti dari fungsi. Fuction body dapat terdiri atas 0 statement (kosong) hingga banyak statement.
\item
  \textbf{return} : Fungsi ada yang memiliki \emph{output} atau \emph{return value} ada juga yang tidak. Jika fungsi memiliki \emph{return value} maka \emph{return value} dapat diproses lebih lanjut
\end{itemize}
\end{quote}

Berikut adalah contoh penerapan \emph{user define function}:

\begin{Shaded}
\begin{Highlighting}[]
\CommentTok{# Fungsi tanpa argument}
\NormalTok{bilang <-}\StringTok{ }\ControlFlowTok{function}\NormalTok{()\{}
  \KeywordTok{print}\NormalTok{(}\StringTok{"Hello World!!"}\NormalTok{)}
\NormalTok{\}}

\CommentTok{# Print}
\KeywordTok{bilang}\NormalTok{()}
\end{Highlighting}
\end{Shaded}

\begin{verbatim}
## [1] "Hello World!!"
\end{verbatim}

\begin{Shaded}
\begin{Highlighting}[]
\CommentTok{# Fungsi dengan argumen}
\NormalTok{tambah <-}\StringTok{ }\ControlFlowTok{function}\NormalTok{(a,b)\{}
  \KeywordTok{print}\NormalTok{(a}\OperatorTok{+}\NormalTok{b)}
\NormalTok{\}}

\CommentTok{# Print}
\KeywordTok{tambah}\NormalTok{(}\DecValTok{5}\NormalTok{,}\DecValTok{3}\NormalTok{)}
\end{Highlighting}
\end{Shaded}

\begin{verbatim}
## [1] 8
\end{verbatim}

\begin{Shaded}
\begin{Highlighting}[]
\CommentTok{# Fungsi dengan return value}
\NormalTok{kali <-}\StringTok{ }\ControlFlowTok{function}\NormalTok{(a,b)\{}
  \KeywordTok{return}\NormalTok{(a}\OperatorTok{*}\NormalTok{b)}
\NormalTok{\}}

\CommentTok{# Print}
\KeywordTok{kali}\NormalTok{(}\DecValTok{4}\NormalTok{,}\DecValTok{3}\NormalTok{)}
\end{Highlighting}
\end{Shaded}

\begin{verbatim}
## [1] 12
\end{verbatim}

\hypertarget{debugging}{%
\section{Debugging}\label{debugging}}

Sering kali fungsi atau sintaks yang kita tulis menghasilkan error sehingga output yang kita harapkan tidak terjadi. \emph{Debugging} merupakan langkah untuk mengecek error yang terjadi. Untuk lebih memahami proses \emph{debugging}, berikut penulis sajikan contoh error pada suatu fungsi dapat terjadi:

\begin{Shaded}
\begin{Highlighting}[]
\NormalTok{f1 <-}\StringTok{ }\ControlFlowTok{function}\NormalTok{(x)\{}
\NormalTok{  xsq <-}\StringTok{ }\NormalTok{x}\OperatorTok{^}\DecValTok{2}
\NormalTok{  xsqminus4 <-}\StringTok{ }\NormalTok{xsq }\OperatorTok{-}\StringTok{ }\DecValTok{4}
  \KeywordTok{print}\NormalTok{(xsqminus4)}
  \KeywordTok{log}\NormalTok{(xsqminus4}\DecValTok{-4}\NormalTok{)}
\NormalTok{\}}

\KeywordTok{f1}\NormalTok{(}\DecValTok{6}\OperatorTok{:}\DecValTok{1}\NormalTok{)}
\end{Highlighting}
\end{Shaded}

\begin{verbatim}
## [1] 32 21 12  5  0 -3
\end{verbatim}

\begin{verbatim}
## Warning in log(xsqminus4 - 4): NaNs produced
\end{verbatim}

\begin{verbatim}
## [1] 3.332205 2.833213 2.079442 0.000000      NaN      NaN
\end{verbatim}

Untuk mengecek error yang terjadi dari sintaks tersebut, kita dapat menggunakan fungsi \texttt{debug()}. Pembaca tinggal memasukkan nama fungsi kedalam fungsi \texttt{debug()}. Fungsi tersebut akan secara otomatis akan menampilkan hasil samping dari pengaplikasian fungsi \texttt{f1()} untuk melihat sumber atau tahapan dimana error mulai muncul.

\begin{Shaded}
\begin{Highlighting}[]
\KeywordTok{debug}\NormalTok{(f1)}
\KeywordTok{f1}\NormalTok{(}\DecValTok{1}\OperatorTok{:}\DecValTok{6}\NormalTok{)}
\end{Highlighting}
\end{Shaded}

\begin{verbatim}
## debugging in: f1(1:6)
## debug at <text>#1: {
##     xsq <- x^2
##     xsqminus4 <- xsq - 4
##     print(xsqminus4)
##     log(xsqminus4 - 4)
## }
## debug at <text>#2: xsq <- x^2
## debug at <text>#3: xsqminus4 <- xsq - 4
## debug at <text>#4: print(xsqminus4)
## [1] -3  0  5 12 21 32
## debug at <text>#5: log(xsqminus4 - 4)
\end{verbatim}

\begin{verbatim}
## Warning in log(xsqminus4 - 4): NaNs produced
\end{verbatim}

\begin{verbatim}
## exiting from: f1(1:6)
\end{verbatim}

\begin{verbatim}
## [1]      NaN      NaN 0.000000 2.079442 2.833213 3.332205
\end{verbatim}

Berdasarkan hasil \emph{debugging}, \texttt{NaN} (\textbf{missing value}) muncul pada tahapan \emph{debug} ke-4 (pembaca dapat melakukan enter terus menerus sampai proses \emph{debug} selesai). Hal ini disebabkan karena terdapat nilai negatif pada objek \texttt{xsqminu4-4} yang selanjutnya dilakukan transformasi logaritmik. Untuk menghentikan proses \emph{debugging} pembaca dapat mengetikkan \texttt{undebug(f1)}.

\hypertarget{referensi-3}{%
\section{Referensi}\label{referensi-3}}

\begin{enumerate}
\def\labelenumi{\arabic{enumi}.}
\tightlist
\item
  Bloomfield, V.A. 2014. \textbf{Using R for Numerical Analysis in Science and Engineering}. CRC Press
\item
  Primartha, R. 2018. \textbf{Belajar Machine Learning Teori dan Praktik}. Penerbit Informatika : Bandung.
\item
  Rosadi,D. 2016. \textbf{Analisis Statistika dengan R}. Gadjah Mada University Press: Yogyakarta.
\end{enumerate}

\hypertarget{numericmethod}{%
\chapter{Pengantar Metode Numerik}\label{numericmethod}}

\emph{Chapter} ini memberikan pengantar bagi pembaca untuk mengenal terlebih dahulu mengenai metode numerik. Pada \emph{chapter} ini akan dibahas mengenai apa itu metode numerik, perbedaannya dengan metode analitik, dan analisis error.

\hypertarget{numericmethodintro}{%
\section{Mengenal Metode Numerik}\label{numericmethodintro}}

Metode numerik merupakan teknik penyelesaian permsalahn yang diformulasikan secara matematis dengan menggunakan operasi hitungan (aritmatik) yaitu operasi tambah, kurang, kali, dan bagi. Metode ini digunakan karena banyak permasalahan matematis tidak dapat diselesaikan menggunakan metode analitik. Jikapun terdapat penyelesaiannya secara analitik, proses penyelesaiaannya sering kali cukup rumit dan memakan banyak waktu sehingga tidak efisien.

Terdapat keuntungan dan kerugian terkait penggunaan metode numerik. Keuntungan dari metode ini antara lain:

\begin{enumerate}
\def\labelenumi{\arabic{enumi}.}
\tightlist
\item
  Solusi persoalan selalu dapat diperoleh.
\item
  Dengan bantuan komputer, perhitungan dapat dilakukan dengan cepat serta hasil yang diperoleh dapat dibuat sedekat mungkin dengan nilai sesungguhnya.
\item
  Tampilan hasil perhitungan dapat disimulasikan.
\end{enumerate}

Adapun kelemahan metode ini antara lain:

\begin{enumerate}
\def\labelenumi{\arabic{enumi}.}
\tightlist
\item
  Nilai yang diperoleh berupa pendekatan atau hampiran.
\item
  Tanpa bantuan komputer, proses perhitungan akan berlangsung lama dan berulang-ulang.
\end{enumerate}

\hypertarget{diffanalitycnumeric}{%
\subsection{Perbedaan Antara Metode Numerik dan Analitik}\label{diffanalitycnumeric}}

Perbedaan antara metode numerik dan metode analitik dapat dijelaskan sebagai berikut:

\begin{enumerate}
\def\labelenumi{\arabic{enumi}.}
\item
  Solusi metode numerik selalu berbentuk angka, sedangkan solusi metode analitik dapat berbentuk fungsi matematik yang selanjutnya dapat dievaluasi untuk menghasilkan nilai dalam bentuk angka.
\item
  Solusi dari metode numerik berupa hampiran, sedangkan metode analitik berupa solusi sejati. Kondisi ini berakibat pada nilai error metode analitik adalah 0, sedangkan metode numerik \(\neq 0\).
\item
  Metode analitik cocok untuk permasalahan dengan model terbatas dan sederhana, sedangkan metode numerik cocok dengan semua jenis permasalahan.
\end{enumerate}

\hypertarget{tahapan-penyelesaian-menggunakan-metode-numerik}{%
\subsection{Tahapan Penyelesaian Menggunakan Metode Numerik}\label{tahapan-penyelesaian-menggunakan-metode-numerik}}

Terdapat beberapa tahapan dalam menyelesaikan suatu permasalahan dengan metode numerik. Tahapan-tahapan tersebut antara lain:

\begin{itemize}
\tightlist
\item
  Pemodelan
\end{itemize}

Persoalan dunia nyata dimodelkan ke dalam persamaan matematika. Persamaan matematika yang terbentuk dapat berupa persamaan linier, non-linier, dan sebagainya sesuai dengan persoalan yang dihadapi.

\begin{itemize}
\tightlist
\item
  Penyederhanaan Model
\end{itemize}

Model matematika yang dihasilkan dari tahap 1 mungkin saja terlalu kompleks. Semakin kompleks suatu model, semakin rumit penyelesaiaannya, sehingga model perlu disederhanakan.

Seberapa sederhana model yang akan kita buat? tergantung pada permasalahan apa yang hendak pembaca selesaikan. Model yang terlalu sederhana akan tidak cocok digunakan untuk digunakan sebagai pendekatan sistem nyata atau lingkungan yang begitu kompleks. Penyederhanaan dapat berupa asumsi sejumlah variabel yang terlibat tidak signifikan, atau asumsi kondisi reaktor (\emph{steady} atau \emph{non-steady}).

\begin{itemize}
\tightlist
\item
  Formulasi Numerik
\end{itemize}

Setelah model matematika sederhana diperoleh, tahap selanjutnya adalah memformulasikan model matematika secara numerik. Tahapan ini terdiri atas:

\begin{verbatim}
+ menentukan metode numerik yang akan dipakai bersama-sama dengan analisis galat (error) awal.
+ menyusun algoritma dari metode numerik yang dipilih.
\end{verbatim}

\begin{itemize}
\tightlist
\item
  Pemrograman
\end{itemize}

Tahap selanjutnya adalah menerjemahkan algoritma ke dalam program komputer. Pada tahapan ini pembaca bisa memilih bahasa pemrograman yang pembaca kuasai.

Dalam buku ini kita hanya akan berfokus pada bahasa pemrograman \texttt{R}. Pembaca dapat menggunakan bahasa pemrograman lain selain dari buku ini. Pembaca hanya perlu memperhatikan bagaimana penulis membangun algoritma penyelesaian dan memtransfernya menjadi bentuk sintaks \texttt{R}. Dari sintaks tersebut pembaca dapat melihat bagaimana meletakakkan tiap tahapan algoritma menjadi sintaks pada bahasa pemrograman.

\begin{itemize}
\tightlist
\item
  Operasional
\end{itemize}

Sebelum digunakan dengan data sesungguhnya, program komputer perlu dilakukan uji coba dengan data simulasi dan dievaluasi hasilnya. jika hasil keluaran diyakini sudah sesuai, baru dioperasikan dengan data yang sesungguhnya.

\begin{itemize}
\tightlist
\item
  Evaluasi
\end{itemize}

Bila program sudah selesai dijalankan dengan data yang sesungguhnya, maka hasil yang diperoleh dilakukan interpretasi, meliputi analisis hasil keluaran dan membandingkannya dengan prinsip dasar dan hasil-hasil empriik untuk menaksir kualitas soluasi numerik termasuk keputusan untuk menjalankan kembali progrma dengan memperoleh hasil yang lebih baik.

\hypertarget{acuracy}{%
\section{Akurasi dan Presisi}\label{acuracy}}

Perhatikan Gambar \ref{fig:akurasi} berikut:

\begin{figure}

{\centering \includegraphics[width=0.75\linewidth]{./images/akurasi} 

}

\caption{4 ilustrasi terkait akurasi dan presisi}\label{fig:akurasi}
\end{figure}

Pada Gambar \ref{fig:akurasi} terdapat 4 buah kondisi ketika kita menembakkan beberapa perluru pada sebuah sasaran. Tujuan kita disini adalah untuk menembak bagian tengah sasaran tersebut.

Pada Gambar (a) dan (c) pada Gambar \ref{fig:akurasi} merupakan gambar yang menunjukkan seseorang telah berhasil mengenai bagian tengah sasaran tersebut dapat kita katakan pula tembakan pada kedua gambar tersebut akurat. Akurat dalam hal ini dapat diartikan suatu kondisi dimana kedekatan lubang peluru dengan pusat sasaran. Secara umum akurasi diartikan sebagai tingkat kedekatan pengukuran kuantitas terhadap nilai sebenarnya.

Terdapat dua buah cara untuk mengukur akurasi. Metode pengukuran akurasi antara lain: error absolut dan error relatif.

Error absolut merupakan nilai absolut dari selisih antara nilai sebenarnya \(x\) dengan nilai observasi \(x'\). Error absolut dapat dituliskan menggunakan Persamaan \eqref{eq:errorabsolut}.

\begin{equation}
   \epsilon_A=\left|x-x'\right|
  \label{eq:errorabsolut}
\end{equation}

Pengukuran lain yang sering digunakan untuk mengukur akurasi adalah error relatif. Berbeda dengan error absolut, error relatif membagi selisih antara nilai sebenarnya \(x\) dan nilai observasi \(x'\) dengan nilai sebenarnya. Hasil yang diperoleh merupakan nilai tanpa satuan. Persamaan error relatif disajikan pada Persamaan \eqref{eq:errorrelatif}.

\begin{equation}
   \epsilon_R=\left|\frac{x-x'}{x}\right|
  \label{eq:errorrelatif}
\end{equation}

Dalam suatu pengukuran, hal lain yang perlu diperhatikan selain akurasi adalah presisi. Presisi adalah sejauh mana pengulangan pengukuran dalam kondisi yang tidak berubah mendapat hasil yang sama. Berdasarkan Gambar \ref{fig:akurasi}, Gambar (a) dan (b) menunjukkan kepresisian yang tinggi. Hal ini terlihat dari jarak antara lubang peluru yang saling berdekatan dan mengelompok.

Berdasarkan Gambar \ref{fig:akurasi} dapat kita simpulkan bahwa dalam suatu sistem pengukuran akan terdapat 4 buah kondisi. Pengukuran akurat dan presisi (Gambar (a)), tidak akurat namun presisi (Gambar (b)), akurat namun tidak presisi (Gambar (c)), dan tidak akurat serta tidak presisi (Gambar (d)).

Dari kondisi-kondisi tersebut, akan meuncul yang dinamakan error. Dalam analisa numerik error atau kesalahan menjadi hal yang perlu diperhatikan.

\hypertarget{numerror}{%
\section{Error Numerik}\label{numerror}}

Kesalahan numerik merupakan error atau kesalahan yang timbul akibat adanya proses pendekatan atau hampiran. Kesalahan numerik terjadi karena tiga hal, antara lain:

\begin{itemize}
\item
  \textbf{Kesalahan bawaan (\emph{inherent error})}, merupakan kesalahan data yang timbul akibat adanya pengkuran, \emph{human error} seperti kesalahan pencatatan, atau tidak memahami hukum-hukum fisik dari data yang diukur.
\item
  \textbf{Kesalahan pembulatan (\emph{round-off error})}, adalah kesalahan yang terjadi karena adanya pembulatan. Contoh: 3,142857143\ldots{} menjadi 3,14.
\item
  \textbf{Kesalahan pemotongan (\emph{truncation error})}, adalah kesalahan yang ditimbulkan pada saat dilakukan pengurangan jumlah angka signifikan.
\end{itemize}

Kesalahan atau error dapat diukur menggunakan Persamaan \eqref{eq:errorabsolut} dan Persamaan \eqref{eq:errorrelatif} yang telah penulis jelaskan pada Chapter \ref{acuracy}.

\hypertarget{referensi-4}{%
\section{Referensi}\label{referensi-4}}

\begin{enumerate}
\def\labelenumi{\arabic{enumi}.}
\tightlist
\item
  Howard, J.P. 2017. \textbf{Computational Methods for Numerical Analysis with R}. CRC Press.
\item
  Sidiq, M. Tanpa Tahun. \textbf{Materi Kuliah Metode Numerik}. Repository Universitas Dian Nuswantoro.
\item
  Subakti, I. 2006. \textbf{Metode Numerik}. Institut Teknologi Sepuluh Nopember.
\item
  Sutarno,H., Rachmatin,D. 2008. \textbf{Hands Out Metode Numerik}. Universitas Pendidikan Indonesia.
\end{enumerate}

\hypertarget{linearaljabar}{%
\chapter{Aljabar Linier}\label{linearaljabar}}

Pada \emph{chapter} ini penulis akan menjelaskan mengenai cara untuk menyelesaikan sistem persamaan linier. Adapun yang akan dibahas pada \emph{chapter} ini antara lain:

\begin{itemize}
\tightlist
\item
  operasi Vektor dan Matriks
\item
  Metode Eliminasi Gauss
\item
  Metode Dekomposisi Matriks
\item
  Studi Kasus
\end{itemize}

\hypertarget{vecmat}{%
\section{Vektor dan Matriks}\label{vecmat}}

Pada Chapter \ref{vector} dan Chapter \ref{matriks} telah dijelaskan sekilas bagaimana cara melakukan operasi pada vektor dan matrik. Pada \emph{chapter} ini, penulis akan menambahkan operasi-operasi lain yang dapat dilakukan pada vektor dan matriks. Dasar-dasar operasi ini selanjutnya akan digunakan sebagai dasar menyusun algoritma penyelesaian sistem persamaan linier.

\hypertarget{operasi-vektor}{%
\subsection{Operasi Vektor}\label{operasi-vektor}}

Misalkan saja diberikan vektor \(u\) dan \(v\) yang ditunjukkan pada Persamaan \eqref{eq:vectoruv}.

\begin{equation}
u = \begin{bmatrix}
      u_1            \\[0.3em]
      u_2            \\[0.3em]
      \vdots         \\[0.3em] 
      u_n
     \end{bmatrix}
dan\ v\ = \begin{bmatrix}
      v_1            \\[0.3em]
      v_2            \\[0.3em]
      \vdots         \\[0.3em] 
      v_n
     \end{bmatrix}
  \label{eq:vectoruv}
\end{equation}

Jika kita menambahkan atau mengurangkan nilai elemen vektor dengan suatu skalar (konstanta yang hanya memiliki besaran), maka operasi penjumlahan/pengurangan akan dilakukan pada setiap elemen vektor.

\begin{equation}
u \pm x = \begin{bmatrix}
      u_1            \\[0.3em]
      u_2            \\[0.3em]
      \vdots         \\[0.3em] 
      u_n
     \end{bmatrix}
\pm x = \begin{bmatrix}
      u_1 \pm x            \\[0.3em]
      u_2 \pm x           \\[0.3em]
      \vdots         \\[0.3em] 
      u_n \pm x
     \end{bmatrix}
     \label{eq:addvector}
\end{equation}

Jika kita melakukan penjumlahan pada vektor \(u\) dan \(v\), maka operasi akan terjadi pada masing-masing elemen dengan indeks yang sama.

\begin{equation}
u \pm v = \begin{bmatrix}
      u_1            \\[0.3em]
      u_2            \\[0.3em]
      \vdots         \\[0.3em] 
      u_n
     \end{bmatrix}
\pm v\ = \begin{bmatrix}
      v_1            \\[0.3em]
      v_2            \\[0.3em]
      \vdots         \\[0.3em] 
      v_n
     \end{bmatrix}
= \begin{bmatrix}
      u_1 \pm v_1            \\[0.3em]
      u_2 \pm v_2           \\[0.3em]
      \vdots         \\[0.3em] 
      u_n \pm v_n
     \end{bmatrix}
     \label{eq:addvector2}
\end{equation}

Untuk lebih memahami operasi tersebut, berikut penulis berikan contoh penerapannya pada \texttt{R}:

\begin{Shaded}
\begin{Highlighting}[]
\NormalTok{u <-}\StringTok{ }\KeywordTok{seq}\NormalTok{(}\DecValTok{1}\NormalTok{,}\DecValTok{5}\NormalTok{)}
\NormalTok{v <-}\StringTok{ }\KeywordTok{seq}\NormalTok{(}\DecValTok{6}\NormalTok{,}\DecValTok{10}\NormalTok{)}

\CommentTok{# penjumlahan}
\NormalTok{u}\OperatorTok{+}\NormalTok{v}
\end{Highlighting}
\end{Shaded}

\begin{verbatim}
## [1]  7  9 11 13 15
\end{verbatim}

\begin{Shaded}
\begin{Highlighting}[]
\CommentTok{# penguranga}
\NormalTok{u}\OperatorTok{-}\NormalTok{v}
\end{Highlighting}
\end{Shaded}

\begin{verbatim}
## [1] -5 -5 -5 -5 -5
\end{verbatim}

Bagaimana jika kita melakukan operasi dua vektor, dimaana salah satu vektor memiliki penjang yang berbeda?. Untuk memnjawab hal tersebut, perhatikan sintaks berikut:

\begin{Shaded}
\begin{Highlighting}[]
\NormalTok{x <-}\StringTok{ }\KeywordTok{seq}\NormalTok{(}\DecValTok{1}\NormalTok{,}\DecValTok{2}\NormalTok{)}
\NormalTok{u}\OperatorTok{+}\NormalTok{x}
\end{Highlighting}
\end{Shaded}

\begin{verbatim}
## Warning in u + x: longer object length is not a multiple of shorter object
## length
\end{verbatim}

\begin{verbatim}
## [1] 2 4 4 6 6
\end{verbatim}

Berdasarkan contoh tersebut, \texttt{R} akan mengeluarkan peringatan yang menunjukkan operasi dilakukan pada vektor dengan panjang berbeda. \texttt{R} akan tetap melakukan perhitungan dengan menjumlahkan kembali vektor \(u\) yang belum dijumlahkan dengan vektor \(x\) sampai seluruh elemen vektor \(u\) dilakukan operasi penjumlahan.

Operasi lain yang dapat dilakukan pada vektor adalah menghitung \emph{inner product} dan panjang vektor. Inner product dihitung menggunakan Persamaan \eqref{eq:innerproduct}.

\begin{equation}
u.v=\sum_{i=1}^nu_1v_1+u_2v_2+\dots+u_nv_n
  \label{eq:innerproduct}
\end{equation}

Panjang vektor atau vektor yang telah dinormalisasi dihitung menggunakan Persamaan \eqref{eq:panjangvektor}

\begin{equation}
\left|u\right|=\sqrt{u_1^2+u_2^2+\dots+u_n^2}
  \label{eq:panjangvektor}
\end{equation}

Berikut adalah contoh bagaimana cara menghitung \texttt{inner\ product} dan panjang vektor menggunakan \texttt{R}:

\begin{Shaded}
\begin{Highlighting}[]
\CommentTok{# inner product}
\NormalTok{u}\OperatorTok\NormalTok{v}
\end{Highlighting}
\end{Shaded}

\begin{verbatim}
##      [,1]
## [1,]  130
\end{verbatim}

\begin{Shaded}
\begin{Highlighting}[]
\CommentTok{# panjang vektor u}
\KeywordTok{sqrt}\NormalTok{(}\KeywordTok{sum}\NormalTok{(u}\OperatorTok{*}\NormalTok{u))}
\end{Highlighting}
\end{Shaded}

\begin{verbatim}
## [1] 7.416198
\end{verbatim}

\hypertarget{operasi-matriks}{%
\subsection{Operasi Matriks}\label{operasi-matriks}}

Misalkan kita memiliki 2 buah matriks \(A\) dan \(B\).

\begin{equation}
A = \begin{bmatrix}
       a_{1.1} & a_{1.2} &\cdots& a_{1.n}           \\[0.3em]
       a_{2.1} & a_{2.2} &\cdots& a_{2.n}           \\[0.3em]
       \vdots  & \vdots  &\ddots& \vdots            \\[0.3em]
       a_{m.1} & a_{m.2} &\cdots& a_{m.n}
     \end{bmatrix}
dan\ B = \begin{bmatrix}
      b_{1.1} & b_{1.2} &\cdots& b_{1.n}           \\[0.3em]
      b_{2.1} & b_{2.2} &\cdots& b_{2.n}           \\[0.3em]
      \vdots  & \vdots  &\ddots& \vdots            \\[0.3em]
      b_{m.1} & b_{m.2} &\cdots& b_{m.n}
     \end{bmatrix}
  \label{eq:matriksuv}
\end{equation}

Jika salah satu matriks tersebut dijumlahkan atau dikurangkan dengan skalar.

\begin{equation}
A \pm x = \begin{bmatrix}
       a_{1.1} & a_{1.2} &\cdots& a_{1.n}           \\[0.3em]
       a_{2.1} & a_{2.2} &\cdots& a_{2.n}           \\[0.3em]
       \vdots  & \vdots  &\ddots& \vdots            \\[0.3em]
       a_{m.1} & a_{m.2} &\cdots& a_{m.n}
     \end{bmatrix}
\pm x = \begin{bmatrix}
      a_{1.1}\pm x & a_{1.2}\pm x &\cdots& a_{1.n}\pm x           \\[0.3em]
      a_{2.1}\pm x & a_{2.2}\pm x &\cdots& a_{2.n}\pm x           \\[0.3em]
      \vdots  & \vdots  &\ddots& \vdots            \\[0.3em]
      a_{m.1}\pm x & a_{m.2}\pm x &\cdots& a_{m.n}\pm x
     \end{bmatrix}
  \label{eq:addmatriks}
\end{equation}

Jika kedua matriks \(A\) dan \(B\) saling dijumlahkan atau dikurangkan. Perlu diperhatikan bahwa penjumlahan dua buah matriks hanya dapat dilakukan pada matriks dengan ukuran yang seragam.

\begin{equation}
A \pm B = \begin{bmatrix}
       a_{1.1} & a_{1.2} &\cdots& a_{1.n}           \\[0.3em]
       a_{2.1} & a_{2.2} &\cdots& a_{2.n}           \\[0.3em]
       \vdots  & \vdots  &\ddots& \vdots            \\[0.3em]
       a_{m.1} & a_{m.2} &\cdots& a_{m.n}
     \end{bmatrix}
\pm \begin{bmatrix}
      b_{1.1} & b_{1.2} &\cdots& b_{1.n}           \\[0.3em]
      b_{2.1} & b_{2.2} &\cdots& b_{2.n}           \\[0.3em]
      \vdots  & \vdots  &\ddots& \vdots            \\[0.3em]
      b_{m.1} & b_{m.2} &\cdots& b_{m.n}
     \end{bmatrix}
= \begin{bmatrix}
       a_{1.1}\pm b_{1.1} & a_{1.2}\pm b_{1.2} &\cdots& a_{1.n}\pm b_{1.n}           \\[0.3em]
       a_{2.1}\pm b_{2.1} & a_{2.2}\pm b_{2.2} &\cdots& a_{2.n}\pm b_{2.n}           \\[0.3em]
       \vdots  & \vdots  &\ddots& \vdots            \\[0.3em]
       a_{m.1}\pm b_{m.1} & a_{m.2}\pm b_{m.2} &\cdots& a_{m.n}\pm b_{m.n}
     \end{bmatrix}
  \label{eq:addmatriks2}
\end{equation}

Untuk lebih memahaminya, berikut disajikan contoh operasi penjumlahan pada matriks:

\begin{Shaded}
\begin{Highlighting}[]
\NormalTok{A <-}\StringTok{ }\KeywordTok{matrix}\NormalTok{(}\DecValTok{1}\OperatorTok{:}\DecValTok{9}\NormalTok{,}\DecValTok{3}\NormalTok{)}
\NormalTok{B <-}\StringTok{ }\KeywordTok{matrix}\NormalTok{(}\DecValTok{10}\OperatorTok{:}\DecValTok{18}\NormalTok{,}\DecValTok{3}\NormalTok{)}
\NormalTok{C <-}\StringTok{ }\KeywordTok{matrix}\NormalTok{(}\DecValTok{1}\OperatorTok{:}\DecValTok{6}\NormalTok{,}\DecValTok{3}\NormalTok{)}

\CommentTok{# penjumlahan dengan skalar}
\NormalTok{A}\OperatorTok{+}\DecValTok{1}
\end{Highlighting}
\end{Shaded}

\begin{verbatim}
##      [,1] [,2] [,3]
## [1,]    2    5    8
## [2,]    3    6    9
## [3,]    4    7   10
\end{verbatim}

\begin{Shaded}
\begin{Highlighting}[]
\CommentTok{# penjumlahan A+B}
\NormalTok{A}\OperatorTok{+}\NormalTok{B}
\end{Highlighting}
\end{Shaded}

\begin{verbatim}
##      [,1] [,2] [,3]
## [1,]   11   17   23
## [2,]   13   19   25
## [3,]   15   21   27
\end{verbatim}

\begin{Shaded}
\begin{Highlighting}[]
\CommentTok{# penjumlahan}
\NormalTok{A}\OperatorTok{+}\NormalTok{C}
\end{Highlighting}
\end{Shaded}

Operasi pehitungan lain yang penting pada matriks adalah operasi perkalian matriks. Perlu diperhatikan bahwa untuk perkalian matriks, jumlah kolom matriks sebelah kiri harus sama dengan jumlah baris pada matriks sebelah kanan. Perkalian antara dua matriks disajikan pada Persamaan \eqref{eq:kalimatriks}.

\begin{equation}
A_{m.n}\times B_{n.r}=AB_{m.r}
  \label{eq:kalimatriks}
\end{equation}

Pada \texttt{R} perkalian matriks dilakukan menggunakan operator \texttt{\%*\%}. Berikut adalah contoh perkalian matriks pada \texttt{R}:

\begin{Shaded}
\begin{Highlighting}[]
\CommentTok{# Perkalian matriks}
\NormalTok{A}\OperatorTok\NormalTok{B}
\end{Highlighting}
\end{Shaded}

\begin{verbatim}
##      [,1] [,2] [,3]
## [1,]  138  174  210
## [2,]  171  216  261
## [3,]  204  258  312
\end{verbatim}

\hypertarget{operasi-dasar-pada-baris-matriks}{%
\section{Operasi Dasar Pada Baris Matriks}\label{operasi-dasar-pada-baris-matriks}}

\bibliography{book.bib,packages.bib}


\end{document}
